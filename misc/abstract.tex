\begin{abstract}

\doublespacing

%Microbubble contrast agents resonate at the acoustic frequencies amenable to diagnostic imaging (\unit{1-5}\mega\hertz),
%and the resulting acoustic response of the bubbles is typically greater than biological tissue.
%This enhanced acoustic response enables diagnostic ultrasound contrast imaging, 
%for micron sized bubbles injected into the blood enhance the acoustic echo from a tissue that is otherwise echo poor.
%However, the use is limited to imaging the vasculature because the size of the bubble is too great to be transported across blood vessels.
%This greatly limits their use,  precluding, for example, targeted imaging where a contrast agent is designed to have affinity to a tumour.

%To lift this restriction one can either try to create a bubble {\em in situ} around a targeted nucleating agent,
%or to better control the size of a bubble within the body so that a bubble small enough to be transported
%can be temporarily grown to a size appropriate for imaging.
%The separation of function between controlling and imaging the bubble is most simply investigated by using two acoustic waves. 
% 
%
This thesis investigates the simultaneous generation and imaging of sub-micron bubbles with the aim of 
having better control over the lifespan and acoustic response of medical ultrasound contrast agents.
Two acoustic waves are used, 
the first  a low frequency (\unit{0.5}\mega\hertz)
wave to generate and manipulate the bubble,
the second a higher frequency (\unit{20}\mega\hertz) wave
to  image the generated bubble by pulse echo.

This thesis considers:
\nlist{
\item the nucleation of the bubble,
\item the influence of the nucleating wave on the high frequency scattering of the bubble,
\item a two wave technique to localise a bubble interacting with both waves.
}
These steps are first developed theoretically and are then tested experimentally.
In the theoretical study it is shown that a low frequency wave can alter the response of a bubble to a higher frequency pulse by altering the bubbles size.
An experimental setup is suggested to maximise this effect for the cloud of bubbles that would be found in diagnostic applications.
The experimental results were suggestive but inconclusive. 
%The lower frequency wave, the cavitating wave,
%pulsates the bubble and induces transitory changes in its response to a higher frequency, imaging pulse.
%The experimental setup investigates a cloud of bubbles - as would be found in diagnostic applications - but is disappointingly inconclusive.

The acoustically measured response of a microbubble to a sound wave is also investigated.
This is done by first considering the process of acoustic measurement more generally.
It is found that the use of pulse-echo demands that \hl{the observables of} acoustic measurement be invariant to the Lorentz group,
with the sound speed being the limiting velocity.
\hl{The models of acoustic measurements are in this sense relativistic}, and such corrections are necessary \hl{when modelling} the bubble wall velocities \hl{measured} in ultrasound contrast imaging.
A Lorentz invariant model for bubble wall motion is derived.

%A second consequence is that  acoustics, when measured acoustically, 
%is an exact analogue of Maxwell's electromagnetism.  
%Sound is a transverse wave of vorticity and Coriolis acceleration and there exists a conserved acoustic current.



%Micron-sized bubbles make highly effective ultrasound contrast agents but are restricted to the blood due to their large size. 
%Smaller bubbles have proven problematic to make due to their short lifetimes.


%One solution to this problem is to generate a bubble in situ with an acoustic pulse.
%The generated bubble 



%In this thesis 

%Perfluorocarbon oil droplets can be made smaller and can be vapourised into a bubble with ultrasound.
%Here we investigate this technique theoretically: first the nucleation and then the imaging.

%The nucleation of  perfluorobutane and perfluoropentane is investigated with classical nucleation theory,
%with a  density functional technique used to evaluate its performance.


%The acoustically measured response of a bubble to a low frequency cavitating wave and a higher frequency imaging wave is then investigated computationally.
%It is shown that the cavitating wave can be used to manimpulate the scattering of the high-frequency wave.

%We also demonstrate that acoustics, when measured with ultrasound, obeys they same relations as electromagnetism.
%A full analogy between these two subjects is provided.
\end{abstract}

%%% Local Variables: 
%%% mode: latex
%%% TeX-master: "../tshorrock_thesis"
%%% End: 
