\begin{abstract}

Micron-sized bubbles make highly effective ultrasound contrast agents but are restricted to the blood due to their large size. 
Smaller bubbles have proven problematic to make due to their short lifetimes.


One solution to this problem is to generate a bubble in situ with an acoustic pulse.
The generated bubble 



In this thesis 

Perfluorocarbon oil droplets can be made smaller and can be vapourised into a bubble with ultrasound.
Here we investigate this technique theoretically: first the nucleation and then the imaging.

The nucleation of  perfluorobutane and perfluoropentane is investigated with classical nucleation theory,
with a  density functional technique used to evaluate its performance.

The acoustically measured response of a microbubble to a sound wave is then investigated.
This is done by first considering the process of acoustic measurement more generally.
It is found that the use of pulse-echo to measure distances and times demands that acoustically measured objects are invariant to the Lorentz group,
with the sound speed being the limiting velocity.
Ultrasound is relativistic theory.
A Lorentz invariant model for the bubble wall motion is derived. 
We claim that it is accurate at high Mach numbers.
The acoustically measured response of a bubble to a low frequency cavitating wave and a higher frequency imaging wave is then investigated computationally.
It is shown that the cavitating wave can be used to manimpulate the scattering of the high-frequency wave.

We also demonstrate that acoustics, when measured with ultrasound, obeys they same relations as electromagnetism.
A full analogy between these two subjects is provided.
\end{abstract}

%%% Local Variables: 
%%% mode: latex
%%% TeX-master: "../tshorrock_thesis"
%%% End: 
