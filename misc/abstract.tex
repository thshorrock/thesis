\begin{abstract}

This thesis investigates the simultaneous generation and imaging of sub-micron bubbles.
Two acoustic waves are used, 
the first  a low frequency (\unit{0.5}\mega\hertz)
wave to generate and manipulate the bubble,
the second at higher frequency (\unit{20}\mega\hertz)
to  image the generated bubble by pulse echo.

This thesis develops the concept by considering
\nlist{
\item the nucleation of the bubble,
\item the influence of the nucleating wave on the high frequency scattering of the bubble,
\item a two wave technique to localise a bubble interacting with both waves.
}
These steps are first developed theoretically and are then tested experimentally.

The acoustically measured response of a microbubble to a sound wave is then investigated.
This is done by first considering the process of acoustic measurement more generally.
It is found that the use of pulse-echo to measure distances and times demands that acoustically measured objects are invariant to the Lorentz group,
with the sound speed being the limiting velocity.
Ultrasound is relativistic theory.
A Lorentz invariant model for the bubble wall motion is derived
that is accurate at high Mach numbers.

A second consequence is that  acoustics, when measured acoustically, 
is an exact analogue of Maxwell's electromagnetism.  
Sound is a transverse wave of vorticity and Coriolis acceleration and their exists a conserved acoustic current.


%Micron-sized bubbles make highly effective ultrasound contrast agents but are restricted to the blood due to their large size. 
%Smaller bubbles have proven problematic to make due to their short lifetimes.


%One solution to this problem is to generate a bubble in situ with an acoustic pulse.
%The generated bubble 



%In this thesis 

%Perfluorocarbon oil droplets can be made smaller and can be vapourised into a bubble with ultrasound.
%Here we investigate this technique theoretically: first the nucleation and then the imaging.

%The nucleation of  perfluorobutane and perfluoropentane is investigated with classical nucleation theory,
%with a  density functional technique used to evaluate its performance.


%The acoustically measured response of a bubble to a low frequency cavitating wave and a higher frequency imaging wave is then investigated computationally.
%It is shown that the cavitating wave can be used to manimpulate the scattering of the high-frequency wave.

%We also demonstrate that acoustics, when measured with ultrasound, obeys they same relations as electromagnetism.
%A full analogy between these two subjects is provided.
\end{abstract}

%%% Local Variables: 
%%% mode: latex
%%% TeX-master: "../tshorrock_thesis"
%%% End: 
