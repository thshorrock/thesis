
\newcommand{\txtincludes}{\ \textrm{includes}\ }
\newcommand{\txtcovers}{ \ \textit{covers}\ }
\newcommand{\statement}[1]{\textit{#1}}
\newcommand{\covered}{\prec}
\newcommand{\aesthetic}{{\ae}sthetic}
\newcommand{\aesthetics}{\aesthetic s}
\newcommand{\telos}{telos} %{\greektext t'elos}
\chapter*{Preface}\label{ch:mysticism}

%There is 

%\begin{quote}
% ``Good Lord,'' they'll scream at you, ``you can't possibly deny that:
% twice two {\em is} four! Never does nature ask you for your opinion;
% she does not care a damn for your wishes, or whether you like her laws
% or not. 
% You are obliged to accept her as she is and, consequently, all her
% results.  A stone wall, that is, is a stone wall ... etc., etc.''
% But, goodness gracious me, what do I care for the laws of nature and
% arithmetic if for some reason or other I don't like those laws of
% twice-two?
% No doubt I shall never be able to break through such a stone wall with
% my forehead, if I really do not possess the strength to do it, but I
% shall not reconcile myself to it just because I have to deal with a
% stone wall and haven't the strength to knock it down. 
%
% \hfill Fyodor Dostoevsky \cite*{DostoevskyQuote}
%\end{quote}

\begin{quote}
  The seeing commit a strange error.  
  They believe that we know the world only through our eyes.
  For my part, I discovered that the universe consists of pressure,
  that every object and every living being reveals itself to us by
  a kind of quiet yet unmistakable pressure that indicates its intention and its form.

 \hfill Jacques Lusseyran \cite*{Lusseyran2006}
\end{quote}

%The preface of Henri \Poincare's {\em Science and Hypothesis} 

%\begin{quote}
% To the superficial observer scientific truth is unassailable,
% the logic of science is infallible;
% and if scientific men sometimes make mistakes,
% it is because they have not understood the rules of the game.
% Mathematical truths are gained from a few self evident propositions,
% by a chain of flawless reasonings; they are imposed not only on us, but on Nature itself.
% By them the Creator is fettered, as it were, and His choice is limited to a relatively small number of solutions...

% ... But upon more mature reflection the position held by hypothesis was seen; it was recognised that it is as necessary to the experimenter as it is to the mathematician.

% \hfill Henri \Poincare \cite*{Science and Hypothesis}
%\end{quote}


%In the course of preparing this document it has become my firm anI have become convinced of Jacques Lusseyran's charge. 
 
%it has become my firm and unweavering conviction that It has become m

%When 
%It is a curious charge
%When we measure the world acoustically 
%

%When beginning a study that measures the world acoustically
%it is natural to first ask whether the world is known the same.




%Th

%The strict objectivity of science broke in the nineteenth 

%t was the development of two branches of mathematics that bro
%In the nineteenth century the 

%\begin{quote}
%  There is a difference between liberty and the purely arbitrary.%
%
% \hfill Henri Poincare \cite*{Science and hypothesis}
%\end{quote}

%\begin{quote}
% To the superficial observer scientific truth is unassailable,
%the logic of science is infallible;
%and if scientific men sometimes make mistakes,
%it is because they have not understood the rules of the game.
%Mathematical truths are gained from a few self evident propositions,
%by a chain of flawless reasonings; they are imposed not only on us, but on Nature itself.
%By them the Creator is fettered, as it were, and His choice is limited to a relatively small number of solutions...
%
% \hfill Henri Poincare \cite*{Science and knowledge}
%\end{quote}





%Objectivity simply does not follow from the existence of the world and
%the possibility of its expression.
%%I can paint a sunflower in front of me,
%%and if I show it to others they may recognise what I have drawn,
%%but they do not understand my sunflower independently of me.
%It is in between the equations that can be written down and their translation into 
%5a conception of the world that there exists the liberty of which we
%have been speaking.








%Physicists say sometimes that they see beauty in their equations,
%but for me, almost without exception I see a melancholy.
%Authors trying to vanish from their page, with any trace of 
%personality carefully scored out in a draft.
%%Why do they hide?  
%A more careful and conscientious amanuensis will not be found.
%But for whom are they writing?  
%Who is dictating the laws of nature to them, 
%but without the need for them, they say,
%the laws that they so carefully write down, 
%carefully deleting any references to their eyes, 
%their hopes and  their loves.
%Why are they so determined to write out of existence their 
%life as if it were so filled with nonsense?
%Where is the beauty in this!


%dictates the world that the physicist should quietly and anonymously
%translate,
%the view here is that beauty is a creation of our own.
%It is an observance and an occupation, an attitude that we take to the world,
%with which we build the world.
%%for I agree with Chandrasekhar that the world is beautiful.

%Our \aesthetic\ is found in what Michel
%\cite*{FoucaultWhatIsEnlightenment}
%calls the `attitude of modernity'. 
%We bring an acceptance that both our person and our personality are
%integral to the interpretation that we have of the world.
%%, and to the
%%observations of the world that we make.
%Of equal importance to the experiments that we carry out upon the world are the
%scars and regrets that the world in turn has marked upon us.
%Our attitude of modernity is characterised by a relationship to the
%present that determines  ``a way of acting and behaving that at one and the same time marks a
%relation of belonging and presents itself as a task''.
%We cannot be satisfied with a relation to the present that seeks just to record,
%note and  archive.  
%No good alone are the transcripts that through dry detail hope to away the
%shock of the fleeting now and the fear of the becoming of age.
%Analysing the historical development of this attitude through the
%writings of Baudelaire, Foucault continues, 
%

In the mid-nineteenth century Baudelaire introduced the term modernity 
to describe the fleeting nature of the existence in the large cities of France.
For Baudelaire the task of an artist was not to just paint the cities' buildings and their people,  
it was equally to capture the ``transitory, the fugitive, the contingent''\cite{BoudelairePainterofModernLife} in these cities.
The artist was to portray the truth of the moment, even if that meant sacrificing the truth of the visual representation.
The sociologist Michel Foucault notes in his essay, {\em What is enlightenment}\cite{FoucaultWhatIsEnlightenment}, that this marked a change in how artists viewed their role:
\begin{quote}
  Baudelairean modernity is an exercise in which extreme attention to
  what is real is confronted with the practise of a liberty that
  simultaneously respects this reality and violates it.
\end{quote}
The artist was not to eliminate their trace from their creation,
but rather was to embrace the freedom
that lies at the interplay between the external world  and the world
that the artist creates.  The modern artist utilises their liberty to ``imagine 
[the world] other than it is ... to transform it not by destroying it but by grasping in it
what it is''. 
The artist is compelled ``to face the task of producing himself''\cite{FoucaultWhatIsEnlightenment}.


At about the same time a similar transformation in attitude was occurring in the sciences.
Henri Poincar{\'e} prefaces {\em Science and Hypothesis}\cite*{Poincare1902} by noting
that only for the ``superficial observer scientific truth is unassailable'',
who retain the view that %
% the logic of science is infallible;
% and if scientific men sometimes make mistakes,
% it is because they have not understood the rules of the game.
% M
``mathematical truths are gained from a few self evident propositions,
 by a chain of flawless reasonings, ... imposed not only on us, but on Nature itself...''
 %By them the Creator is fettered, as it were, and His choice is limited to a relatively small number of solutions...
%
% ... But upon more mature reflection the position held by hypothesis was seen; it was recognised that it is as necessary to the experimenter as it is to the mathematician.
%\end{quote}
In contradistinction Poincar{\'e} and his contemporaries elevated the ``position held by hypothesis'',
not only when validating a model but also in conceiving the world.
Within the hypothesis an abstraction should be made that captures a kernel of truth in a particular subsystem of interactions.
If idealisations have to be brought to the world to bring out this truth then so be it.
% that should
In the hypothesis there is the freedom to create a picture of the world
%The scientist may start with any set of propositions and explore where they lead
and ``it was recognised that [hypothesis] is as necessary to the experimenter as it is to the mathematician''.


%The role of the modern scientist is to make a hypothesis of the world, 
%to make an abstraction that captures a kernel of truth in a particular subsystem of interactions.
%Ignoring the influence of friction should make Newton's laws no less of a triumph.


%To make use of mathematical truths we must hypothesise that their propositions are in fact in accord with the world that is measured.
%
%, as it once had been, to describe the world exactlyx 
%The liberty of the sc
%There is a liberty that is obtained
%Hypothesis gives freedom to the scientist to 
%
%
%
%
%The question as to where this liberty is found is more directly answered
%in asking how the physicist is  made anonymous.
%There are two steps.
%First is the abstraction of the external world into geometry.
%
%The role of hypothesis plays two roles in science.  
%
%
%
%And the most fundamental hypothesis of all is made in the abstraction of the world into geometry.

But the scientist is a part of their picture just as much as the artist is.
For their hypothesise to have meaning they must be tested against the world.
To do so the scientist must define the relation their propositions have to the world by defining their systems of measurement, their notions of time and space,
the location of entities within a measurable geometry.  The scientist is located in the model through these definitions.
%They must state how they are to measure the world. 
And so it is here that we place the charge of Jacques Lusseyran - that the seeing believe that they know the world only through their eyes.
%There are many ways in which the world can be measured.
The charge to us as individuals, that we miss much of what there is to know in the false sagacity of believing what we see, 
applies equally to our scientific models.  
The model and the world we understand as individuals is joined by the choice of measurement system.

Light is certainly the most common experimental probe,
and its use was convenienced by the redefinition of the metre in 1983 with reference to the speed of light\cite{CGPM1984}.
There are, however, many ways in which the world can be measured and in this thesis we use sound. 
One may then ask as to which propositions in science are so rooted in a system of measurement
that revision should be expected when reframed in an acoustic basis.
By Noether's theorem one might hope that the general principles remain the same:
the conservation of energy-momentum, for instance,
from the expectation that the spatio-temporal invariance of the underlying space is unaltered by the measurement system.
But the actual manifestation of the phenomena must surely be different.
The spatio-temporal locations of  events  will not in general be preserved between different notions of time and space that are linked to respective measurement systems.  
%%
%
%  
%Do we really believe only what we see?
%There are many ways to measure the world, and in this thesis we use sound.
%So as not to commit this 
%In this thesis we use sound to measure the world.  
%So as not to unwittingly commit the strange error of the sightedThere are many ways to measure the world, and in this thesis we use sound.  
The world that is seen and the world that is heard form two different pictures of the same underlying phenomenon.
%
%When we speak of the world, we should be careful to be clear 
With Jacques Lusseyran's charge in mind, more time than is usual will be devoted to defining our notions of time and space so that they are consistent with our acoustic measure.


%What can be said of the world if our understanding is so tightly coupled to our prepositions and our system of measurement?


%Presumably, since there is a scientific consensus, the prepositions o 
%What can be said if there is more than one way to measure the Is there a correct way to measure the world.  
%With many ways to measure the world 
%The interdependencies within each world view must be consistent, the same external events are happening no matter how they are measured,
%but which view is true cannot be spoken off.


%But 
%This step is  elegant because in the abstraction we are free to {\em
%define} our relations to external,  
%such as our notions of time and space.
%While our personal sense of the world can be vague, 
%little more perhaps than that we are able to reach out and knock something over, 
%in the abstraction our picture, 
%any picture,
%of the world can sit.
%so long as we play by the definitions we can
%understand a picture of the world,
%independently of  
%whether or not we are the picture's author.
%This is because the abstraction assumes no constancy in the world,
%either in time or in individual. 
%``A proposition is a picture of reality: for if I understand a
%propostion, I know the situation that it represents. And I understand
%the proposition without having had  its sense explained to me''
%\cite[4.021]{WittgensteinBook}. 
%The abstraction is consistent only within itself.

%Being sufficiently removed from {\em any} reality (personal or
%external),
%we are free
%
%Our picture, any picture, of the world can sit in this abstraction,
%and so long as we play by the definitions we can understand
%any other picture.
%This is because the abstraction assumes no constancy in the world,
%either in time or in individual, because it assumes nothing about the world
%other than it be intelligible.
%All we require is that  we must be able to agree that there is a notion of
%time and space, that we can reach out and have something happen.
%It is this by what we mean by intelligible,  
%we do not need to agree as to how any of  these notions manifest themselves.


%But geometry alone, of course, is no good. % our models and propositions of the world,
%``In order to tell whether a picture is true or false we must compare
%it with reality.  It is impossible to tell from the picture alone
%whether it is true or false'' \cite[2.223, 2.224]{WittgensteinBook}.
%It must be related to a reality.
%Moreover, different pictures of the world need to be compared.
%The second step, then, by which physics removes the individual from their
%equations is in the assumption that all realities are the same.
%It is this assumption that is challenged by Jacques Lusseeyran.
%Baudelaire, Nietzsche  and
%Foucault,
%and it is is this assumption that is manifestly wrong.

But the choice of measurement system is not the only place where the scientist enters 
their representation of the world.
Science is inductive%
\footnote{
This premise is  rejected by Karl Popper\cite{Popper2002} on his opening
page:
``According to the widely accepted view - to be opposed in this book - 
the empirical sciences can be characterised by the fact that they use
`inductive methods', as they are called''.
Popper, as a matter of principle, required science to be fully
objective:
``a subjective experience, or a feeling of conviction, can never
justify a scientific statement, and that within science it can play no
part except that of an object of an empirical (a psychological)
enquiry.''
The ``demand that scientific statements must be objective [means that]
those statements which belong to the empirical basis of science must
also be objective''. 
That is, when this criteria of full objectivity is applied to  induction
infinite regress results.  
It is this that led Popper to reject inductive methods.
(We note further that while methods such as the transformational group of 
Jaynes\cite{Jaynes1973a} enable notions of ignorance to be defined in an
objective way, ignorance is still a subjective notion in the sense
used here.)
%
%Here induction is taken not only as the most natural {\em definition}
%of science,
%but also as one of its most useful tools.
%We are in agreement with Popper when he argues that objectivity is lost as part of
%this definition,
%and while our arguments are contrary,
%this is the essence of both 
%points being made.
%However, the lack of objectivity does not upset us, 
%for not only do we find the premise of objectivity politically
%problematic,  we find it scientifically unhelpful.
%And finally,
%away with this Darwinism philosophy,
%it is not the purpose of science to ``select [a theory] which is by
%comparison the fittest, by exposing them all to the fiercest struggle
%for survival''.
%
%We reject that the premise of objectivity is scientifically useful or
%politically 
%
%Firstly because the search for such an objectivity seems to rest on a
%problematic transcendentalism,
%and secondly because the determinism resulting from this full
%objectivity I find monstrous.
}.
Observations enable us to compare different pictures of the world,
allows us to say that one is more likely than another
based on a comparison of how the external world presents itself to us in a given form of measurement.
%rather than from  a priori intuition or prejudice.
While observation can provide evidence in favour or opposed to our
deepest held views,
our resulting conclusions are  not independent of the a priori.
%Our hopes and our dreams and our faith are on an equal footing to the
%evidence of our experiments.
To see this let us write the law of inference
that dictates how probabilities,
our beliefs of the world,
should be shifted in the light of new evidence.
The law of inference is Bayes Theorem and dates back to 
Laplace (see Jaynes\cite{Jaynes1979} for a discussion on its historical development),
and has since been derived from the definition of probability
\cite{Cox1946, Skilling1991}.

Bayes theorem gives how much credence should be given to a hypothesis of
the world, $\H$,
given a piece of experimental data, $D$, and an a priori view, $I$,
\eql{
  P(\H | D, I) = \frac{P(D|\H, I) P(\H|I)}{P(D|I)}.
}{Bayes}
Our notation is that $P(a|b)$ gives the probability of proposition $a$
given the proposition
$b$.
The comma denotes compound propositions which are always evaluated
before the $|$ sign, which separates what is unknown from what is given.


In the numerator on the right-hand-side of equation \eqnref{Bayes} there are the
two terms.
Firstly there is the  probability of the data given the hypothesis and the a
priori information.
This gives the degree to which the experiment agrees with a model that
we have dreamt up.
Equally important to this, however,
is the second term, 
the probability of the hypothesis given our a priori belief.
The data from the experiment demonstrably has no influence
on this term at all.
Rather it gives the degree to which we believe the hypothesis,
and this term is completely personal.
It may be that I dislike the individual who suggests the model and that I
choose to disbelieve them out of spite.
This is not to reject the possibility of a scientific consensus.
With sufficient quantities of data our inference goes through many
iterations and the importance of our a priori beliefs diminish.
To put it another way,
a truly obstinate attitude is required to outweigh the validity of
good models.

In the denominator of equation \eqnref{Bayes} is  the probability of the data given the a priori view,
and this term is a normalisation constraint so that the  probability varies between
0 and 1.
This normalisation does not depend on the hypothesis being made and so
we are free to compare hypothesis' by simply dividing one by the
other, 
the normalisation cancelling,
\eql{
  \frac{P(\H_1 | D, I)}{P(\H_2 | D, I)} = \frac{P(D|\H_1, I) P(\H_1|I)}{P(D|\H_2, I) P(\H_2|I)}.
}{BaysianComparison}
In this way the quantitative degree to which one model should be favoured
over a second is found.

From equation \eqnref{BaysianComparison} we see that the degree of acceptance
of a model is a matter for 
each of us as individuals, for 
the term $P(\H|I)$ does not vanish.
The laws of nature are a personal matter.
How we interpret the  world with which we interact is a task for
us each to perform.
The existence of the world is assumed in our ability to
collect information, $D$.
We cannot wish away misfortune,
but equally the meaning and structure of the external world is for us each
to construct. 
We do create the world, 
and when we die, so does this world. % which is a tragedy.

And so the scientist, through their measurement system, 
defines their interaction with the world.  
It could be with light or with sound or by some other means.  
They construct a hypothesis of how their measurement system interacts with the world,
and update their model based on the results and their initial views.
The scientist can understand only the part of the world with which that they interact.
About the rest they cannot speak.
Wittgenstein\cite{WittgensteinBook}, after  carefully detailing the hierarchy of proportions%
\footnote{
Wittgenstein's hierarchy of propositions shares much with the modern lattice 
formulation of probability theory developed by Knuth\cite{Knuth2005a, Knuth2012}.
For a  comparison between Wittgenstein's {\em Tractus} and lattice theory see Wolniewicz\cite{WolniewiczBook} 
}, and the inferences that can be made of them,
concludes,
\begin{quote}
 There is no such thing as the subject that thinks or entertains
 ideas.

 If I wrote a book called {\em The world as I found it},
 I should have to include a report on my body, and should have to say
 which parts were subordinate to my will,
 and which were not, etc., this being a method of isolating the subject,
 or rather of showing that in an important sense there is no subject;
 for it alone could not be mentioned in that book.

 The subject does not belong to the world: rather, it is the limit of
 the world. 
\end{quote}
And it is in this sense that
 ``The world and life are one, I am my world. (The microcosm)'' \cite{WittgensteinBook}.

The scientist - as manifested in their model's and in their inferences - has hitherto been quite a lonely creature.
I look forward to the day when it is realised that the manifestation of the world is more social.  


% None of this is to dismiss the possibility of a scientific consensus.
% With sufficient quantities of data,
% the importance of our a priori beliefs diminish, 
% or to put it another way,
% a truly vengeful attitude is required to outweigh the validity of
% good models.
% But the point is that the external wourld is never fully revealed to
% us - we are forced to construct our own picture of the world.
% In this we posses a liberty,  each of our understandings of the world
% is manifestly unique.

%But surely this careful distinction between the external world
%(reality) and our picture of the external world is troublesome?
%Have we not just slipped back into the body/mind, physical/spiritual
%dualism that we started from, and rejected?
%There is a difference.
%Physics is  concerned with constructing models,
%finding a set of propositions and defining their relations.
%But this is all done within the arena of the internal.
%These models need to be compared with each other by pitching
%them against the world in an experiment, 
%%``a proposition can be true or false only in virtue of being a picture
%%of reality'' \cite[4.06]{WittgensteinBook},
%but even when we do this \eqnref{Bayes} tells us that the conclusions we draw are solely our own.
%The results of  inference return to the internal.
%With the loss of an absolute objectivity comes the acceptance that
%reality must be at least partially hidden from us.
%%
%%We have been drawing a careful distinction between the external world  (reality) 
%%and our picture of the external.  
%%The distinction is necessary due to the limits of what 
%%is knowable  in
%%the external reality,
%%for if there is no absolute objectivity, reality must be at least
%%partially hidden from us.
%%Indeed, what 
%%can actually be said of the external world.
%%If there is no absolute objectivity, reality must be at least
%%partially hidden from us.
%Whereas before physics was concerned with the body,
%the mind needing a different language,
%we are now sure of our role within the process of inference,
%but are less certain as to the status of reality in physics.
%Where exactly does the external reality enter into our discussion?
%Is it something of which we can speak, 
%or must we ``pass over [it] in silence'' \cite[7]{WittgensteinBook}?


%And the models that we construct 
%On what basis do our models have meaning?
%They 
%certainly have no objective meaning - 
%There is no objective meaning to the models that we construct - 
%it is egotism to say that {\em the} world is like this or that,
%I may only speak of {\em my} world with such assuredness.
%While it is true that I experience and use the existence of an
%external reality, 
%I nevertheless find myself only able to speak about the world that I
%construct.
%My relation to reality is in the degree to which it confirms my
%models,
%but it is only of my models that I can speak.
%This leads Wittgenstein (5.631, 5.632) to conclude that 




















%``Poetry, music, art, the love I have
%for my grandchild. Even if I could, I wouldn't want to weigh and measure
%that,
%or my relationship with my friends, or with the sunset''
%(Pauline \cite{RuddQuote2008}).  %Biologist at UC Dublin
%%``But equally i do want the ideas I formulate about God to be consistent
%%with my knowledge of science.''

%Physicists say sometimes that they see beauty in their equations,
%but for me, almost without exception I see a melancholy.
%Authors trying to vanish from their page, with any trace of 
%personality carefully scored out in a draft.
%%Why do they hide?  
%A more careful and conscientious amanuensis will not be found.
%But for whom are they writing?  
%Who is dictating the laws of nature to them, 
%but without the need for them, they say,
%the laws that they so carefully write down, 
%carefully deleting any references to their eyes, 
%their hopes and  their loves.
%Why are they so determined to write out of existence their 
%life as if it were so filled with nonsense?
%Where is the beauty in this!


%The physicists that say they see beauty in their equations are in love,
%and like all those happy in love they are blessed by the capricious.
%Their muse has the remarkable quality of extending beyond the head
%of its creator to a world outside, a world that is relatable to others.
%It is the relations between the observer and the world,
%and again between the world and another observer, that concern us, %the beauty we are here
%%concerned with is located,
%at the intersection and at the at least partial agreement of these three worlds. 
%Subrahmanyan \cite*{ChandrasekharQuote1975} ``shuddered'' before the ``incredible fact
%that a discovery motivated by a search after the beautiful in mathematics
%should find its exact replica in Nature''.  
%For Chandrasekhar ``beauty is that to which the human mind responds
%at its deepest and most profound''.
%The view here is different.
%Rather than there being an innate and eternal beauty that calls us,
%dictates the world that the physicist should quietly and anonymously
%translate,
%the view here is that beauty is a creation of our own.
%It is an observance and an occupation, an attitude that we take to the world,
%with which we build the world.
%%for I agree with Chandrasekhar that the world is beautiful.

%Our \aesthetic\ is found in what Michel
%\cite*{FoucaultWhatIsEnlightenment}
%calls the `attitude of modernity'. 
%We bring an acceptance that both our person and our personality are
%integral to the interpretation that we have of the world.
%%, and to the
%%observations of the world that we make.
%Of equal importance to the experiments that we carry out upon the world are the
%scars and regrets that the world in turn has marked upon us.
%Our attitude of modernity is characterised by a relationship to the
%present that determines  ``a way of acting and behaving that at one and the same time marks a
%relation of belonging and presents itself as a task''.
%We cannot be satisfied with a relation to the present that seeks just to record,
%note and  archive.  
%No good alone are the transcripts that through dry detail hope to away the
%shock of the fleeting now and the fear of the becoming of age.
%Analysing the historical development of this attitude through the
%writings of Baudelaire, Foucault continues, 
%``Baudelairean modernity is an exercise in which extreme attention to
%what is real is confronted with the practise of a liberty that
%simultaneously respects this reality and violates it''.
%The focus of Baudelaire's gaze was the problem of the artist that
%cannot eliminate themself from their creation.
%Rather than attempting this extrication the modern attitude was to embrace the freedom
%that lies at the interplay between the external world  and the world
%that the artist creates.  The modern artist utilises their liberty to ``imagine 
%[the world] other than it is ... to transform it not by destroying it but by grasping in it
%what it is''. 
%The artist is compelled ``to face the task of producing himself''.

%%It is worth pausing a little longer with the  analysis of
%%Foucault, for
%It is not just artists that sit at the difficult boundary between the world
%that burns us when we touch something hot and the world that we are at
%liberty to, and necessarily, create.
%%The anonymous physicist has an ideology that is silent.

%Historians too
%are presented with a world over which they %seem to 
%have %so 
%little control.
%They are confronted by the immediate and ongoing problem of
%what  is  to be done with this world.
%How  should a historian proceed and by using what methods?
%%And what should be their relationship with their sources?
%%Sometimes they seem to be ``without choice: [for history] 
%%encourages
%%thorough understanding and excludes qualitative judgements - 
%Unlike in the creation of art, historians can be squeamish to look at
%and accept their liberty.  
%Surely it is proper for them to have ``a sensitivity to all things
%without distortion'' (\cite{FoucaultNietzcheGenealogyHistory}).
%They can never let themselves be caught in the drama of the  present,
%with all its incompleteness, petty squabbles and careless errors.
%They need to see how things were and how this leads to how things are,
%and they  need to avoid  the vulgarity that the present can add to a reflection of
%the past.
%Standing in front of the mirror to the past the historian 
%%must make themself invisible. 
%%They 
%must be a ghost casting no reflection, they must be able to look
%through themself, through 
%%their lungs and 
%their heart, through all the detritus of existence to look cleanly at the
%reflection of the past onto  the present.
%Does the subject of
%%sovereignty of the subject of  
%history not demand this humility
%from its historians? ``After all,
%what right have they to impose their tastes and preferences when they
%seek to determine what actually occurred in the past?''.
%Should the historian not err to the ascetic rather than the disposition
%of the self-absorbed?  Should they not do as the physicists do and take
%``unusual pains to erase the elements in their work
%which reveal their grounding in a particular time and place, their
%preferences in a controversy - the unavoidable obstacles of their
%passion.'' 


% with different monuments to the present to those
%that we build now.

%so that the dependence of their view on their eyes can be safely
%neglected.
%It is a rejection that the world the historian creates has only ever a
%fleeting attachment to a particular time, 
%and a rejection of 
%A historian that sees their reflection and recognises
%themselves as the  product and as the author of the creation of now.  They
%would have to recognise  fresh eyes on the world and
%its interactions (eyes from the past, present and future) sees a world
%that is different, not a clearer but altered. 
%Their work would never be done for their attachment to a world of a
%particular time is forever fleeting.


%But to do so would imply that it is possible to step outside of the world, to
%witnesses a thread in ideas that can be traced back to an origin and that
%leads to a culmination, a fulfilment of an idea that gives a history
%its purpose.
%It is a rejection that the world created by the historian is born from a
%local consciousness, a rejection of the endless reassessment that this
%would entail.  
%It is %a nervousness in front of a mirror, %not seen since Madussa,
%a pretence that all eyes see the world the same.
%It is a rejection that we are each the  product, the author and the
%creation of now,
%and a rejection that this local world's attachment to an exteriority is
%only ever fleeting.
%Rather it is a history that views the
%world as one ``not [of] division, but development; not an
%interplay of relations, but an internal dynamic'' (\cite{FoucaultArchaeologyOfKnowledge}).
%More still it assumes a constancy to the idea being traced,
%that our conception of the world can be compared with the understanding
%of others; that my world is your world is the world.
%
%But for \cite*{FoucaultNietzcheGenealogyHistory}, following Nietzsche, 
%``Nothing in man - not even his body - is sufficiently stable to
%serve as the basis for self-recognition or for understanding other
%men.''

%When the historian does as the physicists do we  must recognise  that
%they create a subject that  ``finds its support outside of time and pretends
%to base its judgements on an apocalyptic objective''.  And we must
%recognise the assumptions of this position, we must see that the view is 
% ``only possible, however, because of its belief in eternal truth,
%the immortality of the soul, and the nature of consciousness as always
%identical to itself.''
%And so just as \cite*{FoucaultOrderOfThings} rejects ``the transcendental consciousness ... which
%places its own point of view at the origin of all historicity'', 
%here too we reject the transcendentalism %and metaphysics 
%that underpins
%the anonymous physicist as a witness to the universal.
%We see the ideology in the search for anonymity, and see the  parallels
%to  physics when \cite*{FoucaultNietzcheGenealogyHistory} writes,
%``As the demagogue is obliged to
%invoke truth, laws of essences, and eternal necessity, the historian must
%invoke objectivity, the accuracy of facts, and the permanence of the
%past.  The demagogue denies the body to secure the sovereignty of a
%timeless idea and the historian effaces his proper individuality so
%that others may enter the stage and reclaim their own speech.  He is
%divided against himself: forced to silence his preferences and overcome
%his distaste, to blur his own perspective and replace it with the fiction
%of a universal geometry, to mimic death in order to enter the kingdom of
%the dead, to adopt a faceless anonymity''.



%We relate and find understanding in others in that which is the
%universal.
%The exact notion as to what this universal actually is varies:
%for the anonymous physicist it is reality,
%for the artist it is the \aesthetic\ of the present,
%for the historian %it is the nature of consciousness,
%and for Kierkegaard's pseudonymous author Johannes \cite{KierkegaardFearAndTrembling} 
%``The ethical as such is the universal.''
%However, in all these cases the universal
%``rests immanent in itself, has nothing outside itself that is its 
%\telos\
%but is itself the \telos\ for everything outside itself,
%and when the ethical has absorbed this into itself, it goes not
%further.''
%It is the transandental that all the notions of the universal share,
%and it is this that makes them the same.
%%
%%The expression of the universal is the expression of the transandental, 
%%and for Wittgenstein (6.421) no distinction is therefore to be made
%%between the different understandings of the universal,
%``Ethics is transandental. (Ethics and \aesthetics\ are one and the
% same)''(\cite[6.421]{WittgensteinBook}).

%The universal is needed to find a  common humanity
%to which we can appeal to find inherent
%meaning and a shared objective.
%A humanism\footnote{
%%\begin{quote}
%``[Humanism] is a theme or, rather, a set of themes that have reappeared
%on several occasions, over time, in European societies; these themes,
%always tied to value judgements, have obviously varied greatly in their
%content, as well as in the values they have preserved. Furthermore,
%they have served as a critical principle of differentiation.  In the
%seventeenth century, there was a humanism that presented itself as a
%critique of Christianity or of religion in general; there was a
%Christian humanism opposed to an ascetic and much more theocentric
%humanism.  
%In the nineteenth century, there was a suspicious humanism, hostile
%and critical toward science, and another that, to the contrary, placed
%its hope in that same science.
%Marxism has been a humanism; so have existentialism and personalism;
%there was a time when people supported the humanistic values
%represented by National Socialism, and when the Stalinists themselves
%said they were humanists.
%%
%
%...  and it is\  a fact that, at least since the seventeenth century,
%what is called humanism has always been obliged to lean on certain
%conceptions of man borrowed from religion, science, or politics.
%Humanism serves to colour and to justify the conceptions of man to
%which it is, after all, obliged to take recourse.''
%
%\hfill\cite{FoucaultWhatIsEnlightenment}
%%\end{quote}
%} 
%into which  we can and have a duty to be absorbed.
%That is, 
%``the single individual, sensately and psychically qualified in
%immediacy,
%is the individual who has his \telos\ in the universal,
%and it is his ethical task continually to express himself in this, to
%annul his singularity in order to become the universal''
%(\cite{KierkegaardFearAndTrembling}).
%But we have argued through the examples of the artist, the historian
%and the physicist, that this process is problematic.
%That the single individual, rather than finding their \telos\ in the
%universal 
%must be higher than the universal, that the single individual defines the limits of their world,
%or to put it another way, is their world.
%But at the same time as arguing this we do not and cannot deny the existence
%of the universal.
%There is a present for the artist to capture,
%the is a past for the historian to study,
%and there is an external reality with which to compare models.

%Despite defining the ethical as the universal, and while at no
%point doubting the existence of an ethical course of action, 
%De Silentio  asks whether it is possible for the ethical
%to be teleologically suspended.
%Can there be an expression that is higher than the ethical,
%that cannot be understood as part of the universal, 
%that cannot be spoken of rationally?
%Yes, says De Silento.
%In science there is not a word to describe how the internal
%world, based upon and tied to reality,
%nevertheless comes to find itself in a position more promenant than
%reality.
%In ethics, for De Silento, there is a word named faith. 
%%
%%And in a way reminiscent of Wittgenstein concluding that ``I am my
%%world'', Kierkegaard concludes the affirmative,
%%with faith representing the paradox that can bring the single
%%individual to be greater than the universal:
%%
%%
%%
%%\begin{quote}
%``The paradox of faith, then, is this: that the single individual is
%higher than the universal... 
%%that the single individual - to recall a
%%distinction in dogmatics rather rare these days - determines his
%%relation to the universal by his relation to the absolute, not his
%%relation to the absolute by his relation to the universal.
%The paradox may also be expressed in this way: that there is an
%absolute duty to God, for in this relationship of duty the individual
%relates himself as the single individual absolutely to the
%absolute... 
%%In this connection, to say that it is a duty to love God means
%%something different from the above, for if this duty is absolute, then
%%the ethical is reduced to the relative...
%%
%%Luke 14:16 offers a remarkable teaching on the absolute duty to God: 
%%``If any one comes to me and does not hate his own father and mother
%%and wife and children and brothers and sisters, yes even his own life,
%%he cannot be my discipile.''  This is a hard saying.
%%Who can bear to listen to it?
%%This is the reason, too, that we seldom hear it... 
%%But this silence is only an escape that is of no avail...
%%But how to hate them? ...
%%If I regard the task as a paradox, then I understand it - that is, I
%%understand it in the way one can understand a paradox.
%The absolute duty can lead one to do what ethics would forbid, but it
%can never lead the knight of faith to stop loving.
%Abraham demonstrates this.
%In the moment he is about to sacrifice Isaac, 
%the ethical expression for what he is doing is: he hates Isaac.
%But if he actually hates Isaac,
%he can rest assured that God does not demand this of him,
%for Cain and Abraham are not identical.
%He must love Issac with his whole soul.
%Since God claims Isaac, he must, if possible, love him even more, and
%only then can he sacrifice him, for it is indeed this love for Isaac
%that makes his act a sacrifice by its paradoxical contrast to his love
%for God.
%But the distress and the anxiety in the paradox is that he, 
%humanly speaking,
%is thoroughly incapable of making himself understandable.
%Only in the moment when his act is in absolute contradiction to his
%feelings,
%only then does he sacrifie Isaac, 
%but the reality of his act is that by which he belongs to the
%universal,
%and there he is and remains a murderer."
%%\hfill 
%\cite{KierkegaardFearAndTrembling}
%%\end{quote}

%It is the identification of this paradox against which we keep bumping
%that makes Kierkegaard important for our discussion,
%the existance of a universal that can be refered to but not direcetly expressed.
%For the problem of the artist that must pratice a liberty that
%simultaneously respects reality and violates it,
%is surely the same problem as the historian that must view the past
%through their local present.
%Likewise,
%is the physicist that carries out experiments upon a world about which
%they cannot speak not equally as silent as the single
%individual who, as an act of faith, must suspend the ethical.
%The physicist can speak only of a personal model tested with their eyes,
%the single individual can  speak only of a spiritual trial.
%The physicist can become understandable only in the fiction that their model
%describes a universal reality,
%the single individual becomes understandable only by subverting the
%ethical in the attempt to become a tragic hero.
%%faith itself cannot be mediated into the universal, for thereby it is canceled. 
%``Faith is this paradox, and the single individual simply cannot make
%himself understandable to anyone. People fancy that the single
%individiual can make himself understandable to another single
%inidiviual in the same situation. Such a view would be unthinkable if
%in our day we were not trying in so many ways to sneak slyly into greatness''.
%(\cite{KierkegaardFearAndTrembling}).

%I find myself left  with the view that we are all lonely within our conception of the world.
%That we may share our models and our views, but that it is a task for us all to
%build the world and to decide whether these models have meaning.
%That we cannot speak of an objective absolute against which to understand
%the world of others.
%But are we then to conclude that we are solely a construction of our own making?
%No, for while we cannot speak of reality we cannot deny its existance.
%We have a history, time passes, we age and we die.
%%But ``at death the world does not alter, but comes to an end.'' (\cite[6.431]{WittgensteinBook})

%In this paradox lies our loneliness but also our liberty.
%We cannot understand each other but neither can we explain one another.
%While we cannot doubt that our conception of the world is affected by
%our biology, our connection to the earth, the passage of time and the
%sorrows and joys that the external world brings to us,
%it is also true that we have no  recorse to humanism or laws of essances with which to demark
%boundaries or determine type.
%We are all absolutely loneley.
%But this is not to say that we are alone.
%For
%%And so, perhaps, rather than the universal it is the lonelyness forced
%%by this paradox that unites us.
%as De Silentio says,
%``faith is a marvel, and yet no human being is excluded from it; for
%that which unites all human life is passion, and faith is passion''.


% and cannot be measured and valued

% And in this paradox

% Is what we construct understandable
% but 
% We are lonely within our own conecptions of the world.
% But this is not to say that we are alone.
% For
% %And so, perhaps, rather than the universal it is the lonelyness forced
% %by this paradox that unites us.
% As De Silentio says,
% ``faith is a marvel, and yet no human being is excluded from it; for
% that which unites all human life is passion, and faith is passion''.

% To conclude, %let us return to the accusation of cruelty
% %with which we begun.
% %In particular, 
% we reconsider the rationalisation of  death
% with which we originally took such exception too.
% The method of science is inference, 
% and Wittgenstein is an exponent of this method who offered his view on
% this subject,
% and so with this we finish.
% \begin{quote}
% So too at death the world does not alter, but comes to an end.
% Death is not an event in life: we do not live to experience death.

% If we take eternity to mean not infinite temporal duration but
% timelessness, then eternal life belongs to those who live in the
% present.
% Our life has no end in just the same way in which our visual field has no limits.

% \hfill \cite[6.431, 6.4311]{WittgensteinBook}
% \end{quote}


% But while not rejecting the existance 

% However,
% just as Boudelaire rejects the absolute \aesthetic\ (without rejecting
% the beautiful),
% and Nietzche and Foucault rejects the objective historian (without
% rejecting the existance of the past),
% and just as physicist cannot speak of the reality upon which they
% experiment upon,
% Kierkegaard asks whether it is possible for the e




%  there is a problem for the artist, historian, physicist and
% humanist that we have been repeatedly bumping against.

% Bit on faith having always existed.



% (\cite{KierkegaardFearAndTrembling}).
 

%  can be appealled to to give life a discernable
% meaning and objective 
 



% For the artist it is as.
% The meaning of the universal is different for diff
% We have cast doubt on the notion of an absolute connection to the
% universal,
% and cast doubt on the degree to which the universal can be spoken of.

% And as such, let us return to the artist, the historian, and the physicist
% %While we oppose the view of the objective 
% We assume \cite{KierkegaardFearAndTrembling} is right when he writes
% ``The ethical as such is the universal...
% It rests immanent in itself, has nothing outside itself that is its 
% {\greektext t'elos}
% but is itself the {\greektext t'elos} for everything outside itself,
% and when the ethical has absorbed this into itself, it goes not
% further.''
% This transendental notion of the ethical is in agreement with
% Wittgenstein (6.421).
% ``The single individual, sensately and phychically qualified in
% immediacy,
% is the individual who has his {\greektext t'elos} in the universal,
% and it is his ethical task continually to express himself in this, to
% annul his singularity in order to become the universal.''





%  actually is the status of the external world.

% We take a model to the world and pitch it against the world.
% But this picture of the world is a part of the internal.
% We have complete knowledge of the model.


% Can we persuade ourselves that there is ever any inherent meaning in our
% models?
% Often they seem to work okay
% We cannot even be sure to agree on the 

% The conclusions that we  draw from the experiements are our own, 
% leaving us slightly isolated

% Unable to be sure of the form of the world we construct a model which
% we pitch against the world. 
% The conclusions that we  draw from the experiements are our own, 
% leaving us slightly isolated
 
% Indeed 


% While the external world exists and shapes our knowledge through the
% results of experiment, the external world itself is not known to us.
% %For its intrinsic structure and natural law we can only imagine,
% %its 
% %for this structure exists in the model of reality
% %for the models of these laws are entirely at the anthropomorphic level.
% What is known to us are the models that we have of the world,
% our propositions  about the world and their relations  
% \cite[4.01,5.2]{WittgensteinBook}, our geometry.
% And while again we  stress that we do not refute reality:
% ``reality is compared with propositions;
% a proposition can be true or false only in virtue of being a picture
% of reality'' \cite[4.05,4.06]{WittgensteinBook},
% we must again state that the inferences from such a comparison are a
% personal task for each of us.
% And so in this sense, 
% %in contradistinction to the view often upheld, 
% physics, our models and propositions, lie in the domain of the
% internal.
% The vanishing physicist, the search for 
% ``the lofty origin [that] is no more than `a metaphysical extension which arises
%  from the belief that things are most precious and essential at the moment
%  of birth.' '' (\cite*{FoucaultNietzcheGenealogyHistory} 
% is the representation of the `` modern conception of the world  founded on the illusion
% that the so-called laws of nature are the explanations of natural
% phenomena.
% Thus people today stop at the laws of nature, treating them as
% something inviolable, 
% just as God and Fate were treated in past ages.
% And in fact both are right and both wrong:
% though the view of the ancients is clearer in so far as they have a
% clear and acknowledged terminus, while the modern system tries to make
% it look as if {\em everything} were explained.'' \cite[6.371,
% 6.372]{WittgensteinBook}.

% But are we not back divided?  
% Have we not just moved our schism from between spiritually and mechanics
% to the `internal' and external reality?
% No, for 
% %We come back to the 
% %And so what is the status of the world.
% %note that the status of this uncertain reality as seperable
% %from our internal constructed but definate world is problematic.
% ``
% there is no such thing as the subject that thinks or entertains
% ideas.
% If I wrote a book called {\em The world as I found it},
% I should have to include a report on my body, and should have to say
% which parts were subordinate to my will,
% and which were not, etc., this being a method of isolating the subject,
% or rather of showing that in an important sense there is no subject;
% for it alone could not be mentioned in that book.
% The subject does not belong to the world: rather, it is the limit of
% the world.''
% \cite[5.631, 5.632]{WittgensteinBook}
% And it is in this sense that
%  ``The world and life are one, I am my world. (The microcosm)''
%  \cite[5.621,5.63 ]{WittgensteinBook}.





% \begin{quote}
% The whole modern conception of the world is founded on the illusion
% that the so-called laws of nature are the explanations of natural
% phenomena.

% Thus people today stop at the laws of nature, treating them as
% something inviolable, 
% just as God and Fate were treated in past ages.

% And in fact both are right and both wrong:
% though the view of the ancients is clearer in so far as they have a
% clear and acknowledged terminus, while the modern system tries to make
% it look as if {\em everything} were explained.
% \end{quote}
% \cite[6.371, 6.372]{WittgensteinBook}




%``The world is independent of my will''.
%\cite[6.373]{WittgensteinBook}.





%``I am my world. (The microcosm)''\cite[5.63]{WittgensteinBook}


%``A proposition is a model of reality as we imagine it'',
%and these propositions ``stand in internal relations to eachother'' 
%\cite[4.01, 5.2]{WittgensteinBook}.


%``We use probability only in default of certainty - if our knowledge
%of a fact is not indeed complete, but we do know something about its
%form.
%The structures of propositions stand in internal relations to one
%another'' \cite[5.156, 5.2]{WittgensteinBook}.


%We are born and we die.
%But our knowledge of the external is only ever partial.
%``A proposition is a picture of reality: for if I understand a
%propostion, I know the situation that it represents. And I understand
%the proposition without having had  its sense explained to me ... In
%order to tell whether a picture is true or false we must compare
%it with realitiy.  It is impossible to tell from the picture alone
%whether it is true or false''(4.021, 2.223,2.224)


%``A proposition is a picture of reality.
%A proposition is a model of reality as we imagine it''. (4.01)



%``Reality is compared with propositions'' (4.05)
%``A proposition can be true or false only in virute of beign a picture
%of reality'' (4.06)

%``The structures of propositions stand in internal relations to
%eachother'' (5.2)

%``It is not {\em how} things are in the world that is mystical but
%{\em that} it exists'' (6.44)

%``To view the world sub specie aeterni is to view it as a whole - a
%limited whole.
%Feeling the world as a limited whole - it is this that is mystical'' (6.45)



% ``Faith is this paradox, and the single individual simply cannot make
% himself understandable to anyone. People fancy that the single
% individiual can make himself understandbale to another single
% inidiviual in the same situation. Such a view would be unthinkable if
% in our day we were not trying in so many ways to sneak slyly into greatness''.
% (\cite{KierkegaardFearAndTrembling}).


















% Our discussions  of our internal world have implied a greater
% structure than Bayes theorem (\eqnref{Bayes}) seems to admit.
% Bayes Theorem enables us to update propositions in the light of
% experimental data,
% but how are these propositions  related?
% Some propositions are compound,
% and some cannot be made simpler.
% There exists an order to our propositions that we need to illucidate.

% Our propositions represent a personal consciousness of the world.
% %We take our propositions to the world and compare them with information
% %from our experiments.
% Our personal picture of the world may or may not match the actual
% state of affairs of the external world, but
% it does, nevertheless, represent our personal 
% understanding of the external world,
% or more succinctly put, it represents {\em our world}.
% %As  says
% ``We picture facts to ourselves. A picture is a model of reality'' 
% \cite[2.1, 2.12]{WittgensteinBook}.
% But how does the structure of our propositions match any underlying
% structure of the external world?
% It is this question that  Wittgenstein addresses directly.
% % it is the relations of the propositions that
% %are
% %so vital, for
% ``The fact that the elements of a picture are related to one another
% in a determinate way represents that things are related to one another
% in the same way. Let us call this connection of its elements the
% structure of the picture, and let us call the possibility of this
% structure the pictorial form of the picture.  What a picture must have
% in common with reality, in order to be able to depict it - correctly
% or incorrectly - in the way it does, it its pictorial form'' (2.15, 2.17).
% Bayes Theorem does not immediately admit this structure,
% and so we leave Bayesian inference for the time being until a method
% of returning is found.

% Central to our argument is that  our wold is not certain.
% ``We use probability only in default of certainty - if our knowledge
% of a fact is not indeed complete, but we do know something about its
% form.
% The structures of propositions stand in internal relations to one
% another'' \cite[5.156, 5.2]{WittgensteinBook}.
% We then order our propositions on a partially ordered set%
% \footnote{
% Note that we are not attempting to construct a one-to-one
% correspondence between Wittgenstein's Tractus and the mathematics of
% partially ordered sets.  The degree to which there is a correspondence
% is analysed closely in \cite{WolniewiczBook}.
% We feel, however, that the mathematics of partially ordered sets gives
% a modern mode of expression to much of the spirit of the philosophy of
% Wittgenstein, 
% and that quoting the {\em Tractus} in this context is in no way a
% misrepresentation.
% We say this even though it must be accepted that the link we make with
% probability theory is very different from the notion of probability
% espoused by Wittgenstein in the {\em Tractus}. 
% }.
% To do so we define a notion of inclusion.
% The statement $b \txtincludes a$, denoted $a\le b$,
% implies that the proposition $a$ is entirely satisfied by the
% proposition $b$.
% For example the statement $ b  = \statement{today is a weekday}$
% includes the statement $a = \statement{today is Monday}.$
% The symbol $<$ is used in the obvious way for when the ordering exists
% but the statements are not equal in content.
% Finally, if it is true that $a<b$ and it is also true that there does
% not exist a proposition $x$ such that $a<x<b$ then it is said that $b
% \txtcovers a$, denoted $a\covered b$.
% A partially ordered set (poset) of propositions can be visualised by
% drawing a line between propositions that cover and are covered.
% An example is illustrated in \figref{DaysOfTheWeekPoset}.

% Discuss join and meet here?

% The partially ordered set of \figref{DaysOfTheWeekPoset} indicates the
% structure of the propositions concerning what day of the week it is.
% However, we have not yet made connection to probability theory.
% The connection is most clearly expressed by Kevin \cite{Knuth2005a},
% and we follow his developments.


%  ``The world and life are one'' \cite[5.621]{WittgensteinBook}
% ``I am my world. (The microcosm)''\cite[5.63]{WittgensteinBook}

% \begin{quote}
% There is no such thing as the subject that thinks or entertains
% ideas.

% If I wrote a book called {\em The world as I found it},
% I should have to include a report on my body, and should have to say
% which parts were subordinate to my will,
% and which were not, etc., this being a method of isolating the subject,
% or rather of showing that in an important sense there is no subject;
% for it alone could not be mentioned in that book.

% The subject does not belong to the world: rather, it is the limit of
% the world. 
% \end{quote}
% \cite[5.632, 5.632]{WittgensteinBook}


% \begin{quote}
% The whole modern conception of the world is founded on the illusion
% that the so-called laws of nature are the explanations of natural
% phenomena.

% Thus people today stop at the laws of nature, treating them as
% something inviolable, 
% just as God and Fate were treated in past ages.

% And in fact both are right and both wrong:
% though the view of the ancients is clearer in so far as they have a
% clear and acknowledged terminus, while the modern system tries to make
% it look as if {\em everything} were explained.
% \end{quote}
% \cite[6.371, 6.372]{WittgensteinBook}

% ``The world is independent of my will''.
% \cite[6.373]{WittgensteinBook}.


% \begin{quote}
% So too at death the world does not alter, but comes to an end.

% Death is not an event in life: we do not live to experience death.

% If we take eternity to mean not infinite temporal duration but
% timelessness, then eternal life belongs to those who live in the
% present.
% Our lie has no end in just the say in which our visual field has no limits.
% \end{quote}
% \cite[6.431, 6.4311]{WittgensteinBook}


%What is  to be done

%a model consists of a set of propositions that may or may not match
%the actual state of affairs of the external world.
%they give a set of possible 

%The propositions that we admit and that we reject from a given 

%The true state of 

% However, there is more to the creation of a model of the world than
% this.
% There is a natural order to different propersitions,
% some propersitions cannot be made simpler,
% while others are compound.
% How is a construction of the world to be make.
% Bayes Theorem is not a panacea to everything.

% Okay, 
% so other than give a very detailed description of why I am not a dancer,
% where have we come.
% We have in fact opened a door to a great deal.
% There are many possible pictures of the world,
% and these can be ranked in accord to experiment on the world and our
% disposition towards such models.
% A question worth asking is do the pictures model the world one to one,
% or can multiple pictures model the same world?


% Quotes:


% Kabir \cite{HelminskiQuote1998} writes about the difficulties of translating Rumi, 
% ``to some extent the traslator must allow himself or herself to be
% annihilated in the presence of the master, and at the same time the
% living inspiration must find a form with the greatest contemporary
% poetic potential.'' 


% Sir John \cite{PolkinghorneQuote2008} ways,
% ``Science is great, but it's not the whole story.
% It deals with repeatable experience,
% but we all know that in our personal lives, experiences aren't
% repeatable.
% And you simply couldn't demonstrate how someone is your friend, or what
% music is.''

% John \cite{LennoxQuote2008} says ``This misapprehension that faith is a
% religious thing not involved in science is simply false.
% I see the twoo as belonging together...
% science is limited.  That is no insult to science,
% but as I recently told Richard Dawkings, I could dissect him, run his
% brain through a scanner, reduce im to chemicals and tell a great deal
% about him.  But I'd never get to know him as a person.  For that, he
% must reveal himself to me.''






% While we may be motivated by ``the lofty origin [that] is no more than `a metaphysical extension which arises
% from the belief that things are most precious and essential at the moment
% of birth.' '' (\cite*{FoucaultNietzcheGenealogyHistory}, 
% it is not the 

% Our fundamental
% relations to the externality, such as notions of time and space, 
% are {\em defined}.
% The second is in the comparison of this abstract space to each of our
% own conceptions of the world.
% The first step is very elegant.
% There is nothing metaphysical in a definition
% A large part of the constancy of the world through time, 



% But it is wrong to say that the equations of physics  do not have our
% existance written into them.
% Our existance is central to science in two different respects.
% The first is in how we abstract the world.
% For us to know what another is talking about,
% our individual notions of existance are carefully {\em defined}.
% While it universily accepted that our models are built on an abstraction
% of reality,
% the importance of this step is not sufficiently discussed.

% Is it possible for more than one conseption of realitiy to
% simultaneously describe the same world.
% Yes.
% Not just in an approximation.



% interaction 
% The first is that someone must make 
% measurements on the world.

 
% And for us to know what others are talking about, 
% our world must be exhas been carefully {\em defined}.
% The i
% And our interaction with the 

% for each of us to know what any other is talking about 
% In our equations describing the external world is a carefully formulated
% set of definitions on which we de


% is 

% equations that do not manifestly have our existance written in them.

% So where have we come?
% Can it be that physicists, like historians,  too often
% search for ``the lofty origin [that] is no more than `a metaphysical extension which arises
% from the belief that things are most precious and essential at the moment
% of birth.' '' (\cite*{FoucaultNietzcheGenealogyHistory} quoting
% Nietzsche),
% but be right in this search?
% Can it be that
% physicists, justified 

% Have we come nowhere at all!

% The two arg


% With suitable and agreed definitions for wh


% %If we were to accept, for example,  that notions of freedom, punishment and
% %sexuality are not constants of history, but rather that
% %these concepts are the product of a discourse, if we were to accept this
% %then the need to liberate ourselves from the humanism that drives the search for origin
% %and development and constancy of spirit would be clear.

% But when it comes to the parallel between physics and other subject that
% I have been drawing, 
% namely the recognition of the transience and uniqueness of our conception of the world, 
% explicitly that we create a world and that this world is rooted  to the 
% external world only fleetingly in its present,
% that in the past and in the future the worlds that each of us create
% will be different; 

% when it comes to this parallel can the physicist not
% argue that that their subject is built, to an even greater extent than
% history on the constancy of the past, for, they will say, if they so
% wish they can repeat the experiments of the past, and if sufficient
% measurements are made, will repeat the same observations.  
% Physicists will say that if they clap their hands, the echo will return
% to them from a distant wall just as it did for Newton?
% And of course the sound will return.  
% But physics is a subject that aims to say more than something happened.
% It aims to systematise and find causalities.  
% We no more deny that the sound will return than the historian would deny
% that there is a past.  The external world is not a product of our fancy.
% Each of us create a world, a mirror in which we relate to the external
% world, based on how we relate to the external world.
% It is not the echo of sound that we reject as an ideological,
% rather the eternity of the physical laws derived from them.
% This is not just the familiar `you can only prove a negative' philosophy
% of Karl \cite{Popper2002},
% for Popper believed passionately in an objective and constant world, 
% that could be viewed objectively and constantly.
% In contradistinction to the philosophy of Popper,
% we take physics to be the methodology of drawing inferences.
% Science is exactly the interplay between the world we create
% and the external world, how the external world impinges on us,
% and how we modify our view.  
% But as for the artist, there is still a freedom at the interplay between
% the external world and the world we create, and the liberty makes us free.

% It will be seen that the inferential definition of science is
% incompatible with the transcendental a priori beliefs of Popper.
% Inferences rejection of the objective caused Popper to reject it as
% method of science believing the rejection made it metaphysical.
% His book is important as it is careful study of just how much (a lot)
% methodology and world view must be rejected to keep constancy to
% physics.
% His ideology does not give us liberty to be autonomous,
% takes away methods that are natural useful to us,
% and jar with the methods of other disciplines.



% a fact that he fully accepts, a fact that leads him to reject

% The lack of constancy of pace of scientific theories has been recognised
% by many.  See for example \cite{Jaynes1968} for a physicists perspective
% on the cycle of revolution, a discontinuity in viewpoint,
% followed by a longer period of establishment.

% There are two approaches.  The first that comes more naturally from this
% school of thought raises questions such as


% ``But what if empirical knowledge, at a given time and in a given
% culture, did posses a well-defined regularity?  If the very possibility
% of recording facts, or allowing oneself to be convinced by them, of
% distorting them in traditions or of making purely speculative use of
% them, if this was not at the mercy of chance?''

% ``The history of science ... describes the processes and products of the
% scientific consciousness.  But, on the other hand, it tries to restore
% what eluded that consciousness: the influences that affected it, the
% implicit philosophies that were subjacent to it ... the unconscious of
% science.''

% ``Can a valid history of science be attempted that would retrace from
% beginning to end the whole spontaneous movement of an anonymous body of
% knowledge?''

% ``I should like to know whether the subjects responsible for scientific
% discourse are not determined in their situation, their function, their
% perceptive capacity, and their practical possibilities by conditions that
% dominate and even overwhelm them.''

% ``What conditions (must be filled) ... not to make a discourse coherent
% and true in general, but to give it, at the time when it was written and
% accepted, value and practical application as scientific discourse''

% ``If there is one approach that I do reject, however, it is that
% ... which places its own point of view at the origin of all historicity -
% which, in short, leads to a transcendental consciousness.''

% ``It seems to me that the historical analysis of scientific discourse
% should, in the last resort, be subject, not to a theory of the knowing
% subject, bu rather to a theory of discursive practise.''


% As said in the forward to the English edition of The Order of Things \cite{FoucaultOrderOfThings}


% The other is inference 

%   and establishement of 

% y derived from 
% ``to preserve, against all decenterings, the sovereignty of the subject,
% and of the twin figures of anthropology and humanism''.
% At its heart, at least the definition of physics that we take here, 
% is the process and methodology of drawing inferences.  

% When I talk of the world that each of us create
% I don't for a moment to claim that the external world that 
% There is an external world that impinges on our fancies.



% %
% %``A characteristic of history is to be without choice: it encourages
% %thorough understanding and excludes qualitative judgements - a
% %sensitivity to all things without distortion...
% %Historians argue that this proves their tact and discretion. After all,
% %what right have they to impose their tastes and preferences when they
% %seek to determine what actually occurred in the past?  
% %Their mistake is to exhibit a total lack of taste, the kind of
% %crudeness that becomes smug in the presence of the loftier elements and
% %finds satisfaction in reducing them to size.
% %The historian is insensitive to the most disgusting things''
% %
% %``This demagoguery, of course, must be masked. It must hide its singular
% %malice under the cloak of universal.




% %
% %disagrees:
% %``We believe, in any event, that the body obeys the exclusive laws of
% %physiology and that it escapes the influence of history, but this too is
% %false... Nothing in man - not even his body - is sufficiently stable to
% %serve as the basis for self-recognition or for understanding other
% %men.''

% %
% %
% %``Nietzsche's criticism [of history] ... always questioned the form of
% %history that reintroduces (and always assumes) a suprahistorical
% %perspective: a history whose function is to compose the finally reduced
% %diversity of time into a totality fully closed upon itself ...
% %The historian's history finds its support outside of time and pretends
% %to base its judgements on an apocalyptic objective.
% %this is only possible, however, because of its belief in eternal truth,
% %the immortality of the soul, and the nature of consciousness as always
% %identical to itself.''

% %
% %and criticises the notion of the origin as ``a metaphysical extension
% %which arises from the belief that things are most precious and essential
% %at the moment of birth''.


% \begin{quote}
%  If the history of thought ... could weave, around everything that men
%  say and do, obscure synthesis that anticipate for him, prepare him, and
%  lead him endlessly towards his future, it would provide a privileged
%  shelter for the sovereignty of consciousness.
% \end{quote}

% ``the guarantee that everything that has eluded him may be restored to
% him''

% ``to preserve, against all decenterings, the sovereignty of the subject,
% and of the twin figures of anthropology and humanism''.

% %``a history that would be not division, but development; not an
% %interplay of relations, but an internal dynamic''

% ``what is being bewailed is the `development' that was to provide the
% sovereignty of the consciousness with a safer, less exposed shelter than
% myths, kinship systems, languages, sexuality, or desire''

% %``what is being bewailed, is that ideological use of history by which
% %one tries to restore to man everything that has unceasingly eluded hi for
% %over a hundred years''.

% As said in the Archaeology of Knowledge \cite{FoucaultArchaeologyOfKnowledge}



% ``Why does Nietzsche challenge the pursuit of the origin? ...
% First, because it is an attempt to capture the exact essence of things,
% their purest possibilities, and their carefully protected forms.
% ...

% %The lofty origin is no more than 'a metaphysical extension which arises
% %from the belief that things are most precious and essential at the moment
% %of birth.' 

% ...

% The final postulate of the origin is inked to the first two in being the
% site of truth.
% From the vantage point of an absolute distance, free from the restraints
% of positive knowledge, the origin makes possible a field of knowledge
% whose function is to recover it, but always in a false recognition due
% to the excess of its own speech.  The origin lies a a place of
% inevitable lose, ...
% the site of a fleeting articulation that discourse has obscured and
% finally lost.''






% %``Historians take unusual pains to erase the elements in their work
% %which reveal their grounding in a particular time and place, their
% %preferences in a controversy, - the unavoidable obstacles of their
% %passion.  Nietzsche's version of historical sense is explicit in its
% %perspective and acknowledges its  system of injustice.  
% Its perception
% is slanted ... it is not given to a discreet effacement before the
% objects it observes and does not submit itself to their processes and
% nor does it see laws, since it gives equal weight to its own sight and to
% its objects.''

% %``A characteristic of history is to be without choice: it encourages
% %thorough understanding and excludes qualitative judgements - a
% %sensitivity to all things without distortion...
% %Historians argue that this proves their tact and discretion. After all,
% %what right have they to impose their tastes and preferences when they
% %seek to determine what actually occurred in the past?  
% %Their mistake is to exhibit a total lack of taste, the kind of
% %crudeness that becomes smug in the presence of the loftier elements and
% %finds satisfaction in reducing them to size.
% %The historian is insensitive to the most disgusting things''%

% %``This demagoguery, of course, must be masked. It must hide its singular
% %malice under the cloak of universal. As the demagogue is obliged to
% %invoke truth, laws of essences, and eternal necessity, the historian must
% %invoke objectivity, the accuracy of acts, and the permanence of the
% %past.  The demagogue denies the body to secure the sovereignty of a
% %timeless idea and the historian effaces his proper individuality so
% %that others may enter the stage and reclaim their own speech.  He is
% %divided against himself: forced to silence his preferences and overcome
% %his distaste, to bur his own perspective an replace it with the fiction
% %of a universal geometry, to mimic death in order to enter the kingdom of
% %the dead, to adopt a faceless anonymity''



% As said in Nietzsche, Genealogy, History \cite{FoucaultNietzcheGenealogyHistory}


% %It is in this final question that the historian places themselves within
% %a tradition.
% %defines what their history means


% %In the choice of method

% %solution, like the artist, 
% %is to turn their problem into a task.
% %The relation 

% %not simply because  discussed,

% %for they too, just as artists, % and physicists,
% %sit  at the difficult boundary between the world
% %that burns us when we touch something hot and a world that we are at liberty to, and necessarily, create.
% %Foucault's method is genealogical.
% %In arguing his approach and placing it within its  contexsetting out his method \cite{FoucaultNietzcheGenealogyHistory}



% %``The world is my world''



% ``It is not {\em how} things are in the world that is mystical but
% {\em that} it exists'' (6.44)

% ``To view the world sub specie aeterni is to view it as a whole - a
% limited whole.
% Feeling the world as a limited whole - it is this that is mystical'' (6.45)




% ``It is obvious that an imagined world, however different it may be
% from the real one, must have something - a form - in common with it''
% (2.022)


% %This attitude is not restricted to the artist, of course.  
% %authorship and Nietchze historians.

% %We shall see that for the physicist the problematic is the the same.
% %The is attitude must be the same.
% %The physicist is as integral to creation of their world as is an artist
% %to the creation of theirs.

  
% %The interplay between the faithful representation of the world and its
% %transformation sot that  ``individual
% %objects appear 'endowed with an impulsive life like the soul of [their]
% %creator'''.
% %The painter's dilema is same as the physicist.
% %Both start with an external world and both then translate it into a
% %world of their own creation.
% %Both need this world to be meaningful to others, and when successful in
% %this aim, both creations are beautiful.

% %``Modern man, for Baudelaire, is not the man who goes off to discover
% %himself, his secrets and his hidden truth; he is the man who tries to
% %invent himself.  This modernity does not 'liberate man in his own
% %being'; it compels him to face the task of producing himself''.

% %Foucoult on Baudelaire comes here


%%% Local Variables: 
%%% mode: latex
%%% TeX-master: "../tshorrock_thesis"
%%% End: 
