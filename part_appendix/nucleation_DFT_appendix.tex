\chapter{Density Functional Theory theory}\label{app:DFT}


\subsection{Introduction}

\Dft\ relaxes the capillary approximation used in \cnt.
The density of the nucleated bubble is not assumed to be that of the bulk,
and the interface is not assumed to be macroscopic and plainer\cite{Oxtoby1992, Oxtoby1998}.
\Dft\ therefore does a much better job at modelling the interface than \cnt.
Rather than it being a sudden boundary,
there is a finite interval over which the density varies from that of the fluid to that of the vapour
and the bubble is modelled for what it is -  a fluctuation in density - 
rather than a vapour entrapped in  flexible boundary.

If  spherical symmetry is assumed then 
the bubble boundary is defined by its radius.
The critical radius is such that\cite{Oxtoby1992,Oxtoby1998}
\begin{align}
  \frac{d \Omega}{d a} =0,\quad\text{at $a = \astar$} \label{eqn:DFT:astarR}
\end{align}
where $\Omega$ is the {\em grand potential}.


The grand potential in \eqnref{DFT:astar} is difficult to evaluate, however,
as it is a functional of the 
phase space positions of all $N$ molecules in the system.
Specifically,  
\begin{align}
  \Omega = -\beta^{-1}\ln \Xi.
\end{align}
where $\Xi$ is the grand partition function
\begin{align}
  \Xi = \Tr \exp\lr{-\beta \lr{H_N - \mu N}}. \label{eqn:nuc:GPF}
\end{align}
 $\Tr$ denotes the trace operator
\begin{align}
  \Tr \equiv \sum_{N=0}^\infty \frac{1}{h^{3N}N!} \iint d\cx_1 d\cp_1
\end{align}
and we have compacted the integral by writing 
\begin{align}
d\cx_n &\equiv dr_n dr_{n+1}\ldots dr_N, &&\quad\text{and}&
d\cp_n &\equiv  dp_n dp_{n+1}\ldots dp_N.
\label{eqn:dshorthand}
\end{align}
In \eqnref{nuc:GPF} $\mu$ denotes the chemical potential and $\H$ denotes the Hamiltonian of the molecules.

The difficulty in evaluating $\Omega$ comes from the interactions between the  molecules.
In order to consider the couplings explicitly we split $\H$ into 
\sub{
\begin{align}
%  \begin{array}{ll}
  \KE &= \sum_i^N \frac{p_i^2}{2m}, && \text{which is the kinetic energy,}\label{eqn:nuc:Kinetic}\\
  \UE &= \UE(\cx_1), && \text{ the internal energy and}\\
  \VE &= \sum_i^N V_\ext(r_i) && \text{the external potential.}
%  \end{array}
\end{align}
}
so that the overall Hamiltonian can be written %in terms of the intrinsic potentials, $\H_\in$ and external potential, $\H_\ext$,
\begin{align}
  \H =  %\H_\in + \H_\ext = 
\KE + \UE + \VE.
\end{align}
Here were have extended the shorthand  employed in  \eqnref{dshorthand} so that 
\begin{align}
\cx_n \equiv r_n,r_{n+1},\ldots,r_N,  \quad\text{and}\quad
\cp_n \equiv  p_n,p_{n+1},\ldots, p_N.
\end{align}
%The coupled terms are therefore the internal energy,  $\UE$.
Separating the Hamiltonian in this way lets us split the grand partition function
\begin{align}
  \Xi = \Tr e^{-\beta (\KE -\mu N)}e^{-\beta(\UE +\VE)}=  \frac{1}{N!}Z_\KE Z_{\UE+\VE}, \label{eqn:XiSeparate}
\end{align}
with the second equality following because $\KE$ is a function of only the particle momentums,
and $\UE$ and $\VE$ are functions of positions.
%At equilibrium the joint probability density of the distribution is the Boltzmann distribution 
%\begin{align}
%  p_0(\cx_1, \cp_1) = \Xi^{-1} \exp\lr{-\beta \lr{H_N - \mu N}}.
%\end{align}
%which can be derived with the Maximum entropy principle\cite{}, for example.
%Here were have extended our shorthand  of \eqnref{dshorthand} so that 
%\begin{align}
%\cx_n \equiv r_n,r_{n+1},\ldots,r_N,  \quad\text{and}\quad
%\cp_n \equiv  p_n,p_{n+1},\ldots, p_N.
%\end{align}
The two functions on the right of in \eqnref{XiSeparate}  can be considered separately:
\nlist{
\item The momentum integrals in \eqnref{XiSeparate} form the partition function of an ideal gas,
with 
\begin{align}
  Z_\KE = \int d \cp_1 e^{-\beta \lr{\sum_i^N \frac{p_i^2}{2m}-\mu N}} = \lr{\frac{m}{2\pi\hbar^2 \beta}}^{3N/2} \equiv n_Q^N.
\end{align}
The term $n_Q$ is sometimes known as the {\em quantum concentration} and is related to the {\em thermal de Broglie wavelength}, $\lambda_T$ by $n_Q = 1/\lambda_T$.
We demote the derivation of this standard result  to \appref{DFT}.
\item
  The remaining $\frac{1}{N! Z_{\UE+\VE}}$ is the  partition function to the joint probability distribution of the  molecular positions,
  \begin{align}
    p_0(\cx) = \frac{1}{N!Z_{\UE+\VE} } e^{-\beta\lr{\UE+\VE}}
  \end{align}
  To make progress  we must  approximate the coupled interaction term, $\UE$.
%so that \eqnref{XiSeparate} can be solved.
Here we assume that only the two particle interactions are important
and write
\begin{align}
  \UE(\cx) \approx \Phi(\cx) =  \sum_{j>i} \sum_i^N \phi(\vr_i, \vr_j),
\end{align}
where $\phi(\vr_i, \vr_j)$ is the two particle potential between a particle at $r_i$ and $r_j$.
%At equilibrium the true joint probability density is the Boltzmann distribution 
%\begin{align}
%  p_0(\cx_1, \cp_1) &= \Xi^{-1} \exp\lr{-\beta \lr{\H_N - \mu N}} 
%\end{align}
%
The approximate Hamiltonian is then $H \equiv \KE + \Phi + \VE$,
and is described by the approximate  probability density, $p$,
  \begin{align}
    p(\cx) = \frac{1}{N!Z_{\UE+\VE} } e^{-\beta\sum_{j>i} \sum_i^N \phi(\vr_i, \vr_j) -\beta\sum_i^N V_\ext(\vr_i)} \label{eqn:pspatial}
  \end{align}
  Marginalising equation \eqnref{pspatial} for the  1-particle distribution gives
  \begin{align}
    p^{(1)}(\vr_1) = \frac{N}{N! Z_{\UE+\VE}} \int e^{-\beta\sum_i^N V_\ext(\vr_i)} d\cx_2. \label{eqn:ponespatial}
  \end{align}
  The 2-particle density is 
  \begin{align}
    p^{(2)}(\vr_1, \vr_2) = \frac{N(N-1)}{N!Z_{\UE+\VE}}\int e^{-\beta\sum_{j>i} \sum_i^N \phi(\vr_i, \vr_j)-\beta\sum_i^N V_\ext(\vr_i)} d\cx_3.
  \end{align}
  The approximate number density, $\rho(\vr)$, is a such that
  \begin{align}
    \int \rho_0(\vr) d\vr = N .
  \end{align}
  It follows that
  \begin{align}
    \rho(\vr) = N!  p^{(1)}(\vr_1). \label{eqn:rhoone}
  \end{align}
  From \eqnref{rhoone} and \eqnref{ponespatial} we find that the {\em density is functional of the external potential.}
}

 The converse is also true:
 {\em the external potential is uniquely determined by the density},
 a result known as the Hohenberg-Kohn theorem.
 The probability density is then determined by the external potential,
 from which it follows that the probability density is a unique functional of the density.
 We outline a proof of the Hohenberg-Kohn theorem in \appref{Hohenberg_Kohn}.
 

It is thereby permissible to work with the mass density rather than the probability density when considering the thermodynamics of the bubble.
Since the density is the term of interest bubble nucleation, the density functional approach is much more direct.



%Re-expressing the grand potential as a functional of mass density rather than probability density
%does not get us any closer to being able to evaluate $\Omega$, however.
%So far the argument is standard from statistical physics.
%To make progress  we must  approximate the coupled interaction term, $\UE$.
%so that \eqnref{XiSeparate} can be solved.
%Here we assume that only the two particle interactions are important
%and write
%\begin{align}
%%  \UE(\cx) \approx \Phi(\cx) =  \sum_{j>i} \sum_i^N \phi(\vr_i, \vr_j),
%\end{align}
%where $\phi(\vr_i, \vr_j)$ is the two particle potential between a particle at $r_i$ and $r_j$.
%At equilibrium the true joint probability density is the Boltzmann distribution 
%\begin{align}
%  p_0(\cx_1, \cp_1) &= \Xi^{-1} \exp\lr{-\beta \lr{\H_N - \mu N}} 
%\end{align}
%
%The approximate Hamiltonian is then $H \equiv \KE + \Phi + \VE$,
%and is described by the approximate  probability density, $p$.
The approximate density function is $\rho$, which defines an approximate grand potential $\Omega_V\lrs{\rho}$.
The task is then to find the distribution $\rho$ that comes closest to approximating $\rho_0$.

The {\em relative entropy } or {\em Kullback-Leibler divergance} gives the amount of information lost 
when using the approximate distribution  $p$ rather than the correct distribution $p_0$,
and is defined
\begin{align}
  \KLD{p}{p_0} = \Tr p \log \frac{p}{p_0} \label{eqn:nuc:KLD}
\end{align}
$\KLD{p}{p_0} \ge 0$, which follows from Gibbs inequality, and only if $p=p_0$ does $\KLD{p}{p_0} = 0$.
We may therefore define
\begin{align}
 \Omega_V\lrs{\rho} \equiv \beta^{-1}\KLD{p}{p_0}+   \Omega\lrs{\rho_0}, 
\end{align}
The approximate  grand potential approaches the true value when 
it vanishes  with respect to $\rho$.
 $\Omega_V\lrs{\rho} $ will then be at thermodynamic equilibrium
which occurs at the critical radius.
Therefore,
condition \eqnref{}
may be expressed\cite{Oxtoby1992}
\begin{align}
  \frac{\delta \Omega_V}{\delta \rho} =0,\quad\text{at $\rho = \rhostar$.} \label{eqn:DFT:astar}
\end{align}

More generally, from \eqnref{} we have
\begin{align}
  \Omega_V\lrs{p} &= \beta^{-1} \Tr p \log \frac{p}{p_0}  - \beta^{-1}\ln \Xi \\
  &= \beta^{-1} \Tr p \log \frac{p}{e^{-\beta\lr{\H - \mu N}}}\\
%  &= T S +    \Tr p \lr{H_N - \mu N} \\
  &= - T S_p + \H_p - \mu N_p \\
  &= F_p - \mu N_p\\
 &= \F + \int V_\ext d\rho - \int \mu d \rho.
\end{align}
where the subscript $p$ indicates an average with respect to the distribution $p$ so that  $S_p$ is the entropy with respect to $p$,
\begin{align}
 \F\lrs {\rho_0} = \Tr p \log\frac{p}{e^{-\beta H_\in}} = \KE + \Phi - TS_p
\end{align}
and the labels `$\in$' and `$ext$' indicates the intrinsic and external parts of the  Hamiltonian.

The energy $\Phi$ at when  $\rho = \rhostar$ (thermodynamic equilibrium)
may be evaluated from by minimising $\Phi\lrs{\phi}$,
\begin{align}
\frac{\delta \Phi}{\delta \phi(r_i, r_j) } &= - \beta^{-1} \frac{\delta \ln Z_{\Phi+\VE}}{\delta \phi(r_i, r_j)} \\
&= \frac{N(N-1)}{2 Z_{\Phi+\VE}}\int d\cx_1 \phi(\vr_1,\vr_2) e^{-\beta\sum_{j>i} \sum_i^N \phi(\vr_i, \vr_j)-\beta\sum_i^N V_\ext(\vr_i)} \\
&= \half \iint d\vr_1 \dr_2 \phi p^{(2)}(\vr_1,\vr_2)
\end{align}
%\begin{align}
%  &= \frac{1}{N!}\frac{e^{-\beta (\KE -\mu N)}}{Z_\KE}\frac{e^{-\beta(\UE +\VE)}}{Z_{\UE+\VE}}  \label{eqn:Jointpzero} %\equiv p_\KE p_{\lr{\UE + VE}}
%\end{align}


%and introduce radial distribution function
%\begin{align}
%  g(r_{12}) = \frac{V^2}{N^2} P_2(r_1, r_2)
%\end{align}


%Since $\UE$ is a measured quantity, it is averaged equilibrium joint probability distribution, $p_0$,
%where $p_0$ is a Boltzmann distribution,
%\begin{align}
% p_0(\cx_1, \cp_1) = \Xi^{-1} \exp\lr{-\beta \lr{H_N - \mu N}}. \label{nuc:pzero}
%\end{align}
%Decoupling the interactions in $\H$ implies that the approximate Hamiltonian, $H$, is described by some likewise decoupled approximate probability distribution, $p$.
%The task is then to vary $p$ so that it matches $p_0$ as closely as possible. %, given its new structure, so that $H$ approaches $\H$.
%From \eqnref{nuc:pzero} $p_0$ is explicitly a function of $\V$.


% This is the 
%  When considering the thermodynamics of the bubble

% The density functional approach is entirely analogous to this next step  but works directly with an approximate mass density, $\rho$, rather than the approximate probability density, $p$.
% The mass density is then varied directly to find the functional form that best matches the equilibrium density, $\rho_0$, and whence the equilibrium Hamiltonian, $\H$.
% Since the density is the term of interest bubble nucleation, the density functional approach is much more direct.
% \Dft\ works at all because
% \nlist{
%   \item 
%     The mass density is a functional of the probability density, $\rho = \rho[p]$.
%     This follows almost trivially from the fact that the equilibrium density is a measured quantity,
%     and therefore an average over $p_0$.
%     Denoting the average
%     \begin{align}
%       \rho\lrs{p} =  \scalar{\rho(\cx_1, \cp_1) }_{p} \equiv \Tr p(\cx_1, \cp_1) \rho(\cx_1, \cp_1), 
%     \end{align}
%     we find that $\rho\lrs{p_0} = \scalar{\rho(\cx_1, \cp_1) }_{p_0}  = \rho_0$.

%     Since the density is a functional of $p_0$, which is in turn a function of $\VE$,
%     it follows that the {\em density is functional of the external potential.}
%   \item
%     The converse is also true:
%     the external potential is uniquely determined by the density,
%     a result known as the Hohenberg-Kohn theorem.
%     The probability density is then determined by the external potential,
%     from which it follows that the probability density is a unique functional of the density.
%     We outline a proof of the Hohenberg-Kohn theorem in \appref{Hohenberg_Kohn}.
%     %It is deferred to the appendix because the proof is not constructive.
%   }
% While many approximations can be made to the internal energy,
% we here consider only  

% A number of approximat

% The approximate Hamiltonian we there

%which we denote
%\begin{align}
%  \UE = \scalar{U(\cx)}_{\rho_0} \equiv Tr  p_0(\cx_1, \cp_1) \rho_0(\cx_1, \cp_1)
%\end{align}
%over the full joint distribution of 

\subsection{Bubble Nucleation}




But density $\rho(r)$ should not be constrained other than to require that it approach the bulk vapor density at large distance.
Then 
\begin{align}
  \frac{\delta \Omega_V}{\delta \rho(r)} = 0
\end{align}
at $\rho(r) = \rho^\ast(r)$.
The mulitdimensainal free energy has a minium at the uniform vapor pressure,
and a second lower minimum at the  uniform liquid density. Between these saddle point sound by setting the funciional deriate to zero.
The matirx of second deriatives containes a negative eigenvalue correspionding to deirection of motion over the barrier.

For equilibrium gas-liquid interface similar except zero eigenvalue not negative. 

sadle point in fucntional space refs in \cite{shen2003}

Density in bubble sufiently far from coexistence differs appreciably from that of stable vapor,
at least order of magnitude ref 28 in \cite{Shen2003}
Agrees well with energy brarrier in vicinity of phase coexistance but does vanish at spinodal.

Nucleation theorem ref 74 in \cite{shen2003}

Have
\begin{align}
\Omega_V = F - \mu N = F - \mu \int dr \rho(r).
\end{align}
Then \begin{align}
\frac{\delta F}{\delta \rho(r)} = \mu
\end{align}
at $\rho(r) = \rho^\ast(r)$.


To make further progress we need to write the grand potential $\Omega$ as a function of the density.

\subsection{Background}
To do so we consider Hamiltonians that are separated in terms of their intrinsic and external contributions
\begin{align}
  \H =  \H_\in + \H_\ext = \lr{\KE + \UE} + \VE
\end{align}
where 
\sub{
\begin{align}
%  \begin{array}{ll}
  \KE &= \sum_i^N \frac{p_i^2}{2m} && \text{is the kinetic energy,}\label{eqn:nuc:Kinetic}\\
  \UE &= U(\cx) && \text{is the internal energy, and}\\
  \VE &= \sum_i^N V_\ext(r_i) && \text{is the external potential.}
%  \end{array}
\end{align}
}
The internal energy depends upon the  locations of the particles, which are denoted  with
\sub{
\begin{align}
\cx_n &\equiv r_n,r_{n+1},\ldots,r_N,  \quad\text{the set of $N-n+1$ particle spatial positions.}
\intertext{Similarly, the momentums of the particles are denoted}
\cp_n &\equiv  p_n,p_{n+1},\ldots, p_N.
\end{align}
}
This notation is usefully extended by defining 
\sub{
\begin{align}
d\cx_n &\equiv dr_n dr_{n+1}\ldots dr_N, &&\quad\text{and}&
d\cp_n &\equiv  dp_n dp_{n+1}\ldots dp_N.
\end{align}
}

Both the energy, $\H$ and the equilibrium  density $\rho_0$ are measured quantities
and as such, both  averaged over the probability of the phase space locations of the $N$ particles $p_0 = p_0(\cx,\cp)$.
The average with respect to $p_0$ is defined
\begin{align}
  \rho_0 =  \scalar{\rho_0(\cx_1, \cp_1) }_{p_0} \equiv \Tr p_0(\cx_1, \cp_1) \rho_0(\cx_1, \cp_1) 
\end{align}
where $\Tr$ denotes the trace operator
\begin{align}
  \Tr \equiv \sum_{N=0}^\infty \frac{1}{h^{3N}N!} \iint d\cx_1 d\cp_1.
\end{align}
The average energy is defined similarly.
The the equilibrium 
joint 
 probability density  is  the Boltzmann distribution 
\begin{align}
  p_0(\cx_1, \cp_1) = \Xi^{-1} \exp\lr{-\beta \lr{H_N - \mu N}}.
\end{align}
where 
\begin{align}
  \Xi = \Tr \exp\lr{-\beta \lr{H_N - \mu N}}. \label{eqn:nuc:GPF}
\end{align}
is called the {\em grand partition function}.
The grand potential then follows according to 
\begin{align}
  \Omega = -\beta^{-1}\ln \Xi.
\end{align}

We may eliminate the momentum terms from the grand partition function, equation \eqnref{nuc:GPF} immediately
\begin{align}
  \Xi = \frac{n_Q^N}{N!} Z_U Z_V Z_N
\end{align}
where $n_Q = \lr{\frac{m}{2\pi\hbar^2 \beta}}^{3/2}$ is the {\em quantum concentration}.
and 
\begin{align}
  Z_U Z_V Z_N = 
\end{align}





Since the density is a functional of $p_0$, which is in turn a function of $V_\ext$,
it follows that the density is functional of the external potential.
The converse is also true:
the external potential is uniquely determined by the density,
a result known as the Hohenberg-Kohn theorem.
The probability density is then determined by the external potential,
from which it follows that the probability density is a unique functional of the density.
We outline a proof of the Hohenberg-Kohn theorem in \appref{Hohenberg_Kohn}.
It is deferred to the appendix because the proof is not constructive.

The approximate  grand potential  can therefore be expressed as a unique functional of the density,
\begin{align}
  \Omega_V\lrs{\rho_0} &=  F - \mu N \\
  &= \lr{\KE + \UE - TS} + \int d\rho\lr{ V_\ext -\mu }\\
  &= \F + \int d\rho\lr{ V_\ext -\mu }\\
\end{align}
where $\F$ is the intrinsic Helmholtz free energy.



The grand potential, through $U$, is a function of $p_0(\cx)$, the  probability describing the locations of all $N$ particles.
This associated multi-particle interactions are complicated and difficult to model.
We decouple these interactions by introducing the approximate probability distribution  $ p = p(\vr_i, \vr_j)$ 
- dependent now only on two particle interactions - 
to evaluate our thermodynamic variables.
This in turn reduces $U$ to two particle interactions, 
\begin{align}
  \Phi = \sum_{i\ne j} \sum_i^N \phi(\vr_i, \vr_j).
\end{align}
Furthermore, we assume that the external potential $\VE$ influences each particle equally.
Therefore, our approximate Hamiltonin is
\begin{align}
\H = \KE + \Phi + NV_\ext.
\end{align}


The {\em relative entropy } or {\em Kullback-Leibler divergance} gives the amount of information lost 
when using the approximate distribution  $p$ rather than the correct distribution $p_0$,
and is defined
\begin{align}
  \KLD{p}{p_0} = \Tr p \log \frac{p}{p_0} \label{eqn:nuc:KLD}
\end{align}
$\KLD{p}{p_0} \ge 0$, which follows from Gibbs inequality with equality if and only if $p=p_0$.

It is convenient for $p$ to be evaluated via a variational principle and so we defined $p_0$ according to 
\begin{align}
 \Omega_V\lrs{\rho} = \beta^{-1}\KLD{p}{p_0}+   \Omega, 
\end{align}
so that the approximate grand potential approaches the true value on application of a variational principle.
It then follows that 
\begin{align}
  \Omega_V\lrs{\rho} &= \beta^{-1} \Tr p \log \frac{p}{p_0}  - \beta^{-1}\ln \Xi \\
  &= \beta^{-1} \Tr p \log \frac{p}{e^{-\beta\lr{\H - \mu N}}}\\
%  &= T S +    \Tr p \lr{H_N - \mu N} \\
  &= - T S_p + \H_p - \mu N_p \\
  &= F_p - \mu N_p\\
  &= \F + \int V_\ext d\rho - \int \mu d \rho.
\end{align}
where 
\begin{align}
 \F\lrs \rho_0 = \Tr p \log\frac{p}{e^{-\beta\H_\in}} = \KE + \UE - TS_p
\end{align}
and
where the subscript indicates that the thermodynamic quantities are evaluated with the approximate $p$ rather than $p_0$.




Have 
\begin{align}
F = \int dr \rho_0 V_\ext + \F\lrs{\rho_0}
\end{align}
and 
\begin{align}
  V_\ext + \mu_\in\lrs{\rho_0} = \mu
\end{align}
where
\begin{align}
  \mu_\in \equiv \deltarho \F.
\end{align}


Integration of interaction potential.



We may eliminate the momentum terms from the grand partition function, equation \eqnref{nuc:GPF} immediately
\begin{align}
  \Xi = \frac{n_Q^N}{N!} Z_{U+V} 
\end{align}
where $n_Q = \lr{\frac{m}{2\pi\hbar^2 \beta}}^{3/2}$ is the {\em quantum concentration}.
We therefore concentrate on $Z_{U+V}$,
the contribution from the potentials.





The joint probability distribution can be marginalised to obtain the single particle density,
\begin{align}
\rho(r_1) = \int p_0(\cx_1) d\cx_2,
\end{align}
where 
\begin{align}
  d\cx_n &\equiv dr_ndr_{n+1}\ldots dr_N \quad \text{so that}
\end{align}
\begin{align}
  \int \rho(r) dr = N.
\end{align}
Likewise the 2-particle density
\begin{align}
 \rho^{(2)}(r_1, r_2) = \frac{N(N-1)}{Z_\phi}\int e^{-\beta \sum_{j>i} \phi(r_{ij})}dr_3 \ldots dr_N
\end{align},
and introduce radial distribution function
\begin{align}
  g(r_{12}) = \frac{V^2}{N^2} P_2(r_1, r_2)
\end{align}


\begin{align}
\frac{  \delta}{\delta \phi(r_i, r_j) } = - \beta^{-1} \frac{\delta \ln \Xi}{\delta \phi(r_i, r_j)} = \frac{N(N-1)}{Z_\phi} = \half \rho^{(2)}(\vr_1,\vr_2)
\end{align}
Integrated by choosing
\begin{align}
  \phi_\alpha \equiv \phi(r_1,r_2;\alpha) = \phi_r + \alpha \lr{\phi - \phi_r}
\end{align}
to 
\begin{align}
  \F\lrs{\rho_0} = \F_r \lrs \rho_0 + \half \int_0^1d\alpha \int dr_1dr_2 \rho^{(2)}\lr{\phi-\phi_r}
\end{align}

If $\phi_r = 0$ then $\F_r = \F_\ideal$ and
\begin{align}
\F\lrs{\rho_0} = \F_\ideal + \half \int_0^1d\alpha \int dr_1dr_2 \rho^{(2)}\phi_\alpha \phi.
\end{align}

Split potential into reference and perturbation
\begin{align}
  \phi(r) = \phir + \phip
\end{align}

Then 
\begin{align}
c^{(2)} - c_r^{(2)} = \given{\frac{\beta \delta^2}{2\delta \rho(r_1) \delta \rho(r_2)} \int d\alpha int dr_1 dr_2 \rho^{(2)} \phip}{\rho(r) = \rho}.
\end{align}

Simplest random phase approximation 
\begin{align}
  \phi^{(2)}  = \phi(r_1) \phi (r_2)
\end{align}
so that
\begin{align}
c^{(2)} - c_r^{(2)}(r) = -\beta \phip
\end{align}

Next expand to lowest order in $\phip$
so that need to evaluate
\begin{align}
  \int dr_1 dr_2 \rhor^{(2)} \phi_p
\end{align}

This can be done by expanding 
\begin{align}
  \rhor^{(2)} = \rhor^{(2)} + \half \lr{\rho(r_1)  + \rho(r_2) - 2\rho } \frac{\d \rhor^{(2)}}{\d\rho} + \ldots + \half \lr{\rho(r_1)  + \rho(r_2) - 2\rho } \frac{\d^2 \rhor^{(2)}}{\d^2\rho}
\end{align}
with $\rhor^{(2)}=\rho^2 g_r(r)$
Then
\begin{align}
  c^{(2)} - c_r^{(2)}(r) = -\frac{\beta}{2} \phip \frac{\d^2 \rhor^{(2)}}{\d\rho^2}
\end{align}
which is the mean density approxmition.


average energy 
\begin{align}
  U &= - \frac{\d}{\d\beta} \ln Z = \frac{3}{2}N\kB T -  \frac{\d}{\d\beta}\ln Z_\phi
\end{align}
where
\begin{align}
   \frac{\d}{\d\beta}\ln Z_\phi &= \frac{1}{Z_\phi} \int \sum_{j>i} \phi(r_{ij})e^{-\beta \sum_{j>i} \phi(r_{ij})} d\cx\\
   &= \frac{N(N-1)}{2 Z_\phi} \int \phi(r_{12}) e^{-\beta \sum_{j>i} \phi(r_{ij})} d\cx\\
   &= \frac{1}{2} \int \phi(r_{12}) P_2 d\cx\\
   &= \frac{N^2}{2V^2} \int \phi(r)g(r)4\pi r^2 dr.
\end{align}


%consider
\begin{align}
  \Omega\lrs p = \Tr_{cl} f\lr{H_N - \mu N + \beta^{-1} \ln f}.
\end{align}
%Have
%\begin{align}
%  \Omega\lrsquare{p_0} = -\beta^{-1}\ln \Xi \equiv \Omega,
%\end{align}
%the grand potential.
%Then 
%\begin{align}
%  \Omega[p] = \Omega[p_0] + \beta^{-1} \lr{Tr_\cl p\ln \lr{p/p_0}   }
%\end{align}


Correlation functions,
\begin{align}
  \F\lrs{\rho} = \F_\ideal\lrs{\rho} - \Phi\lrs{\rho}
\end{align}
so that
\begin{align}
  \beta \mu_\in \lrs{\rho;r} = ln \lr{\lambda^3 \rho} - c
\end{align}
where
\begin{align}
  c \lrs{\rho;r} \equiv \beta \deltarho Phi.
\end{align}
Equilibrium density is then
\begin{align}
  \rho = z \exp \lr{-\beta V_\ext + c}
\end{align}

\subsection{Approach}
Capillary approximation removed by varying density.

Eg Random phase approximation to give pair distribtuion function\cite{Ruckensteirn2005}.
\begin{align}
\rho^{(2)} (r, r^\prime, \phi_\alpha^{(2)}) \approx \phi(r) \phi(r^\prime)
\end{align}
Next need free energy of reference.
Use hard sphere\cite{Ruckensteirn2005}
\begin{align}
F_1\lrs{\rho(\vr)} = \int d\vr \fh \lr{\rho(\vr)}.
\end{align}
which is a local density approximation good for weakly inhomoenous system - no solid interface\cite{Ruckensteirn2005}.
Then 
\begin{align}
  \Omega\lrs{\rho\lr{\vr}} = F\lrs{\rho} - \mu \int dr \rho
\end{align}
minimise to get\cite{Ruckensteirn2005}
\begin{align}
  \mu = \mu_h + \int dr^\prime \phi^\two \rho(r^\prime)
\end{align}
In homogenous limit get \cite{Ruckensteirn2005}
\begin{align}
  f(\rho) = \fh(\rho) - \half \alpha \rho^2
\end{align}
where \cite{Ruckensteirn2005}
\begin{align}
  \alpha = -\int dr^\prime \phi^\two(r^\prime)
\end{align}

Pressure and chemical potential from free energy\cite{Ruckensteirn2005}
\begin{align}
  p = p_h - \half \alpha \rho^2
\end{align}
\begin{align}
  \mu = \mu_h - \alpha \rho
\end{align}

Hard sphere pressure from Carnaham Starling from\cite{Ruckensteirn2005}
\begin{align}
  p_h = \frac{kT\lr{1 + \theta + \theta^2 - \theta^3}}{\lr{1 - \theta}^2}
\end{align}
and chemical potential \cite{Ruckensteirn2005}
\begin{align}
  \frac{\d p_h}{\d \rho} = \rho \frac{\d \mu_h}{\d \rho}
\end{align}

For pertabation method need  WCA scheme\cite{Ruckensteirn2005}
to give repulsive.
then
\begin{align}
  \mu = \mu_h + \int dr^\prime \phi^\two2_\WCA \rho(r^\prime)
\end{align}

\subsection{Applications to liquid surfaces}


For plainar surface
\begin{align}
\gamma = \given{\frac{\d F}{\d A}}{T, V, N}
\end{align}
or 
\begin{align}
\gamma = \given{\frac{\d \Omega}{\d A}}{T, V, \mu}
\end{align}
where
\begin{align}
  \Omega = -pV + \gamma A
\end{align}
Also
\begin{align}
  \gamma = \int_{-L/2}^{L/2} dz \lr{\sigma_N(z) - \sigma_T(z)}
\end{align}


Also have

\begin{align}
  \rho^{(m)} = \Xi^{-1} \sum_{N\ge }^\infty \frac{z^N}{N-m} \int d r_{m+1} dr_N exp\lr{\beta(V + U)}
\end{align}
so $\rho^{(1)} \equiv \rho_0$
and 
\begin{align}
  \scalar{\hat \rho(r) \hat \rho(r^\prime)} =\scalar{\sum_{i\ne j} \delta(r-r_i) \delta(r^\prime - r_j)} + \scalar{\sum_{i} \delta(r-r_i) \delta(r^\prime - r_j)}
= \rho^{(2)} + \rho_0\delta{\rho- \rho^\prime}
\end{align}
Now for pairwise potential
\begin{align}
 \Xi = \sum_0 ^\infty \frac{\lambda^{-3N}}{N!} \int dr_1 \ldots d r_N \exp\lr{\beta \int \dr u \hat \rho - \beta/2 \int dr dr^\prime \hat I (r,r^\prime)\phi(r,r^\prime)}
\end{align}
where $\hat I = \sum_{i\ne j} \delta(r - r_i)\delta(r^\prime - r_j)$.
Then
\begin{align}
  \frac{\delta \Omega}{\delta \phi(r,r^\prime)} = -\beta^{-1} \frac{\delta \ln \Xi}{\delta \phi(r, r^\prime)} = \half \scalar{\hat I} = \half \rho^{(2)}(r,r^\prime).
\end{align}


In \cnt\ density at centre assumed equal to bulk liquid density, 
and shape of and free energy of surface taken to be a planar interface.
Critical nucleus then found by setting 
\begin{align}
\frac{d \Omega}{d R} =0
\end{align}
at $r = R^\ast$.

But density $\rho(r)$ should not be constrained other than to require that it approach the bulk vapor density at large distance.
Then 
\begin{align}
\frac{\delta \Omega_V}{\delta \rho(r)} = 0
\end{align}
at $\rho(r) = \rho^\ast(r)$.
The mulitdimensainal free energy has a minium at the uniform vapor pressure,
and a second lower minimum at the  uniform liquid density. Between these saddle point sound by setting the funciional deriate to zero.
The matirx of second deriatives containes a negative eigenvalue correspionding to deirection of motion over the barrier.

For equilibrium gas-liquid interface similar except zero eigenvalue not negative. 

Have
\begin{align}
\Omega_V = F - \mu N = F - \mu \int dr \rho(r).
\end{align}
Then \begin{align}
\frac{\delta F}{\delta \rho(r)} = \mu
\end{align}
at $\rho(r) = \rho^\ast(r)$.

Cahn and Hilliard in 1959 first,
used
\begin{align}
F[\rho(r)] = \int dr \lrsquare{f_u (\rho(r)) + K(\del \rho(r))^2}
\end{align}
where $f_u$ is Helhotz free energy per unit volume of a uniform system withe density $\\rho$ everywhere.

Functional derivateive finds
\begin{align}
\frac{\d f_u}{\d \rho} - 2K \del^2 \rho - \frac{\d K}{\d \rho}\lr{\del^2 \rho}^2 = \mu.
\end{align}
Near the spinodal critical nucleus is large but small in amplitude.
Near spinodal examind by Unger and Klein. 

Oxtoby and evans investigatesd when $J$ is near 1 cm cubed per second. 
Used hard sphere sluid and Yukawa attractive tail,
\begin{align}
\phi_{att}(r) = \frac{\alpha \lambda^3 \exp(-lambda r)}{4\pi \lambda r}.
\end{align}
Free energy functional then
\begin{align}
F\lrsquare{\rho(r)} = \int d r f_h(\rho(r)) + \frac{1}{2}\int \int dr dr^\prime \rho(r) \rho(r^\prime) \phi_{att}\lr{\abs{r-r^\prime}}
\end{align}
where $f_h$ is free energy of uniform hard sphere fluid, treated locally.

Attrractive pottentioanl not treated not in square gradient approach of Cahn and Hilliard.
Then functional derivative gives
\begin{align}
\mu_h\lrsquare{\rho(r)} = \mu - \int d r^\prime \rho(r^\prime) \phi_{att} \lr{\abs{r-r^\prime}}
\end{align}
Integral equation solved by iteration having some inital guess of radius.
Equilibrium is unstable however (saddle point).
If $R_0$ is too large tends to blow up, if too small tends to shrink into nothing,
if ``correct'' forms nearly converges then shinks or blows up at one side or other of saddle point

In Oxtoby and Evans 1988 cavitation studied, not just condensation as here.
Used Becker-Doring prexponetial to find rates. 

\subsection{General Notes}




To simplify these, we
consider  only pair-wise interactions,
\begin{align}
%  \UE(\cx) \approx \sum_{j>i} \sum_i^N 
\Phi(\cx) = \half \iint d \vr_i d\vr_j \phi(\vr_i, \vr_j)\rho(\vr_i, \vr_j) 
 =  \half \iint d \vr_i d\vr_j \phi(\vr_i, \vr_j)\rho(\vr_i)\rho(\vr_j) 
\label{eqn:nuc:two_interactions}
\end{align}
where $\phi(\vr_i, \vr_j)$ is the two particle potential between a particle at $\vr_i$ and $\vr_j$,
and $\rho(\vr_i, \vr_j) $ is the two particle density function.
In the second equality in \eqnref{}
it has been specified  that
$\rho(\vr_i, \vr_j)= \rho(\vr_i)\rho(\vr_j)$.
This is the simplest possible two particle density function.
It assumes that the densities are uncorrelated - an assumption known as the {\em random phase approximation}.
The approximation has been found to be accurate in the absence of rapid \todo{get this sentence right} \cite.
If the oil droplet is sufficiently large then this approximation should hold.
The random phase appromxation may well need to be refined for very small droplets, however, 
where the oil-water interface cannot be ignored.


The integral in \eqnref{nuc:two_interactions} is still too difficult to solve directly.
However, the interactions of most fluids are dominated  by volume exclusion effects (van der Waal type interactions).
Longer range interactions are, in general, only of secondary importance.
The interaction term $\phi$ can therefore be split 
by considering {\em hard sphere} volume exclusion, $\phi_\hs$ and  an attractive perturbation $\phi_\attr$.

To do so we first eliminate everything but the interaction by differentiating with respect to $\phi(\vr_i, \vr_j)$,
\begin{align}
  \frac{\delta \Omega_V}{\delta \rho(\vr_i, \vr_j)} = \frac{\delta F}{\delta \rho(\vr_i, \vr_j)} = \frac{1}{2}\rho(\vr_i, \vr_j). \label{eqn:deltaOmega_pert}
\end{align}
Equation \eqnref{deltaOmega_pert} is then integrated from the  hard-sphere reference, $F_\hs$
to full potential along the path parameterised by $\alpha$, 
\begin{align}
  F\lrs{\rho} = F_\hs\lrs{\rho} +\half \int d\alpha  \iint d \vr_i d\vr_j \phi_a( \vr_i, \vr_j)\rho(\phi_\alpha;\vr_i, \vr_j), \label{eqn:nuc:two_interactions}
\end{align}
where 
\begin{align}
  \phi_\alpha(\vr_i, \vr_j) = \phi_\hs(\vr_i, \vr_j) + \alpha \phi_a (\vr_i, \vr_j)\quad\text{and}\quad 0\le\alpha\le 1.
\end{align}
The potential is  gradually `turned on' along the integration path 
and the full potential, $\phi$, is recovered on the boundary located at $\alpha=1$.


By assuming that the contribution to the hard sphere energy is entirely local
the total free energy of the reference can be written
\begin{align}
  F_\hs\lrs{\rho} \approx \int d\vr f_\hs(\rho(\vr))
\end{align}
where $f_\hs(\rho(\vr))$ is the potential of a uniform  hard-sphere fluid\cite{OxtobyBook}.
This is the local-density approximation.









For binary mixture 
\begin{align}
  F = \int d\vr \fh \lrs{\rho(\vr_1), \rho(\vr_2)} + \half \int
\end{align}

%%% Local Variables: 
%%% mode: latex
%%% TeX-master: "tshorrock_thesis"
%%% End: 
