

\chapter{Tensor derivation to Maxwell's equations}\label{ch:acousticSRTensor}


\section{Introduction}\label{sec:AcousticSRTensor:introduction}

In \chapref{acousticSR} Geometric Algebra was used for the derivations.
Since Geometric Algebra is not of widespread use in the physics community, this appendix repeats the derivation using Tensor Algebra.
The same equation numbers will be used in this appendix so the derivations can be easily compared.
Doing so makes a good advert for the economy of expression afforded by Geometric Algebra.

\section{The acoustics analogue to Maxwell's relation}

The energy-momentum tensor of an ideal fluid is\cite{LandauBook, Taub1978}
\eqa{
%  \GATA{T(a) = (\epsilon + p) a \cdot u u - a p,}
{T^{i j} = (\epsilon + p) u^i u^j - g^{i j} p}
%  T(a) = (\epsilon + p) a \cdot u u - a p, % w a \cdot u u - a p =
\tag{\ref{eqn:EMtensor}}
}
where, $\epsilon \equiv \epsilon(p)$ is the barotropic total energy density,
$p$ is the pressure,
$g^{i j}$ is a diagonal metric tensor with $g^{00}=1$ and $g^{i i} = -1$ for $i=1,2,3$,
and 
$u$ is the velocity vector of the spacetime path, with the parametrisation chosen such that $u^2 = u^i u_i = 1$. % and $P$ is the pressure.
That is, the units of length and time are chosen so that velocity of sound is set to unity.
%That is, c.g.s. units have been chosen.

The speed of sound, $c$,  given at constant entropy density, $\sigma$, is\cite{LandauBook,Taub1978} 
\begin{align}
  c^2 = \given{\frac{\d p}{\d \epsillon}}{\sigma}. 
\tag{\ref{eqn:soundspeed}}
\end{align}
This is the same as the non-relativistic expression except that the energy density has replaced the mass density.
The speed of sound equals the speed of light (unity) if 
% \begin{align}
%   \epsilon = p^\prime - p_0 + \epsilon_0
% \end{align} 
%where $p^\prime$ is a pressure that fluctuates with position, so that $dp = dp^\prime$,
%and $p_0$  and $\epsilon_0$  are the ambient  pressure and mean energy density, respectively.
%%Rather can carry the constants $ p_0 $ and $\epsilon_0$ through the rest of the derivation we write
%The thermodynamic pressure is therefore
%\begin{align}
%  p \equiv p^\prime - p_0 + \epsilon_0. \label{eqn:pshort}
%\end{align}
% the equation of state first introduced by Taub\cite{Taub1978},
\eqa{
  \epsilon(p) = p.
\tag{\ref{eqn:eos}}
}
This equation of state was first introduced by Taub\cite{Taub1978}.
%At infinity $p^\prime = p_0$ 
%and so $\epsilon_\infty = p_\infty = \epsilon_0 \ne p_0$.

%therefore enforces that the sound speed equals that of light (unity).
%The notation of equation \eqn{eos} is quite compact.
%For \eqnref{eos} to be consistent with the integral of the sound speed (equation \eqnref{soundspeed}),
%the pressure $p$ must be interpreted as follows,
%\begin{align}
%  p = p^\prime - p_0 + \epsilon_0. \label{eqn:pshort}
%\end{align}
%Here $p^\prime$ is the pressure perturbation measured with a hydrophone,
%$p_0$  and $\epsilon_0$  are the mean  pressure and energy density respectively.



Applying \eqnref{eos} to \eqnref{EMtensor} simplifies the energy momentum tensor,
\eqa{
 % \GATA
%{
%T(a) &=  p\lr{2 a \cdot u u  - a} \\
%&\equiv  \frac{\scalefactor^2}{4}  A a A, }
{
  T^{i j}  = p\lr{2 u^i u^j - g^{i j}} 
           \equiv \frac{\scalefactor^2}{2 } \lr{ A^i A^j - A^k A_k g^{i j}/2} 
}
%  
\tag{\ref{eqn:EMFluid}}
}
where the vector potential, $A$,  satisfies
\eqa{
%\GATA
%{
%A = 2\tfrac{1}{\scalefactor}p^{1/2}u =2\tfrac{1}{\scalefactor} \epsilon^{1/2} u.
%}
{
A^i = 2\tfrac{1}{\scalefactor}p^{1/2}u^i =2 \tfrac{1}{\scalefactor} \epsilon^{1/2} u^i.
}
\tag{\ref{eqn:defnA}}
}
The constant scale-factor, $\scalefactor$, is determined from the ambient proper number density of the fluid, $n_0$, and the ambient pressure, $p_0$, as follows,
\begin{align}
\scalefactor = \frac{n_0}{\sqrt{p_0}}. 
\tag{\ref{eqn:scalefactor}}
\end{align}
%As will be seen, its role in the formulation is analogous to the free permitivity of space in electromagnetism.

The motivation for introducing the 4-vector $A$ is that it represents a potential flow.
To demonstrate this, we first note that the relativistic generalisation to the velocity potential, $\psi$,  is defined\cite{LandauBook} by
\begin{align}
  \d_i \psi \equiv - \frac{\epsilon+p}{n} u_i = - \frac{2 p}{n} u_i,
% \del \psi = - \frac{\epsilon+p}{n} v = - \frac{2\epsilon}{n} v,
\tag{\ref{eqn:relPot}}
\end{align}
where $\d_j \equiv \frac{\d}{\d x^j}$ and  $n$ is the proper particle number density of the fluid.
Equation \eqnref{eos} has been used to obtain the second equality.
To show that this is equal to the negative of the potential $A$, we use a thermodynamic argument given by Taub\cite{Taub1978}.
The internal energy density, $\epsilon$, is equal to the sum of the rest mass and the internal energy per particle\cite{LandauBook, Taub1978}, $e$, 
\begin{align}
  \epsilon(p) = nm( 1 + e(p)),
\tag{\ref{eqn:relEpsilon}}
\end{align}
where $m$ is the particle mass at rest.
From the isentropic thermodynamic relation $m de = - p d\lr{\frac{1}{n}}$
it follows that 
\begin{align}
 n d\epsilon = \epsilon dn - n^2 p d \lr{\frac{1}{n}} = \lr{\epsilon + p} dn.
\tag{\ref{eqn:relNEpsilon}}
\end{align}
Applying equation \eqnref{eos} and integrating we obtain
\begin{align}
n =\scalefactor \sqrt{   p },
\tag{\ref{eqn:relN}}
\end{align}
where $\scalefactor$ is the constant introduced in \eqnref{scalefactor}.
With the aid of equation \eqnref{eos} it follows that
\begin{align}
A_i = 2\tfrac{1}{\scalefactor}\sqrt p  u_i =\frac{\epsilon + p}{n} u_i =  -\d_i \psi,
\tag{\ref{eqn:AisPotFlow}}
\end{align}
as asserted.
%Therefore, and as asserted, the presence of a potential flow in the acoustic medium corresponds to a gauge transformation of Maxwell's equations,
%and hence, the description of the acoustic wave is unaffected by potential flows of the medium.




In the absence of external fields, the equations of motion are obtained by setting the 
 divergence of the energy momentum tensor (equation \eqnref{EMFluid}) to zero.
By projecting the divergence of \eqnref{EMFluid} along the timelike component we find
\eqa{
  u_i \d_j T^{i j} = \half\scalefactor^2 u_i A^i \d_j A^j = 0.
%  u\cdot\scope T(\scope\del)= \half  u\cdot A \del \cdot A = 0.
\tag{\ref{eqn:projEMFLuidTime}}
}
%The check denotes the scope of the derivative.
Since, from \eqnref{defnA}, the vector $A$ is parallel to $u$  it follows that 
\eqa{
  \d_j A^j = 0
  %\del \cdot A  =0
\tag{\ref{eqn:eomTime}}
}
and so the vector potential $A$ is conserved.
The spacelike projection,
$\d_j T^{k j} - u^k u_i \d_j T^{i j}$, gives in turn,
% $\scope T(\scope \del) - u u\cdot \scope T(\scope\del)$, gives in turn,
\eqa{
%  u^j \d_j A^k - u^l \d_j A_l g^{k j} = 
u_j \lr{\d^j A^k - \d^k A^j} = 0.
%  u \cdot \lr{\del \wedge A} = 0.
\tag{\ref{eqn:eomSpace}}
}
The relativistic vorticity tensor, $F^{j k}$, is the exterior derivative  of the vector potential, 
\eqa{
F^{j k} \equiv \d^j A^k - \d^k A^j
%F = \deal \wedge A,
\tag{\ref{eqn:DefnVorticity}}
}
and so \eqnref{eomSpace} implies that the vorticity tensor is orthogonal to the velocity.

By taking the divergence of \eqnref{DefnVorticity} and using \eqnref{eomTime} it follows that 
%the vector identity
%$\del \del \cdot A = \del^2 A - \del\cdot\lr{ \del \wedge A} $
% we find,
\eqa{
  \d_i \d^i A^j = \d_i F^{i j}.
%  \del^2 A = \del \cdot F = \del F.
\tag{\ref{eqn:wave}}
}
%The second equality follows because $\del F = \del \cdot F + \del \wedge F$ 
%and because the  operator identity $\del\wedge \del = 0$ causes $\del \wedge F$ to vanish.
The left-hand-side of equation \eqnref{wave} is a wave equation and so we interpret the right-hand-side as an acoustic source,
a 4-current, $J$.
Therefore 
\sub{
\eqa{
  \d_i F^{i j} \equiv J^j.
\tag{\ref{eqn:Maxwell:a}}
%  \del  F = J.
}
Furthermore, from \eqnref{DefnVorticity} we have
\begin{align}
\tag{\ref{eqn:Maxwell:b}}
  \epsilon_{i j k l} \d^j F^{k l} = \epsilon_{i j k l} \d^j \lr{\d^k A^l - \d^l A^k} = 0,
\end{align}
}
which follows due to the use of the repeated differential with the Levi-Civita permutation tensor, $\epsilon_{i j k l}$.
The two equations of \eqnref{Maxwell} constitute Maxwell's relation and equation \eqnref{eomTime} has specified the Lorenz gauge.
%We do not explore this observation further here.



%%% Local Variables: 
%%% mode: latex
%%% TeX-master: "../tshorrock_thesis"
%%% End: 
