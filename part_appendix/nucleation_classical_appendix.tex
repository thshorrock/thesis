\chapter{Classical nucleation theory}\label{app:CNT}

\section{Introduction}
In \secref{nuc:CNT} the derivation of \cnt\ was very briefly sketched.
To help this thesis be self contained we fill in the details here.

\subsection{The Poynting Correction}
The bubble is in thermodynamic equilibrium when it is at its critical radius, 
\begin{align}
  \astar = \frac{2\gamma}{p_v-p_L}. \tag{\ref{eqn:LaplaceRelation}}
\end{align}
Then the chemical potentials are equal
\begin{align}
  \mu_v(p_v^\ast) = \mu_L(p_L)\tag{\ref{eqn:thermoEqlbm}}.
\end{align}
However, due to the curvature of the bubble, the critical pressure within the bubble, $p_v^\ast$,
is not the same as the equilibrium vapour pressure.
At a flat interface the vapour pressure is  $p_\infty$ (where the $\infty$ denotes the radius of the bubble for a flat interface).
The flat interface is therefore at equilibrium when
\begin{align}
  \mu_v(p_{\infty}) =  \mu_L(p_{\infty}). \label{eqn:thermoEqlbmInfity}
\end{align}
Equations \eqnref{thermoEqlbm} and  \eqnref{thermoEqlbmInfity} are related by the Gibbs-Duhem relation,
\begin{align}
  d \mu = -s dT + v dp,
\end{align}
 where $s$ and $v$ are the entropy and volume per molecule and $T$ is the temperature.
Choosing an adiabatic path for the chemical potential of the vapour gives
\begin{align}
\mu_v(p_v) - \mu_v(p_{\infty}) =  \int_{p_{\infty}}^{p_v^\ast} v dp. \label{eqn:GDvapour} % = \mu_L(p_L) - \mu_L(p_{\infty}) =  \int_{p_{\infty}}^{p_L} v dp.
\end{align}
Similarly for the liquid we have
\begin{align}
\mu_L(p_L) - \mu_v(p_{\infty}) =  \int_{p_{\infty}}^{p_L} v dp.\label{eqn:GDLiquid}  % = \mu_L(p_L) - \mu_L(p_{\infty}) =  \int_{p_{\infty}}^{p_L} v dp.
\end{align}
Assuming the gas is ideal, so that $ p v = \kB T$, \eqnref{GDvapour} gives
\begin{align}
  \mu_v(p_v^\ast) - \mu_v(p_{\infty}) = \kB T \ln \lr{\frac{p_v^\ast}{p_\infty}}.
\end{align}
Using equations \eqnref{thermoEqlbm} and \eqnref{thermoEqlbmInfity} we may
equate equations \eqnref{GDvapour} and \eqnref{GDLiquid} to obtain
%Equating this to  $\mu_L(p_L) - \mu_L(p_{\infty})$ (using \eqnref{thermoEqlbm} and \eqnref{thermoEqlbmInfity}) gives
\begin{align}
  p_v^\ast = p_\infty \exp \lr{\int_{p_\infty}^{p_L} dp\frac{V}{RT}  }\label{eqn:PoyntingCorrectionFormal}
\end{align}
where $V$ is the molar volume.
Equation \eqnref{PoyntingCorrectionFormal} is known as the Poynting correction.
If the fluid is incompressible with number density $N_L$  (per mole) then we may write
\begin{align}
  p_v^\ast = p_\infty \exp \lr{ \frac{\lr{p_L -p_\infty} }{N_L RT}  },\tag{\ref{eqn:PoyntingCorrection}}
\end{align}
which is the result quoted in \secref{nuc:CNT}

An alternative derivation is to start  from Kelvin's equation,
\begin{align}
\ln\lr{\frac{p_v}{p_\infty}} = - \frac{2\gamma }{a N_L RT}, \label{eqn:KelvinRelation}
\end{align}
which more directly relates pressure and curvature. (see Skinner's review\cite{Skinner1972} for derivation and general discussion).
Equation \eqnref{PoyntingCorrection} follows immediately by substituting  \eqnref{LaplaceRelation} into \eqnref{KelvinRelation}.

% \subsection{The rate of nucleation}



% This taken from \cite{Oxtoby1992}.
% Grand potential of single phase
% \begin{align}
%   \Omega = - pV
% \end{align}
% and for two phase with planar interface
% \begin{align}
%   \Omega = -pV + \sigma A.
% \end{align}
% where $\sigma$ is surface tension and $A$ is area.
% The first term is negative and second positive.

% Spherical droplet of vapour in liquid background
% \begin{align}
%   \Omega  = -p_l V_l  - p_vV_v + 4\pi R^2 \sigma.
% \end{align}
% where
% \begin{align}
%   V_l &= V-\frac{4\pi}{3}R^3 \\
%   V_v &= \frac{4\pi}{3}R^2
% \end{align}
% so that (check sign here)
% \begin{align}
%   \Delta \Omega = -  \frac{4\pi}{3} R^3 \Delta p + 4\pi R^2 \sigma.
% \end{align}
% where $\Delta p = p_l - p_v$.

% If liquid assumed to be incompressible
% \begin{align}
% \Delta p - \rho_l\lr{\mu - \mu_{sat}}.
% \end{align}
% If vapour ideal gas
% \begin{align}
%   \mu - mu_{sat} = \kB \ln\lr{\frac{p_v}{p_{sat}}} = \kB T \ln S.
% \end{align}

% Replace $R$ with number of particles
% \begin{align}
% \Delta \Omega=-n\kB T \ln S + \lr{36 \pi}^{1/3} \rho_l^{-2/3} \sigma n^{2/3}.
% \end{align}
% Maximum occurs when 
% \begin{align}
% n^\ast = \frac{32 \pi \sigma^3}{2\rho_l^2 \lr{\kB T}^3 \lr{\ln S}^3}
% \end{align}
% and height of barrier is
% \begin{align}
%   \Delta \Omega^\ast = \frac{16\pi \sigma^3}{3\rho_l^2 \lr{\kB T}^2\lr{\ln S}^2}
% \end{align}
% As $S$ increases from 1 at coexistence, the hight of barrier falls.

% To get rate of nucleation need to find the number of critical nuclei per unit volume.
% Make this equal to 
% \begin{align}
%   J = J_0 \exp\lr{-\Delta \Omega^\ast / \kB T}
% \end{align}
% where 
% \begin{align}
% J_0 = a \sqrt{2/\pi}\frac{\rho_v^2}{\rho_l} \sqrt{\frac{\gamma}{m}}
% \end{align}
% where m is molecular mass and a is sticking probability, usually taken to be unity.






% In this section we give the kinetic argument for \cnt.
% We use the notation of McClurg\cite{},
% although the argument is much older, with Katz\cite{} doing much of the work to put it into it's modern form.
% Like McClurg we consider only the simplest case, 
% the  growth and shrinking of a bubble by the absorption and release of individual molecules.
% For the more general case of groups of molecules joining and releasing see \cite{}.

% In the derivation the oil is split into a solute and the solution.
% The solute are the molecules of the oil that have sufficient energy to join the bubble.
% The solution is the rest of the liquid oil.
% While this split is somewhat artificial,
% the result does not depend upon the details.
% It is used only as a conceptual crutch.



% The rate of growth of a bubble with $i$ molecules is
% \begin{align}
%   J_i = \beta  A_i C_i - E_{i+1} C_{i+1}. \label{eqn:Jn}
% \end{align}
% $\beta$ is the impingement rate of the bubble vapour on a unit area.
% From the kinetic theory of ideal gasses it is equal to 
% \begin{align}
% \beta = \frac{p_v}{\sqrt{2\pi M_r \kB T}}, \label{eqn:nuc:beta}
% \end{align}
% where  $M_r$ is the molecular weight.
% The ratio 
%  \begin{align}
% \frac{\beta}{\beta^\eqlbm} = \frac{p_v}{p_\infty} \equiv  S
% \end{align}
% is the  supersaturation ratio.

% $C_i$ is the number concentration of bubble containing $i$ molecules and $E_{i+1}$ is the rate at which a molecule leaves a bubble with $i+1$ molecules.
% %Equation \eqnref{Jn} is a recurrence relation for a Markov process.

% Next  thermodynamic equilibrium is assumed (denoted with a $\eqlbm$ superscript).
% At equilibrium the probability of adding a molecule is equal to the probability of loosing one, 
% and so 
% \begin{align}
%   J_i^\eqlbm = 0 =  \beta^\eqlbm A_i C_i^\eqlbm - E_{i+1}^\eqlbm C_{i+1}^\eqlbm.
% \end{align}
% If the number density of the molecules that are capable of joining the bubble  (the solute) is small compared to the rest of the oil (the solvent)
% then the rate at which the bubble looses  molecules will be independent of the concentration of the solute, 
% so that $E_{i+1} = E_{i+1}^\eqlbm$ and 
% \begin{align}
%   E_{i+1} =  \beta^\eqlbm A_i C_i^\eqlbm/C_{i+1}^\eqlbm. \label{eqn:Enpone}
% \end{align}
% Substituting  equation \eqnref{Enpone} and multiplying by $\lr{\beta^\eqlbm / \beta}^n$ gives 
% % \begin{align}
% %   J_i = \beta A_i C_i^\eqlbm \lrs{\frac{C_i}{C_i^\eqlbm} - \frac{\beta^\eqlbm C_{i+1}}{\beta C_{i+1}^\eqlbm}}. \label{eqn:JnTwo}
% % \end{align}
% % Multiplying \eqnref{JnTwo}   by $\lr{\beta^\eqlbm / \beta}^n$ gives
% \begin{align}
%   \frac{J_i} {\alpha_i \beta C_i^\eqlbm\lr{\beta/\beta^\eqlbm}^n} 
% = \frac{C_i}{C_i^\eqlbm}\lr{\frac{\beta^\eqlbm}{\beta}}^n - \frac{C_{i+1}}{C_{i+1}^\eqlbm}\lr{\frac{\beta^\eqlbm}{\beta}}^{i+1} \label{eqn:JnThree}
% \end{align}
% The final step is to assume that the rate of nucleation is steady.
% Then $J_1 = J_2 = ... =  J_b = J$.
% Most of the terms in the sum cancel.
% If we let the sum tend to infinity then
% % then upon summing the first $b$ terms equation \eqnref{JnThree} becomes
% % \begin{align}
% %   J\sum_{i=1}^b\frac{1} {\alpha_i \beta C_i^\eqlbm\lr{\beta/\beta^\eqlbm}^n} = \frac{C_1}{C_1^\eqlbm}\lr{\frac{\beta^\eqlbm}{\beta}} - \frac{C_{b+1}}{C_{b+1}^\eqlbm}\lr{\frac{\beta^\eqlbm}{\beta}}^{b+1}.
% % \end{align}
% % If $b$ is let to go to infinity then 
% \begin{align}
%   J = \frac{\beta}{\sum_{i=1}^\infty\lrs{ A_i C_i^\eqlbm S^n}^{-1}} \label{eqn:rateOne}
% \end{align}

% From the thermodynamic argument in \secref{nuc:CNT} we found that the bubbles with $n$ molecules is
% \begin{align}
%   C_i = n \exp \lr{- \frac{\Delta G(n)}{\kB T}}. \tag{\ref{eqn:CnBoltzman}}
% \end{align}

% It is convenient to turn the sum in \eqnref{rateOne} into an integral over $i$, the number of molecules in a bubble.
% Formally the  integral run from 1 molecule to $\infty$,
% however, since the integral is dominated over the narrow range  near to the critical radius we 
% can let the integral run from $-\infty$ to $\infty$, and expand the change in Gibbs energy to quadratic terms,
% \begin{align}
%   \Delta G = \Delta G^\ast - \frac{\lr{i-i^\ast}^2}{2}\frac{\d^2 \Delta G(i)}{\d i ^2}. \label{eqn:DeltaGExp}
% \end{align}
% The minus sign in \eqnref{DeltaGExp} follows because the energy change is a maximum at the critical radius.
% In addition, we assume that in the integral  $A = A^\ast \equiv 4\pi \astar^2$ and $S^i = S^{i^\ast}$. 
% %\equiv S^{\rho_v\Delta G^\ast / M_r}$  take their critical values.
% %To obtain this latter relation we have noted that a bubble of volume 
% %$V_b = 4\pi a^3/3$ has $i= V_b /N_v$ molecules, and $N_v$ is the number density of the vapour.
% Equation \eqnref{rateOne} therefore becomes
% \begin{align}
%   J &= \frac{ A^\ast S^{i^\ast} \beta^\ast n\exp{ \lr{- \frac{\Delta G^\ast}{\kB T}}   }}{\int_{-\infty}^\infty dn \exp{\lr{- \frac{\lr{n-n^\ast}^2}{2}\frac{1}{\kB T}\frac{\d^2 \Delta G(n)}{\d n ^2  }}}}
% \\ &= { A^\ast \beta^\ast   Z  n\exp{ \lr{ i^\ast\ln S -\frac{\Delta G^\ast}{\kB T}}   }}.
%  \label{eqn:rateTwo}
% \end{align}
% where
% \begin{align}
% Z = \frac{1}{\rho_v^\ast A^\ast}\sqrt{-\frac{1}{2\pi \kB T}\lrsquare{\frac{\d}{\dr}\lr{\frac{1}{r^2}\frac{\d }{\d r}}}\Delta G(r)} 
% = \frac{1}{\rho_v^\ast A^\ast}\sqrt{\frac{4\sigma}{\kB T}} %\frac{M_r}{4\pi \astar^2  \rho_v^\ast}\sqrt{\frac{4\sigma}{\kB T}}
% \label{eqn:Zeldovich}
% \end{align}
% is called the Zeldovich factor.

% Pulling together equations \eqnref{nuc:beta}, \eqnref{Zeldovich} and \eqnref{rateTwo}  gives
% \begin{align}
%   J \approx S^{{\rho_v\Delta G^\ast / M_r}}   
%   \frac{p_v n}{  \rho_v^\ast\kB T}\sqrt{\frac{2\sigma M_r}{\pi }}
%   \exp{ \lr{- \frac{\Delta G^\ast}{\kB T}}   }.
% \end{align}
% The pre-exponential factor is therefore
% \begin{align} 
% J_0=  S^{i^*}   
%   \frac{p_v n}{  \rho_v^\ast\kB T}\sqrt{\frac{2\sigma M_r}{\pi }}.
%   \tag{\ref{eqn:nuc:Jzero}}
% \end{align}

%%% Local Variables: 
%%% mode: latex
%%% TeX-master: "../tshorrock_thesis"
%%% End: 
