

\chapter{What is measured by ultrasound?\\And how should it be modelled?}\label{ch:observables}



\section{Introduction}
%In the ultrasound literature it is assumed that known physics of the interrogated medium
%carry over when imaging the medium with ultrasound.
In the ultrasound literature it is generally asserted, without comment,
that a model that describes the propagation and scattering of sound in a medium will describe the results of an ultrasound experiment.
That is, there is an implicit assumption that the acoustic measurement process captures the physics
that has long been verified as correct by other means. %that is known to be correct.
This chapter argues against this assumption.
%This short chapter argues the contrary:
%ultrasound does in general correctly measure the physics of the medium and its scatterers.
Ultrasound must assume certain properties of the world in order to be able to measure anything at all.
When these assumptions do not hold, what is measured diverges from what actually happens.
To model the measurements made with ultrasound a distinct acoustic model must therefore be constructed:
an operational model that deals with the quantities that are measured,
rather than the actual, correct, values of those quantities. 
%which will be different to the known, correct model of the medium.
In this chapter the constraints on an acoustic model are presented,
and in \chapref{medium} the process of constructing the model is completed.

It is here only noted that the distinction between what is measured and what is true is unusual in physics.
Often the two concepts are conflated,
and there are certain strands of positivism that would insist that what is measured is by definition true.
%A broader discussion in relation to the results of the next two chapters is deferred to \chapref{digression}.

%the results of an ultrasound experiment must be modelled with an acoustic model
% that is necessarily contry to w

\section{What is measured by ultrasound}\label{sec:measurement}


In medical ultrasound two physical quantities are measured:
\nlist{
  \item The time it takes for a pulse emitted from a transducer to return after being reflected - the {\em pulse-echo} time.
  \item The pressure profile of the pulse recorded by the transducer.
}
From these two measurements all the physical quantities of interest must be derived.

The most used derived quantity is distance.
Distances are measured using the time it takes a pulse of sound to propagate from a transducer
to a reflecting object and then to return again. 
If the sound is emitted from the transducer at a time, $\tm$,
and the sound returns at a time,  $\tp$,
then the task is to find from these two numbers the spatio-temporal location, $x$,
of the point of reflection.

What happens to the sound in between leaving the transducer and returning
cannot be known by acoustic measurement.
In this ignorance ultrasound practitioners assume that the time at which the echo 
occurred is the midpoint of $\tm$ and $\tp$,
\sub{
\label{eqn:radar}
\begin{align}
 \tau(x) &= \frac{\tp + \tm}{2}.\label{eqn:radarTime}
\intertext{Other choices could certainly be made\cite{Debs1996}, 
  but would imply a knowledge of the world beyond that learnt from $\tm$ and $\tp$ alone.
  To measure distances from the times $\tm$ and $\tp$ a sound speed, $c$, is required.
  Assuming, again in ignorance, that the sound returns at the same speed at which it left gives
}
 \rho(x) &= \frac{\tp - \tm}{2}c. \label{eqn:radarDistance}
\end{align}
}
These are the definitions of time and space that are used in ultrasound.
%They are also identical to definitions used by \Poincare\cite{Poincare1908, Pierseaux2005} and Einstein\cite{Einstein1905,Dolby2001}
%with the exception that the speed, $c$, is here the speed of sound rather than the speed of light.
%The assumption that the sound returns at the same speed as it left may be relaxed.
%Then $\tau(x) =\epsilon \tp + (1-\epsilon)\tm$ and 
%$\rho(x) = \epsilon \tp c - (1-\epsilon)\tm c$ for any
%$0<\epsilon<1$.
%This does not change any of the arguments that follow\cite{Debs1996}.



Equation \eqnref{radarDistance} requires an {\em a priori} knowledge of the sound speed
for otherwise distances cannot be determined from temporal measurements.
In diagnostic ultrasound scanners this speed is usually taken to be \unit{1540}\metre\reciprocal\second.
%The constancy of the speed of sound is identical to Einstein's  second postulate for special relativity\cite{Einstein1905},
%except that the sound speed takes the role of the speed of light.
The speed of sound is here a constant not because of some physical law, 
or because it is a quantity independently measured,
but because when using sound to make measurements 
there is no other choice  but to assume the sound speeds constancy. %that the speed of sound is constant.
%As discussed in the introduction, this conforms more to \Poincare's view of the light postulate than to Einstein's.

The required constancy of the speed of sound has a rather uncomfortable consiquence,
particularly for the experiments conducted in a fluid medium that are considered in this thesis.
%, ultrasound experiments conducted within a fluid medium.
For it is well known that, in general, the speed of sound in a fluid is not constant.
A density fluctuation induces a temperature fluctuation and this in turn induces a local change to the sound speed.
Further, the continuity equation of mass contains a self advection term that also impacts the speed at which a signal propagates.
The world as measured by ultrasound must therefore be different from the world as measured by other means.
Acoustic measurements {\em are not correct}, 
and we cannot claim that the physical predictions derived from, say, the Euler equation,
will be observed with ultrasound.
Any attempt to do so leads to a logical contradiction -
a prediction of a varying speed of sound that cannot, {\em by definition}, be measured.
We must therefore distinguish the {\em correct} physical models of the acoustic medium,
from an operational model that predicts what is {\em measured} by ultrasound. 
%The two models have a mapping between them,
%but this mapping is not trivial.
%When modelling the outcome of an ultrasound experiment,
%therefore,
%we must necessarily depart from the true model of the medium.
%-
%the model that has been formulated and long verified to be accurate by other means.
%Instead,
%to model the outcome of an ultrasound experiment -
To predict what is measured,
the operational model must somehow shoehorn in the incorrect assumption that the sound speed is constant.

%We need to therefore distinguish the acoustic measurements displayed on the screens of our ultrasound machines
%from what actually occurs in the subject of the experiment.
%We therefore have two physcial models,

To form a complete picture of an ultrasound experiment we are therefore faced with two slightly unfamiliar tasks
\nlist{
\item Create an acoustic model that describes the ultrasound experiment.
  By construction, this model must predict a constant sound speed.
  %To abreviate ``the model of the world when measured acoustically'',
  %we shall simply refer to this model as the {\em acoustic model}.
  %This model is differentiated from the familiar {\em correct model} of what is actually happening.
\item Map the acoustic model with the correct model.
}
The focus of this thesis will be the first of these tasks.
The goal, as a first step, is to describe what is actually measured,
rather than what actually happened.
From time to time we will compare the two models -
but we will in general refrain from attempting to compute the map between the two.

We wish to construct an acoustic model that is self contained.
When an interaction occurs between two entities that are measured acoustically,
we would like to explain the interaction from the acoustic measurements alone.
It should not be necessary to map the interaction to the correct model to predict the outcome, 
and then map that prediction back to the acoustic model for verification.
In short, we need to understand the physics of the acoustic model - the symmetries and conservation laws
that determine how interactions play out.

We make two assumptions further assumptions for the measurements of ultrasound.
\nlist{
\item The measurements are invarient to translations in coordinate space.
  \item The measurements are invarient to translations in time. 
}
Without these properties ultrasound would be such an unreliable modality
that it would not be worth the effort that has been invested in it.
As is well known by Noether's theorem,
these assumed symmetries imply that our acoustic measurements must conserve momentum and energy.
By applying these conservation laws to interactions we can start to model the outcome of interactions,
and we can be confident that the underlying physics of the acoustic model 
cannot be so radically different from what is known to be correct.



In this thesis the acoustic medium will be a fluid,
and so we concentrate our efforts on building an acoustic model in this domain.

%\subsection{Physics in the acoustic model}
%\todo[inline]{Picture mapping between what is modelled with ultrasound and what is modelled for other imaging modalities}

%Ultrasound measures only the pulse-echo time and the pressure profile of the returning wave.
%All other quantities must be derived.
%We have already seen that in order to define a measure of distance,
%ultrasound erroneously defines the speed of sound to be everywhere constant.
%We have also seen the consiquence, that when modelling the outcome of an ultrasound experiment
%we must necessarily depart from the true model of the medium.
%%-
%%the model that has been formulated and long verified to be accurate by other means.
%%Instead,
%%to model the outcome of an ultrasound experiment -
%%to test what is actually measured,
%%one must shoehorn into the model the incorrect assumption that the sound speed is constant.

%But to understand the interactions that occur on our screen,
%do we constantly need to refer to the true model and map back to what is measured before and after the interaction?
%Or is it possible to model the interactions directly in the acoustic model,
%and understand the evolution of the entities on our screen in an entirely self-contained way?
%We assume 




%Since we depart from the correct model,
%do we also have to start again and experimentally determine how the entities on out screen interact.




%There are therefore two models,
%the model of the world as measured by ultrasound,
%and the model of the world as measured by other means.
%We will refer to the former as the  {\em acoustic model} and the latter as the  {\em correct model}.
%%The situation 
%%As 
%%have a world view as described in \figref{modelMap},
%To form a complete picture of an ultrasound experiment we are therefore faced with two slightly unfamiliar tasks
%\nlist{
%\item Create an acoustic model that describes the ultrasound experiment.
%  By construction, this model must predict a constant sound speed.
%  %To abreviate ``the model of the world when measured acoustically'',
%  %we shall simply refer to this model as the {\em acoustic model}.
%  %This model is differentiated from the familiar {\em correct model} of what is actually happening.
%\item Map the acoustic model with the correct model.
%}
%The focus of this thesis will be the first of these tasks.
%The goal, as a first step, is to describe what is actually measured,
%rather than what actually happened.
%From time to time we will compare the two models -
%but we will in general refrain from attempting to compute the map between the two.

%Nonetheless,
%it is illustrative to see a simple example of how the mapping manifests itself.


%In this thesis the acoustic medium will be a fluid,
%and so we concentrate our efforts on building an acoustic model in this domain.





\section{How the difference between the acoustic and correct models manifests itself}

The acoustic notion of distance not only assumes
that sound propagates at a constant speed,
but also assumes that the propagation is indefinite.
There is not a maximum distance,
calculable from the acoustic model,
at which the sound ceases to propagate and the coordinate system `runs out'.
The energy in sound must propagate to infinity.
The notion of distance therefore assumes that the sound propagation abides by a wave equation.
The constant sound speed further demands that the wave equation be linear.
%
%The constancy of the speed of sound rests upon the assumption that signals propagate with a linear wave equation.
%Since the pulse-echo definition implicitly assumes that the sound wave is linear
It follows that the sound profile of the emitting signal is expected %- from the pulse-echo definition -
to return in exactly the same form as it was emitted.
%All changes in the returned sound profile are therefore ascribed to the reflecting object,
%rather than any non-linear transformation resulting from the medium
%As will be shown the sound speed further demands that the wave equation be linear.

%Ultrasound not only assumes that the acou

%One of the consequences of the pulse-echo definition of distance is that 
%all changes in the returned sound profile are ascribed to a reflecting object or acoustic source.
%It is assumed that the signals propagate indefinately.
%If the acoustic medium propagating sound from the transducer to the object does not return an echo,
%then from ultrasound's perspective the medium is entirely transparent.  
%It cannot silently transform the signal at all.
%In short, the notion of distance implically assumes that the signals
%propagate with a linear wave equation.

%A true fluid medium predicts a non-constant speed of sound due to its self advection,
%and local perturbations in density and temperature that in turn affect the speed of sound locally.
%Both these problems become vanishingly small if the disturbance is small,
%with the disturbance propagating at a fixed speed according to the linear wave equation.
%A fluid that is only perturbed slightly is therefore conformant with the requirements of acoustic measurement.
%One expects, therefore,
%that the acoustic model and the correct model converge in this limit.

%We therefore have a starting point for our acoustic model: the linear fluid.
%If we were assert that ultrasound - due to the incorrect assumption of a constant sound speed - measures linear fluids
%then we have a model for acoustic measurement that is self-consistent.
%This is a step forward,
%because in contrast with the correct model of the medium,
%it enables distances within the fluid medium to be defined in a consistent manner with what is acoustically measured.
%Let us investigate what else can be predicted by this model.

%If we assume that ultrasound only measures linear fluids then we have a consistent model.
%Let us investigate the predictive power of this model.


%\subsection{The linear fluid}






%\subsection{Mapping between the acoustic and the correct models}


%One of the consequences of the pulse-echo definition of distance is that 
%all changes in the returned sound profile are ascribed to a reflecting object or acoustic source.
%The medium itself is 
%It is assumed that the signals propagate indefinately.
%If the acoustic medium propagating sound from the transducer to the object does not return an echo,
%then from ultrasound's perspective the medium is entirely transparent.  
%It cannot silently transform the signal at all.
%In short, the notion of distance implically assumes that the signals
%propagate with a linear wave equation.
%.


%One of the consequences of the pulse-echo definition of distance is that 
%all changes in the returned sound profile are ascribed to a reflecting object.
%If the acoustic medium propagating sound from the transducer to the object does not return an echo,
%then from ultrasound's perspective the medium is entirely transparent.  
%The required constancy of the sound speed implies that the medium be considered linear.
%From an ultrasound perspective it cannot silently transform the signal at all.

 \begin{figure}
     \subfloat[]{
           \label{fig:nonLinearOriginal}
           \includegraphics{ch_medium_nonlinear_pulse.0}}
         \hfill
     \subfloat[]{
           \label{fig:nonLinearSawTooth}
           \includegraphics{ch_medium_nonlinear_echo.1}}
      \caption{A simple plane wave becomes more sawtooth-like due to self advection and a non-constant sound speed.  
Figure \subref{fig:nonLinearOriginal} gives the sinusoid profile of the signal emitted from the transducer.
A disturbance in particle speed is induced with distance.
Figure \subref{fig:nonLinearSawTooth} gives an a profile of the signal that returns some time later.
 }
      \label{fig:nonLinear}
 \end{figure}

And so here we are confronted with description that deviates from the facts.
In reality the non-constant sound speed and self advection of the medium distort the echo that returns from an object.
%
%It is well known that in general an acoustic medium does alter a sound pulse.  
%In the compression phase of a sound pulse the medium is denser and hotter, 
%and so locally the sound travels faster.
%Additionally, for a fluid medium,
%the medium undergoes a self-advection when perturbed.
%Both of these effects distort the sound pulse, 
A simple plane wave that was emitted as a sinusoid pulse (\figref{nonLinearOriginal}) is sheared and returns more like a sawtooth (\figref{nonLinearSawTooth}) after travelling some way in a fluid medium.
%If we consider a simple plane wave undergoing a reflection and returning. %, such as in  \figref{nonLinearPlaneReflecter},
%A  simple plain wave would return transformed from the sinusoid in \figref{nonLinearOriginal} to the saw tooth in \figref{nonLinearSawTooth}.

So how would such a pressure profile be understood from an ultrasound perspective?
%due to the non-linear propagation of the sound pulse.
Since ultrasound is forced to interpret the propagation as linear,
%the sound profile of the emitting signal is expected %- from the pulse-echo definition -
%to return in exactly the same form as it was emitted.
all changes in the returned sound profile must therefore be ascribed to the reflecting object. %,
%rather than any non-linear transformation resulting from the medium.
%The differences 
%then it follows that the returned pulse must result from a moving reflecting surface.
Ultrasound  therefore  {\em measures} a fluctuating reflecting surface.
%moving in such a way so as to cause the observed shearing in the sound pulse.
%As seen in \figref{nonLinearPlaneReflecter}, 
The reflector must be interpreted as moving in such a way so as to cause the observed shearing in the sound pulse.

%There is an important lesson in these observations:
%what ultrasound {\em measures} will not in general be the same as what {\em actually occurs}.
%As illustrated in \figref{mapping}, there are two different models that can be used.
And so here we see the relationship between the two models.
The first is a model of what actually happens,
the non-linear propagation of the sound wave in the medium reflecting of a stationary reflector.
The second is a model of what is measured,
the linear propagation of the sound wave reflecting off an oscillating entity.
If we want to understand the results of our ultrasound experiments then it is the second of these models that is the more important.
%
%The second would be a model
%While a mapping between what is measured and what occurs may be possible in simple cases such as \figref{nonLinearPlaneReflecter},
%the mapping should not be mistaken as the primary objective.
%The most post pertinent question is to create a physical model that interprets what is {\em measured}.
If ultrasound physicists cannot model what is displayed on their apparatus then ultrasound is lost as an imaging modality.
One can then seek a mapping from what is measured to what occurred if necessary.
%
%Just explaining what actually happens is not enough,
%and the creation of a mapping may not always be possible.

Unfortunately, there is a further problem highlighted by this example.
After a certain distance,
the varying sound speed and self advection terms have had time to exert themselves to such an extent that the profile in
\figref{nonLinearSawTooth} becomes a true sawtooth, with a discontinuity in the density profile of the fluid.
Thereafter, the density profile becomes multi-valued and ceases to be physical.
As is well known,
this is the moment at which a shock wave forms in the fluid.
For the acoustic observer, the discontinuity would be understood as a the reflector moving away from the transducer at the speed of sound.
%
%With some thought, however, it becomes clear that the acoustic model has a problem even before the formation of the shock wave.
%If we admit the possibility of a reflector being measured as moving towards the transducer in an instance,
%then we should also admit the converse, of a reflector measured as moving with great speed away from the transducer.
%But this is not possible. 
%But this, again, is contry to the pulse-echo measurement technique.
This is a problem, because an entity that moves away from the transducer at the speed of sound will never return an echo,
and is therefore unmeasurable.
We cannot admit in our model the measurement of a surface moving at faster than the speed of sound
without logical contradiction.
%If follows that our analysis has already broken down long before the creation of a shock wave. 

This belies a more general problem.
To calculate the interactions between entities it is often convenient to change coordinate frame
and consider the problem as measured by an observer at rest with the interaction.
If we are to understand the physics of our operational model then acoustic measurement is no different in this requirement.
%
% interactions that
But how do we understand the physics as measured by ultrasound between different frames of reference?
We clearly cannot freely transform to an entity moving at faster than speed of sound
to understand how the measured physics would appear in this coordinate system.
The simple fact is that there is no measured physics in this coordinate system.
%We need to understand how velocities are measured in a given coordinate system.
We need a method of transforming between reference frames to model the perspective of two
observers that move with respect to one another.



%\todo[inline]{Comment that at certain distance ultrasound measures the reflector moving instantaneously.
%However, this is a problem, because a reflector moving at faster than the speed of sound cannot be measured}





%The required constancy of the sound speed implies that the medium be considered linear.
%From an ultrasound perspective it cannot silently transform the signal at all.


%The constancy of the speed of sound rests upon the assumption that signals propagate with a linear wave equation.
%It follows that the sound profile of the emitting signal is expected - from the pulse-echo definition -
%to return in exactly the same form as it was emitted.
%All changes in the returned sound profile are therefore ascribed to the reflecting object,
%rather than any non-linear transformation resulting from the medium.



%The constancy of the speed of sound does however imply a physical model.  
%The indefinite propagation of sound at a fixed speed implies that the sound propagates according to a linear wave equation
%There are two important consequences to this, which are detailed in the following sections,
%\nlist{
%  \item The quantities {\em measured} by ultrasound are generally not the quantities {\em actual} value.
%    We therefore need to differentiate what is observed from that which is true.
%  \item Distances and times as measured transform according to the Lorentz transform.
%    The limiting velocity is the speed of sound.
%}

%\section{Modelling in Ultrasound}
%\todo[inline]{Picture mapping between what is modelled with ultrasound and what is modelled for other imaging modalities}
%\subsection{What ultrasound measures differs from what is true}


\section{Acoustically measured transformations}




From the measurement rules of equations~\ref{eqn:radarTime} and \ref{eqn:radarDistance}
it follows that an entity that moves away from the transducer at a speed that is faster than the speed of sound cannot be measured.  
This is not because such motions are impossible but because the sound will never catch  up with the entity and so there will never be an echo to record.
Transformations between different frames therefore have the speed of sound as a limiting velocity.

%A transformation between a frame that moves slower than the speed of sound to one that moves faster than the speed of sound (with respect to the bulk) does not make sense 

To understand the transformation rule that must be used when changing frames in ultrasound,
we  recall that ultrasound assumes that distances propagate with a linear wave.
The speed of this wave is by definition the same for all reference frames,
and it is this common description of how sound propagates that determines how distances are measured between frames. 
%The transformation between frames must therefore respect the symmetries of the propagating signal.


The linear wave is described by the D'Alembert operator, which in one dimension is given as follows,
\begin{align}
  \frac{\partial^2 }{\partial x^2} - \frac{1}{c^2}\frac{\partial^2 }{\partial t^2} \equiv
  \left(\frac{\partial }{\partial x}  + \frac{1}{c}\frac{\partial }{\partial t}\right)
  \left(\frac{\partial }{\partial x}  - \frac{1}{c}\frac{\partial }{\partial t}\right),
\end{align}
where $x$ is the spatial coordinate, $t$ is time, and $c$ is the speed of sound.

If we have a second reference frame $(x^\prime, t^\prime)$ defined in terms of the first as follows
\sub{
\begin{align}
  x^\prime \equiv x^\prime(x,t),\\
  t^\prime \equiv t^\prime(x,t),
\end{align}
}
then if both are to describe the same linear wave then the D'Alembert operator must transform like so,
\begin{align}
  \left(\frac{\partial }{\partial x}  + \frac{1}{c}\frac{\partial }{\partial t}\right)
  \left(\frac{\partial }{\partial x}  - \frac{1}{c}\frac{\partial }{\partial t}\right) =
  \left(\frac{\partial }{\partial x^\prime}  + \frac{1}{c}\frac{\partial }{\partial t^\prime}\right)
  \left(\frac{\partial }{\partial x^\prime}  - \frac{1}{c}\frac{\partial }{\partial t^\prime}\right)
\end{align}

We proceed by using the argument of Heras\cite{Heras2016} and assume a linear scalar valued transformation, $A$,
so that
\sub{
\label{eqn:waveTransform}
\begin{align}
  \left(\frac{\partial }{\partial x}  - \frac{1}{c}\frac{\partial }{\partial t}\right) &=
  A \left(\frac{\partial }{\partial x^\prime}  - \frac{1}{c}\frac{\partial }{\partial t^\prime}\right) \\
  \left(\frac{\partial }{\partial x}  + \frac{1}{c}\frac{\partial }{\partial t}\right) &=
  A^{-1}\left(\frac{\partial }{\partial x^\prime} + \frac{1}{c}\frac{\partial }{\partial t^\prime}\right).
\end{align}
}

At small velocities 
the details of the pulse-echo measurement process should become irrelevant,
and the mapping between frames should reduce to the Galilean transform.
Assuming this to be so gives,
%We further assume that the Galilean transform holds for small velocities.
%That is
\sub{
  \label{eqn:gallileanLimitWave}
\begin{align}
  x^\prime  &= x - vt &&\textnormal{as} && \frac{\partial x}{\partial t} \rightarrow 0,
  \intertext{or in operator form}
  \frac{\partial }{\partial t} &= \frac{\partial }{\partial t^\prime} - v  \frac{\partial }{\partial x^\prime} & & \textnormal{when } & & \frac{\partial }{\partial t} =  0.
\end{align}
}
Inserting equations \eqnref{gallileanLimitWave} into \eqnref{waveTransform} yields,
\sub{
  \label{eqn:waveTransformationStage}
\begin{align}
  \frac{\partial }{\partial x}   &=
  A \left(\frac{\partial }{\partial x^\prime}  - \frac{v}{c}\frac{\partial }{\partial x^\prime}\right) \\
  \frac{\partial }{\partial x}  &=
  A^{-1}\left(\frac{\partial }{\partial x^\prime}  + \frac{v}{c}\frac{\partial }{\partial x^\prime}\right)
\end{align}
}
and by equating these equations it follows that
\sub{
  \label{eqn:waveTransformationFnc}
\begin{align}
  A = \gamma\left(1 + v/c\right) \\
  A^{-1} = \gamma\left(1 - v/c\right) 
\end{align}
}
where $\gamma = \frac{1}{\sqrt{1-\frac{v^2}{c^2}}}$.
Inserting equation \eqnref{waveTransformationFnc} into \eqnref{waveTransform} and adding and finally subtracting the result gives
\sub{
\begin{align}
\frac{\partial }{\partial x} &= \gamma\left(\frac{\partial }{\partial x^\prime} -  \frac{v}{c^2}\frac{\partial }{\partial t^\prime}\right)\\
\frac{\partial }{\partial t} &= \gamma\left(\frac{\partial }{\partial t^\prime} - v\frac{\partial }{\partial x^\prime} \right)
\end{align}
}
which are the well known Lorentz transforms for the derivatives.

We find therefore, that the linear wave equation transforms according to the Lorentz equation, where the speed of sound plays the limiting velocity.
The sound must be measured the same in all reference frames, and is used to define distances.
It follows that when space and time are acoustically measured,
the models built on top of these measurements should be also be Lorentz invariant.

The ultrasound literature does not comply with these remarks.
Currently, when modelling an ultrasound experiment, a fluid medium is always described by a Galilean invariant theory such as Euler's equation or the Naiver-Stokes equation.
The resulting model is then capable of predicting motions that are faster than the speed of sound
that simply cannot be measured with ultrasound.
There is a conflation between the acoustic and correct models of the fluid.
%Ultrasound simply cannot make these measurements.
To form a model of what is measured by ultrasound (rather than what is actually occurring)
it is necessary to use a Lorentz invariant description of the world.

\subsection{Some  consequences of a Lorentz invarient measurement process}

With the transformation between two acoustic frames of reference in hand
we can now turn to how physical interactions would be measured  by different acoustic observers.
We assume that mass and momentum are conserved,
in accord to the fundamental symmetries we demand of any measurement system,
and consider the famous argument of Lewis and Tollman\cite{Lewis1909, Pauli1958, Inverno1992}
that examines how mass is measured in an elastic collision of two identical balls.
Since we have already found that the measured spatio-temporal coordinates vary with the relative speed between reference frames,
we admit the possibility that the measured mass also varies with the speed of the ball.


%We consider the elastic collision between two equal particles.
First we consider the frame where the ball of mass $m(u)$ travelling at speed $u$ collides elastically
with a stationary particle of mass $m(0)$, resulting in a particle of combined mass $M(U)$ travelling at speed $U$.
The conservation of mass and momentum yield,
\eqa{
  m(u) + m(0) &= M(U) \\
  m(u)u  &= M(U)U
}
from which it follows that
\eqa{
  \label{eqn:relMass:stepOne}
  m(u) = m(0) \lr{\frac{U}{u-U}}
}

In the frame where the resultant particle is at rest,
the two particles approach each other at speed $U$.
Since the two frames relate by the Lorentz transform the two velocities relate by the well known velocity composition law,
\eqa{
  u = \frac{2U}{1 + U^2/c^2}
}
where $c$ is the speed of sound.
Solving for $U$,
\eqa{
  U = \frac{c^2}{u}\lrsquare{1 - \sqrt{1-\frac{u^2}{c^2}}},
}
and inserting into equation \eqnref{relMass:stepOne} gives
\eqa{
  \label{eqn:relMass}
  m(u) = \frac{1}{\sqrt{1-u^2/c^2}} m(0).
}
In ultrasound the measured mass is therefore increased above the measured rest mass
by the Lorentz factor.
It is to be emphasised
that the gain in mass   stems from the measurement process and the assumption that momentum is conserved in the measurements.
It is a consequence of the measurement process rather than a phenomenon that has somehow been missed in other imaging modalities.

Continuing,
we note from Newton's second law and equation~\eqnref{relMass} that the acoustically measured rate of work
when moving a particle is
\eqa{
  \frac{dE}{dt} &= \frac{d(m\vu)}{dt} \cdot \vu = \frac{m(0)}{\lr{1-u^2/c^2}^\frac{3}{2}} u \frac{du}{dt}\\
    &= c^2\frac{dm}{dt}
}
where the chain rule has been used in the second line.
If follows that the acoustically measured energy is
\eqa{
  E = mc^2 + \text{const}.
}
By convention, we set the arbitrary constant to zero.
The acoustically measured energy is therefore related to the acoustically measured mass by the speed of sound.




%\todo[inline]{Comment on the consequences to density - the density depends on frame of reference.
%How does this relate to rest mass}


%This requires a change when modelling a fluid.

%As noted by \cite{LandauBook}, the particle density $n$ is with reference to the proper volume.
%It is therefore dependent on the frame of reference.
%Likewise the energy density is  

%The velocity  potentials in a Galilean coordinate system is defined in terms of the speed of the fluid particles, $\vu$.
%\begin{align}
% \vu \equiv \vdel \psi 
%\end{align}
%This can be expressed in terms of the enthalpy, $w$, by noting that if the entropy, $s$ is constant,
%\begin{align}
%  d w = T ds - \frac{1}{\rho}dp = \frac{1}{\rho}dp
%\end{align}
%From Euler's equation we have
%\begin{align}
%  \frac{\partial \vu}{\partial t} + \vu \cdot \vdel \vu &= -  \frac{1}{\rho} \vdel p
%  \intertext{and so, by using the identity ... } 
%\frac{\partial \vu}{\partial t} + \frac{1}{2}\vdel \vu^2 + \vu \times \vdel \times \vu &= -  \vdel w
%  \intertext{and so, by inserting the potential we get } 
%\end{align}


% becomes
%\begin{align}
%  \del \psi = - \frac{wu}{nc}
%\end{align}
%where $w$ is the heat function per proper volume and $n$ is the particle number per proper volume.



%\section{blar}

%\subsection{An alternative view}
%The initial assumption of \secref{}, that the physical description of the medium does not depend on the imaging modality, has been shown to be wrong.
%If we try to derive the quantities of interest, such as the pressure, the temperature, or the density away from the transducer,
%and we use physics derived without ultrasound in mind to do so, 
%then not only do we get the wrong answer, 
%but we get a logical contradiction.

%Getting the wrong answer is not in of itself a problem.  
%If we want to 
%so long as the answer is the derived quantity as mea

\subsection{Discussion}


What is actually measured in an ultrasound experiment is limited
to the pulse-echo time and the pressure profiled of the returning signal.
From these measurements all other quantities of interest have to be derived.
To do so,
certain assumptions have to be made.
Most notable is that the sound speed is everywhere constant,
but if we hope for reproducible experiments then we must also assume conservation of energy and momentum.

What is measured by ultrasound will differ from what actually occurs when these assumptions do not hold.
In such cases, 
we can either abandon the whole exercise and simply state that ultrasound is unreliable in these conditions
or we can be more constructive and try to model these derived quantities. 
This thesis takes the latter approach.
The next step would be to find a general way of mapping the acoustic model back to what actually occurred but this is not pursued here.

To model the quantities that are derived in ultrasound, 
it is necessary to adhere to the assumptions of the measurement system.
It is found that the pulse-echo technique, the constancy of sound speed, and the assumption of translational invariance
imply that the models of acoustic measurement must be Lorentz invariant.
This invariance, coupled with the assumed conservation of energy and momentum,
imply a velocity dependent inertial mass will be measured,
and an equivalence between the acoustically measured mass and energy.
 
There are obvious parallels between what has been outlined in this chapter
and special relativity.  
However, it is to be emphasised that the limiting speed in ultrasound is the speed of sound, not the speed of light.
Indeed, no reference to the speed of light has been made in our argument.
Parallels, such that they exist, result from the broadly similar definitions of space used in ultrasound and relativistic physics -
the pulse-echo time in the first case, radar time\cite{Dolby2001} in the latter - and the demand of both measurement systems for the same basic symmetries of time and space.












%It is at this point that the required constancy of the speed of sound poses some interpretation problems to the acoustic observer,
%for it is well known that the sound speed in a fluid is not in general constant.
%The sound speed is a function of the medium's thermodynamic quantities such as  density, pressure, temperature and entropy,
%and these vary throughout the fluid, not least due to the propagation of the sound wave itself.
%The equations of state that would most naturally be used to characterise a medium often imply a varying sound speed inconsistent with the measurement process.
%Conversely, quantities are measurable by ultrasound only if they belong to an equation of state that is consistent with a constant speed of sound.
%The uncomfortable conclusion is that the quantities measured with ultrasound do not in general have the same value as when they are measured with other modalities.
%In ultrasound we speak of the  {\em observable} value rather than the {\em physical} value.



%\subsection{blar}





%Equations of state that are confirmed by other experimental techniques do not correctly predict the outcomes of ultrasound experiments
%due to ultrasound's incorrect reliance on a constant sound speed.


%It is at this point that the required constancy of the speed of sound poses some interpretation problems to the acoustic observer.




%for it is well known that the sound speed in a fluid is not in general constant.
%The sound speed is a function of the medium's thermodynamic quantities such as  density, pressure, temperature and entropy,
%and these vary throughout the fluid, not least due to the propagation of the sound wave itself.
%Equations of state that are confirmed by other experimental techniques do not correctly predict the outcomes of ultrasound experiments
%due to ultrasound's incorrect reliance on a constant sound speed.





%This is a difficult square to circle.

%One option would be to use ultrasound for the spatio-temporal location of entities, and then ascribe the real, independently measured physical properties and physical laws to those entities.



%One option would be to use ultrasound for the spatio-temporal location of entities, and then ascribe the real, independently measured physical properties and physical laws to those entities.
%An example would be modelling an acoustic medium as an ideal gas; the locations of echos described by \eqnref{radar} and the thermodynamic properties of the medium described by the ideal gas law.
%While this approach seems reasonable it suffers from a lack of internal consistency. 
%An ideal gas heats when compressed, and this alters the speed of sound.  
%Such physics is unmeasurable with ultrasound, even if the underlying model is correct.
%On the one hand the measurement processes assumes that the sound speed is fixed, 
%while on the other the properties of the medium imply that is not.  
%The measurements made with ultrasound will not agree with the predictions of a physically correct model due to ultrasound's reliance on assuming a constant sound speed.


%Put another way, 
%such an approach implies that ultrasound measurement alone cannot measure the properties ascribed in the model.  

%An alternative is to focus on modelling what is measurable - thereby insisting on the internal consistency of the physics.
%Since ultrasound requires that the speed of sound be constant as part of its measurement process, 
%the approach insists that the  model of the medium be consistent with this requirement.
%The equations of state are therefore determined by what can be measured acoustically, 
%rather than driven by what has been shown to be successful in other domains of physics.

%Both approaches are difficult to accept, 
%because in both we have to accept a distinction between 
%Such an approach refute 
%We show in \secref{nonlinear} that this implies that we model fluids with an the adiabatic index of unity.



%Consider for example the acoustic measurement of a compression of the gas in \figref{compression}.
%When the valve compresses the gas the  ideal gas is compressed, it heats up and this increases the speed of sound. 
%However, this change in sound speed is not reflected in the measurement process, and so the physical theory predicted 
%The acoustic measurement processes continues with the original sound speed and so the measured compression is incorrect.  
%The time it takes for a compression phase of the sound wave to return differs from that of the rarefaction.  


%The difficulty with this is that 

% to try to ensure that the underlying thermodynamic properties of the medium are
%If these thermodynamic properties are assumed to take their real, independently measured values then the implied local speed of sound will in general be different than the sound speed assumed in ultrasound.  
%Accordingly, the location where these perturbations are calculated - from their real values - differs from the location that is measured in ultrasound.  
%One would then have to maintain a mapping between what is measured acoustically, with a detailed knowledge of the medium so the real thermodynamic properties can be maintained correct.
%If such a detailed knowledge of the measured medium exists, then the need for acoustic measurement at all is deminished.




%The measured locations of these perturbations will then differ from where the fluctuation actually occurred.  In short, we have a contradiction - we ascribe a   valuesare measured independently from the acoustic properties of the medium, 
%then these properties will imply 



%%% Local Variables: 
%%% mode: latex
%%% TeX-master: "../../tshorrock_thesis"
%%% End: 

% LocalWords:  advection
