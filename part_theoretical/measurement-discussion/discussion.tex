\chapter{Digression}\label{ch:digression}

\subsection{An acoustic Michelson-Morely  experiment.}

\todo[inline]{Simplify for just change in reference.}


The discussion so far has been somewhat abstract.
It is not clear to what degree the motion of the bulk matters.
To make it concrete it is useful to discuss a simple pulse-echo experiment and  compare the two viewpoints
- the Galilean%
\footnote{Formally the `Galilean' measurements are the distances and times that are  measured with light signals in accordance to Einstein's method\cite{Einstein1905}.
In ultrasound experiments, however, the Galilean approximation is entirely appropriate.}
 world that is {\em seen}, with the world that is measured with ultrasound. % - on a simple pulse echo experiment.
%To do so, an acoustic version of the  Michelson-Morely experiment is considered.
%For this we %
%
% apply the acoustic definitions of time and space, equation~\ref{eqn:radar}, to a simple example.
%This is described in \secref{MMsetup}.

%The analysis of the experiment is done in two parts.
%The first, in \secref{MMGalilean}, is the from the perspective of a `Galilean' observer that {\em looks} at the setup,
%measuring distances with a ruler and times with a single oscilloscope.
%Formally the `Galilean' measurements are the distances and times that are  measured with light signals in accordance to Einstein's method.
%In ultrasound experiments, however, the Galilean approximation is entirely appropriate.
%This is the usual viewpoint from which ultrasound experiment are described.

%The second perspective, in \secref{MMLorentzian}, is taken by considering only  acoustical measurements - the pulse echo times $\tm$ and $\tp$ -
%and using the definitions of equation~\ref{eqn:radar} to determine measured spatial and temporal locations of distant entities.
%This is the viewpoint from which ultrasound experimental data is usually collected.

%\subsubsection{The experiment}




% Since acoustic waves propagate so much more slowly than light,
% an acoustic Michelson-Morely type experiment need not be an interferometry experiment.
% Rather the time it takes for a short pulse to propagate and return can be measured directly.


% The (hypothetical) apparatus  that is to be used in this discussion is illustrated in \figref{MMapparatus}.
% The transducer is capable of generating and receiving an acoustic pulse.
% The central acoustic reflector can be opened or closed 
% The  acoustic source/reciever

% The acoustic source and acoustic receiver are two separate devices that are mounted onto a rail 
% on which both the source and receiver may be translated.
% Both the source and receiver are directional in that they emit and receive in the forward direction only.
% The sound emitted is a short burst.
% Both the source and receiver are facing  an acoustic reflector that is placed in the parallel to the rail on which the source and receiver  translate.
% The shortest distance between the reflector and the rail denoted $l$.
% The whole setup may be rotated, and it is placed in an ideal fluid to propagate the sound.

 \begin{figure}[t]
      \centering
     \subfloat[Apparatus when bulk medium is stationary.]{
           \label{fig:setupA}
           \includegraphics{Michelson.0}}
\hspace{2cm}
     \subfloat[Apparatus when the laminar flow of the bulk medium is $v$]{
           \label{fig:setupB}
           \includegraphics{Michelson.1}}
      % \subfloat[\unit{100}\kilo\pascal]{
      %      \label{fig:R1vel}
      %      \includegraphics{velocity_r2_f2_a0.1.3}}
\label{fig:setups}
      \caption{A pulse-echo experiment when there is, and is not, a relative laminar flow past the apparatus.}
 \end{figure}
The first case to be considered is illustrated in \figref{setupA}.
This apparatus is appropriate when the equipment is stationary with respect to the bulk flow of the medium.
It is  analogous to  Michelson and Morely's famous experiment:
a piezoelectric transducer  replaces both the light source and the receiver while a medium that partially reflects sound  replaces the semi-silvered mirror.
The distance between $A$ and $B$ is denoted $l$ and is the same  as the distance between $A$ and $C$.
In the following the time it takes for the sound to propagate from $A$ to $B$ and back again is compared with the to-fro times between $A$ and $C$.


If the apparatus of \figref{setupA} were not stationary with respect to the bulk flow of the medium then the experiment would fail.
This is because the sound would not travel from  $A$ to $B$ and return;
the motion of the medium would drag the sound pulse with it.
The setup illustrated in \figref{setupB} gives spirit of the Michelson-Morely experiment for the case when the apparatus is not stationary with respect to the bulk flow.
In this case there are two separate partially reflecting surfaces.
%$t_{AB}$ is the time that it takes sound to propagate from $A$ to $B$.
The time it takes the sound to propagate from $A$ to $B$ to $A^\prime$ is now compared with the time it takes the sound 
to go from $A$ to $C$ to $A^\prime$.

When the to-fro times along the two arms are the same, irrespective of the flow  of the bulk medium,  the result is described as {\em null}.
This is in accordance to the description of the Michelson-Morely result.

\subsubsection{A Galilean interpretation}\label{sec:MMGalilean}

 \begin{figure}[t]
      \centering
     \subfloat[Observer that is stationary with respect to the medium.]{
           \label{fig:StatGA}
           \includegraphics{Michelson.2}}
\hspace{2cm}
     \subfloat[Observer moves at speed $v$ with respect to the medium]{
           \label{fig:MovingGA}
           \includegraphics{Michelson.3}}
      % \subfloat[\unit{100}\kilo\pascal]{
      %      \label{fig:R1vel}
      %      \includegraphics{velocity_r2_f2_a0.1.3}}
\label{fig:GalileanA}
      \caption{Observed motion of apparatus for the setup of \figref{setupA}.}
 \end{figure}

First we consider the case of the apparatus being stationary with respect to the bulk flow (\figref{setupA}).
If the propagation of the sound pulse were observed by a Galilean observer that is also stationary with respect to the flow
then \she would observe the sound travelling according to \figref{StatGA}.
The time, $t_{AB}$,  it takes for the sound to propagate from $A$ to $B$ is the same as the time, $t_{BA}$, it takes the sound to propagate from $B$ to $A$.
It is given by $l/c$, where  $c$ is the speed of sound of the medium.
This time interval is the same for the to and fro paths between $A$ and $C$,
\begin{align}
  t_{AB}=t_{BA}=t_{AC}=t_{CA}=l/c\label{eqn:setupA:stationary:Tab}.
\end{align}

An observer for whom  both the medium and apparatus flow past at a speed, $v$, will measure the same time intervals 
but will witness an altogether more complicated experiment.
The acoustic paths that will be observed are illustrated in \figref{MovingGA}.
When the sound travels between $A$ and $B$ the observer will record that the sound travels at an effective speed of
\begin{align}
\label{eqn:ceffone}
c_\eff(v) = \sqrt{c^2 +v^2}.
\end{align}
This is due to  the  contribution of the  laminar flow.
Additionally, the measured distance between $A$ and $B$ will be greater by  $\sqrt{l^2+v^2t_{AB}^2}$.
The increased distance and increased speed cancel so that 
\begin{align}
  \label{eqn:setupA:moving:Tab}
  t_{AB} = t_{BA} = \frac{\sqrt{l^2+v^2t_{AB}^2}}{\sqrt{c^2 +v^2}} = \frac{l}{c},
\end{align}
as before.

%When the medium and apparatus are moving with respect to the observer, 
The bulk flow will also contribute to the effective speed of the pulse from $A$ to $C$  ($c_\eff = c+v$) 
and hinder  the return from $C$ to $A$ ($c_\eff = c-v$).
However, this is again exactly compensated by changes in the total distance that the moving observer measures.
As is illustrated in \figref{MovingGA}, the total distance from $A$ to $C$ is $l+vt$. % when the pulse passes $A$, the position $C$ is seen to be moving away at a speed of $v$ - the speed of the bulk flow - 
%and so effective distance between $A$ and $C$ is $l+vt$.  
When the sound travels from $C$ to $A$ the total  distance is $l-vt$. % because the sound pulse is now travelling the opposite direction.
%%
%
%the travelling from  $A$ to $C$ is also viewed by this observer as having further to travel: a distance of $l+vt$.
%The return journey is looks shrunk for this observer, a distance of $l-v$.
Therefore the measured times are
\begin{align}
  \label{eqn:setupA:moving:Tac}
  t_{AC} =  \frac{l+vt_{AC}}{c+v}= t_{CA} =  \frac{l-vt_{CA}}{c-v}= \frac{l}{c}.
\end{align}

Next, we must check that these timings still hold when the apparatus is moving with respect to the medium (\figref{setupB}).
%The analysis  proceeds similarly to before.
The equivalence of \figref{setupB} and \figref{MovingGA} demonstrates this.
An observer that is stationary with respect to the apparatus (and moving with a speed, $v$, with respect to the medium) will record,
\begin{align}
  \label{eqn:setupB:moving:Tab}
  t_{AB} = t_{BA^\prime} =  \frac{\sqrt{l^2+v^2t_{AB}^2}}{\sqrt{c^2 +v^2}} = \frac{l}{c},
\end{align}
and 
\begin{align}
  \label{eqn:setupB:moving:Tac}
  t_{AC} =  \frac{l+vt_{AC}}{c+v}= t_{CA^\prime} =  \frac{l-vt_{CA^\prime}}{c-v}= \frac{l}{c}.
\end{align}
Equations~\ref{eqn:setupB:moving:Tab} and \ref{eqn:setupB:moving:Tac}  are exactly the same results as equations~\ref{eqn:setupA:moving:Tab} and \ref{eqn:setupA:moving:Tac},
respectively.
If the observer is instead stationary with respect to medium then it is easy to see that equations~\ref{eqn:setupA:stationary:Tab}  are repeated.

In summary, we find that the time it takes the sound to propagate from $A$ to $B$ and back again is
identical to the time it takes the sound to propagate from $A$ to $C$ and back,
irrespective of the speed of the observer with respect to the medium.
The acoustic Michelson-Morely experiment should yield a  {\em null} result.
%A {\em null} result  the acoustic Michelson-Morely experiment is correct.

\subsubsection{An acoustic interpretation}\label{sec:MMLorentzian}

Unlike the Galilean observer, the ultrasound physicist cannot directly measure the propagation of sound.
The sound path of a pulse-echo experiment must be inferred afterwards from the measurements and the  definitions of equation~\ref{eqn:radar}.
In order to  predict a  sound path the ultrasound physicist must have further {\em a priori}  knowledge,
which we assume here to be the dimensions  of the apparatus.

%When distances are measured acoustically  it is difficult to interpret measurements that are not stationary with respect to the medium. %previous experiment is  difficult to interpret.
%This is because the effective sound speed (such as is  used in equation~\ref{eqn:ceffone})
%is contradictory to the acoustic definition of time and space given in equation~\ref{eqn:radar}.

Let us again consider the experiment of \figref{setupA},
where all the apparatus is stationary with respect to the bulk flow of the fluid.
The sound path illustrated in  \figref{StatGA} is the simplest through the apparatus and we assume
that this path is predicted.
%
%In order to {\em test} the predicted sound paths the ultrasound physicist must have further {\em a priori}  knowledge,
%which we assume to be the dimensions  of \her apparatus. % as obtained from the manufacturer's specification, for example.
To test this prediction the ultrasound  physicist measures the to-fro times between $A$ and $B$ and between
$A$ and  $C$.
Both of these times  are equal to $2l/c$ (equation~\ref{eqn:setupA:stationary:Tab}),
which is  consistent with the known lengths, $l$.
The predicted sound paths are to this extent confirmed.


Let us now suppose that the  same setup is measured by a transducer that moves uniformly at a speed, $v$, with respect to the medium and apparatus.
Again with knowledge of the apparatus, we assume that the ultrasound physicist  predicts the simplest path.
This is the path illustrated in \figref{MovingGA} and is the same path that is measured by the  moving Galilean observer.
%Note that while the ultrasound physicist would measure the fluid flowing past the transducer,
%they do not have to consider themselves in motion with respect to the bulk flow.
%Instead, they can  consider themselves to be stationary with the bulk flow 
%and the measured fluid motion as nothing more than a local disturbance.  
%
%
A difference from the Galilean case arises,
however,
because the predicted path is subject to the rules of the measurement system
and, for the ultrasound physicist,  sound always propagates at a  constant speed, $c$.
For the propagation time between $A$ and $B$ the ultrasound physicist therefore predicts (c.f. equation~\ref{eqn:setupA:moving:Tab})
\begin{align}
  \label{eqn:setupA:moving:Tab:acoustic}
  t_{AB} = t_{BA} =  \frac{\sqrt{l^2+v^2t_{BA}^2}}{c} \implies t_{AB} =  t_{BA} =\frac{1}{\sqrt{1-v^2/c^2}} \frac{l}{c}.
\end{align}
%In this case $t_{AB}$ is larger than before but only by a small amount.
For the sound pulse between $A$ and $C$  they  predict
\sub{
\begin{align}
 t_{AC} =  \frac{l+vt_{AC}}{c}\implies t_{AC} = \frac{l}{c-v}
\end{align}
and 
\begin{align}
 t_{CA} =  \frac{l-vt_{CA}}{c} \implies t_{CA} = \frac{l}{c+v}
\end{align}
}
rather than  equation~\ref{eqn:setupA:moving:Tac}.
Therefore the total to-fro time between $A$ and $C$  is   predicted   to be
\begin{align}
\label{eqn:setupA:moving:Tac:acoustic}
t_{AC}+t_{CA^\prime} = \frac{1}{{1-v^2/c^2}} \frac{2l}{c}.
\end{align}

These predictions are of course wrong.
Equation~\ref{eqn:setupA:moving:Tab:acoustic} and \ref{eqn:setupA:moving:Tac:acoustic} do not agree with the
experimentally measured intervals.
The reassignment $c_\eff \rightarrow c$ made by the ultrasound physicist 
has resulted in predicting time intervals for the sound to traverse between $A$ and $B$ and between $A$ and $C$ that are too large
by a factor of $\gamma$ and $\gamma^2$ respectively,
%Unfortunately, both the inferred times for $t_{AB} + t_{AB}$ and $t_{AC}+t_{CA}$ given in \eqnref{} and \ref{} are wrong.
%They are, as comparison with \eqnref{} and \ref{} demonstrates, too large by a factor of $\gamma$ and $\gamma^2$ respectively,
where 
\begin{align}
  \gamma = \frac{1}{\sqrt{1-v^2/c^2}}
  \label{eqn:gamma}
\end{align}
is the Lorentz factor.
Moreover, the to-fro time between $A$ and $B$ is not predicted to equal  the to-fro time between $A$ and $C$,
in contradiction to the  experimental result.
This predicament faced by the ultrasound physicist  is, of course, the same as that which faced Lorentz, \Poincare and Einstein at the beginning  of the twentieth century.



The error is clear to the Galilean observer:
the ultrasound physicist has been forced by the measurement process to  set the effective speed of sound to equal $c$.
To solve the problem the ultrasound physicist  must  compensate for the wrong sound speed by rescaling the temporal and spatial units  used when modelling.

The ultrasound physicist, who cannot measure variations in the sound speed,
must work a little harder to come to this conclusion.
The first explanation that they might try  is to doubt the apparatus.
%The first and most simple explanation for the wrong prediction is to doubt the apparatus.
If the distance between $A$ and $C$  was actually a factor of  $\gamma^2$ shorter than the manufacture claimed then the predicted time for that path would match the 
experimentally measured value.
Likewise, all would be well if the distance between $A$ and $B$  was shorter by a factor of $\gamma$.
However, this explanation can be shown to be incorrect by  counting  the number of cycles of the sound wave that propagate through each arm of the apparatus.
This can be done straight-forwardly with ultrasound,
the  pulse length is simply increased until the received signal starts to interfere with the emitted signal.
%The pulse length is simply increased until the  received signal starts to interfere with the  emitted signal.
The experimental result would be that  $n = 2l/\lr{cT}$ cycles fit both  between  $A$ and $B$ and between $A$ and $C$, where $T$ is the period of the sound wave.
This result implies that the distance between $A$ and $C$ is not simply shorter than between $A$ and $B$, % by a factor of $\gamma$,
for then the number of cycles along each path would be different.
Rather, it implies that {\em all} distances are shorter in the $A$-$C$ direction, including the wavelength of the sound wave.
That is, parallel to the motion the {\em   unit of distance} is contracted  by a factor of $\gamma$.


If the ultrasound physicist incorporates the  number of pulses into  equation~\ref{eqn:setupA:moving:Tab:acoustic}
then they would predict that the  period between $A$ and $B$ is
\begin{align}
 T_\us= \gamma \frac{l/n}{c} = \gamma T,
\end{align}
where $T_\us$ distinguishes the predicted period from the experimentally measured period $T$.
That is to say, the {\em  unit of time} used in the model must be scaled  by a factor of $\gamma$ in order to agree with experiment.
%This result can occur  if the  frequency of the ultrasound pulse is shifted by the factor of $\gamma$.
%However, this interpretation does not make sense physically because the frequency of the pulse is a function of the piezoelectric crystal in the transducer.
%The only alternative explanation is that the {\em  unit of time} itself is scaled  by the factor of $\gamma$.

%The first of these interpretations does not make sense physically because the frequency of the pulse is a function of the transducer piezoelectric crystal.




% The first thing to be done is send  the apparatus back to the manufacture with an angry note.
% Clearly the arms have been made too short. 
% It should have taken $\gamma l/c$ to traverse between $A$ and $B$ whereas in fact it took only a time of $l/c$.
% (The calculation and the results are included in the note).
% However, no refund will be given.
% It will be pointed out that 
% for 
% It will be pointed out that returned signal 







% The ultrasound physicist, therefore,  needs to re-write equation~\ref{eqn:setupA:moving:Tab:acoustic} as
% \begin{align}
%  t_{AB}^\ast+t_{BA}^\ast  &= 2\frac{\sqrt{l^2+v^2t_{AB}^2}}{c},\label{eqn:setupA:moving:Tab:acoustic:rescaled}
% \end{align}
% where $t^\ast$ are the {\em inferred} temporal intervals of the physicist using acoustic measurements.
% By comparing \eqnref{setupA:moving:Tab:acoustic:rescaled} with \eqnref{setupA:moving:Tab} it is clear that
% \begin{align}
%  t_{AB}^\ast+t_{BA}^\ast = \gamma^{-1}\lr{t_{AB}+t_{BA}}.\label{eqn:LorentzTime}
% \end{align}
% %When the effective speed of sound  is different to  the speed of sound of the medium
% %the acoustic observer has no choice but to re-scale their base unit of time.
% That is, when an acoustic observer is in motion with respect to the medium 
% the temporal interval must be smaller by the Lorentz factor:
% There must be fewer ticks to their clock than for a Galilean observer.
% %There have been  must revert to a {\em local-time} that is not the same as the time measured by a Galilean observer.

% %The interpretation of the pulse echo between $A$ and $C$ is more straightforward.
% For  ultrasound measurements to match equation~\ref{eqn:setupA:moving:Tac} 
% an {\em inferred} distance  must also be introduced: %If the time for the sound to travel between $A$ and $C$, 
% %In this case the observer knows that the total distance that the sound must travel is $l +vt + l - vt = 2l$.
% %Therefore 
% %\begin{align}
% %  t_{AC}+t_{CA} = \frac{2l}{c},
% %\end{align}
% %in agreement with \eqnref{}.
% %Unfortunately the acoustic observer as already had to concede that since they are in motion with respect to the medium,
% %the local-time is the unit of time that agrees with the  experiment.
% %To compensate for this in \eqnref{}
% %the unit of distance must be scaled by the same factor so that
% %is to become
% \begin{align}
%   t_{AC}^\ast+t_{CA}^\ast = \frac{2l^\ast}{c},
% \end{align}
% where 
% \begin{align}
% l^\ast = \gamma^{-1} l.\label{eqn:LorentzSpace}
% \end{align}

As before, the equivalence of \figref{setupB} and \figref{MovingGA} guarantee that the same conclusion 
would be drawn when the apparatus is not stationary with respect to the bulk flow.

The comparison between the Galilean and experimental observer can be summarised as follows:
when modelling an ultrasound experiment 
%where the acoustic source is not stationary with respect to the medium,
the  unit of distance used in the model must be
contracted by the Lorentz factor in order to agree with experimental results,
and likewise the unit of time must be reduced by the Lorentz factor.
These are the results of \Poincare's special relativity.
%The   contraction in length is postulated by \Poincare's special relativity.
\Poincare's postulated contraction in length is the manifestation of the dependence of the sound speed upon
the flow of the medium.
It  exists because  the speed of a  signal that is used to measure distances must be assumed to be a constant,
not because the speed is constant but because distances cannot be measured otherwise.



%To finish this section we note that flow of the medium can be chosen arbitrarily.
%
%
% ask what changes  to our reasoning must be made if the flow in the acoustic medium is not uniform.
%In this case the notion of a  bulk flow of the medium is not well defined and we have no rule with which to define the 'stationary' frame of reference.
%We consider what happens if the stationary frame is arbitrarily assigned.
%To do so, we reconsider the above example for the case that the ob
%and so there is no stationary frame may be assigned 
%
% the laminar flow considered here is local, 
%and is different from the far-away bulk flow of the medium.%
%
%That is, what if the flow carrying the 


% \subsection{Calibrating time in an ultrasound experiment}

% %Equations~\ref{eqn:LorentzTime} and \ref{eqn:LorentzSpace} are the Lorentz transformations of space and time that result from \Poincare's special relativity.

% When an ultrasound transducer moves with respect to the bulk flow the oscilloscope should be re-calibrated to the local time.
% Failure to do repeats the mistakes of the past and results in physical predications that err with experiment.
% %The mistakes of the past are repeated when this is not done.
% %When ultrasound physicists fail to re-calibrate their oscilloscopes to a local time 
% %when their transducer's move with respect to the bulk flow they are destined to repeat the mistakes of the past.


%It may be noted at this point that ultrasound physicists are not in the habit of 
%re-calibrating their oscilloscopes to a local time.
%Firstly we note 
%Finally, it may be noted that when ultrasound measurements the 
%time axis (measured on the oscilloscope) is rarely re-calibrated to the local time
%of a transducer moving with respect to the bulk fluid.
%All that can be said to this is that where this is true,
%the physicist is destined to predict the wrong physics in 
%by repeating of the mistakes of the past.
%However, in practice, 
%the need for such re-calibrations is rare - 
%for in general the transducer is chosen to be at rest with respect 
%to the bulk flow of the medium.

%\subsection{Comparison between acoustic relativity with \Poincare's relativity}\label{sec:comparison}



\section{Discussion}
\todo[inline]{Comment that in this thesis focus on modelling ultrasound rather than what is real.
We ignore non-linear propagation - partly for reasons in the next chapter..
Far from the acoustic sources the effect diminishes. }

model what is seen in ultrasound rather than what is real

The measurement rules of equations~\ref{eqn:radarTime} and \ref{eqn:radarDistance} enable two properties of the world as measured by ultrasound to be stated immediately.
The first is that an entity that moves away from the transducer at a speed that is faster than the speed of sound (with respect to the bulk flow of the medium) 
cannot be measured.  
This is not because such motions are impossible but because the sound will never catch  up with the entity and so there will never be an echo to record.

The second is that ultrasound is not capable of  measuring variations in the speed of sound.
%In order to measure distances (and therefore speeds) the speed of sound must be known {\em a priori}.
%Fluctuations in the sound speed  cannot be known without further {\em a priori} knowledge of the medium. 
Since the sound speed must be known before any distance can be measured,
changes in the sound speed  cannot be measured. 
Changes may only be determined  with additional {\em a priori} knowledge of the structure of the medium. 
In acoustics, when distances and times are measured with light,
fluctuations in a medium's density result in fluctuations in the speed of sound,
and since sound is itself a fluctuation in density, 
non-linear sound speeds result in compressible mediums.
But these fluctuations cannot be measured with ultrasound,
and it follows that the acoustic medium must be incompressible (in the relativistic sense\cite{Pekeris1976, Pekeris1977, Taub1978})
and that  sound must propagate according to a linear wave equation. % when the acoustic definitions of time and space are used.
In \secref{Maxwell} it is demonstrated that this linear relation is identical to Maxwell's relation of electromagnetism.

The ultrasound literature does not comply with these remarks.
Currently, when modelling an ultrasound experiment, a fluid medium is always described by a Galilean invariant theory such as Euler's equation or the Naiver-Stokes equation.
The resulting model is then capable of predicting motions that are faster than the speed of sound
and predicts that  a sound pulse  propagates according to  a non-linear wave equation.
Both of these predictions are impossible when the world is measured with sound.
%Such a description is only meaningful only when distances and times are to be defined with light, 
%which travels at such a tremendous speed that the Galilean approximation is appropriate.%
%
%This is appropriate only if distances and times are to be defined with light, which travels at such a tremendous speed that the Galilean approximation is appropriate.
%If distances and times are to be defined with sound, as is done by the ultrasound scanner,
%then the Galilean description of the world is meaningless, in the sense that it predicts  results that are impossible to verify.
The ultrasound literature fails to recognise the distinction between  two equally valid descriptions of the world -
the world that is seen
and the world that is heard.
Curiously, ultrasound physics repeats the   fallacy that  the world must be seen to be believed.
