
\chapter{The pulsations of a bubble as measured with  ultrasound}\label{ch:measurement}


\section{Introduction}\label{sec:measurement:introduction}

Current models that describe the pulsations of a bubble in an acoustic field do not 
account for how measurements are made with ultrasound.
In particular, how the finiteness of the sound speed influences the spatio-temporal locations attributed to echo sources,
and limits the maximum velocities that can be measured.
Measurements made acoustically with ultrasound differ from those
made optically with a microscope due to
\hl{the assumptions that are imposed on the measurements by ultrasound}.
%When a model is proposed for the pulsations of a bubble
%the system of measurement that is to be used to test that model must also be specified.
%Current models have tacitly assumed that tests will be made with a microscope.
In this chapter the Keller-Miksis model 
- taken to be representative of current bubble pulsation models -
is altered to predict for the first time the pulsations that are  measured with ultrasound.\todo{clarify whether defined operationally of physically}
\hl{
  The modelled pulsations are operational in the sense that they represent what is deduced from an ultrasound experiment
  rather than the actual pulsations of the bubble.
  However, these are the correct quantities to work with when using the measurements of ultrasound
  to understand the interactions and dynamics of the measured entities.
}

%To \hl{model pulsations that are acoustic measurable} it is recognised that measurements in ultrasound should Lorentz invariant.
%be subject to the considerations of special relativity,
%where the speed of sound rather than the speed of light takes the role of the limiting velocity.
%This is because the pulse-echo definitions of time and space that are used in ultrasound are identical to the radar definitions used by Einstein.
%Furthermore, since all ultrasound measurements are temporal,
%a fixed and constant speed of sound is necessary before anything can said regarding distance.
%Einstein's second postulate {\em must} be assumed for ultrasound to measure anything at all.
%and there is no choice but to accept the resulting discrepancies with other imaging modalities.
%The only assumption made is an insistence upon invariance to inertial motions
%(Einstein's first postulate).
%It would be perverse for such a fundamental symmetry to be dependent on the choice of imaging modality.
%Not assuming this invariance would mark a fundamental departure for dynamics as when measured with ultrasound from
%any physics currently known.

Simulation results for the new model are presented and compared with the results from the original Keller-Miksis equation.
\hl{This is a comparison between what is measured and what actually occurs as seen under a microscope}.
The acoustically-measured-Keller-Miksis equation presented here correctly predicts 
that motions of the bubble wall that exceed the speed of sound cannot be observed with ultrasound.
The radial response of the two models is similar when the harmonic response of the bubble is not strong -
otherwise the pulsations are quite different when measured acoustically or with a microscope.


%\section{The acoustic measurement process}\label{sec:measurement:introduction}


%In medical ultrasound distances are measured using the time it takes a pulse of sound to propagate from a transducer
%to a reflecting object and then to return again. 
%%This interval is known as the pulse-echo time.
%If the sound is emitted from the transducer at a time, $\tm$,
%and the sound returns at a time,  $\tp$,
%then the task is to find from these two numbers the spatio-temporal location, $x$,
%of the point of reflection.

%What happens to the sound in between leaving the transducer and returning
%cannot be known by acoustic measurement.
%In this ignorance ultrasound practitioners assume that the time at which the echo 
%occurred is the midpoint of $\tm$ and $\tp$,
%\sub{
%\label{eqn:radar}
%\begin{align}
% \tau(x) &= \frac{\tp + \tm}{2}.
%\intertext{Other choices could certainly be made, 
%  but would imply a knowledge of the world beyond that learnt from $\tm$ and $\tp$ alone.
%  To measure distances from the times $\tm$ and $\tp$ a sound speed, $c$, is required.
%  Assuming, again in ignorance, that the sound returns at the same speed at which it left gives
%}
% \rho(x) &= \frac{\tp - \tm}{2}c. \label{eqn:radarDistance}
%\end{align}
%}
%These are the definitions of time and space that are used in ultrasound.
%%The assumption that the sound returns at the same speed as it left may be relaxed.
%%Then $\tau(x) =\epsilon \tp + (1-\epsilon)\tm$ and 
%%$\rho(x) = \epsilon \tp c - (1-\epsilon)\tm c$ for any
%%$0<\epsilon<1$.
%%This does not change any of the arguments that follow\cite{Debs1996}.

%Equation \eqnref{radarDistance} requires an {\em a priori} knowledge of the sound speed.
%This is of course obvious because ultrasound attempts to determine distances from temporal measurements.
%Nevertheless, 
%it does raise the question as to how the sound speed is to be found.

%To determine the speed of sound of a given medium an independent notion of length is required.
%This practically would be found with a set of callipers.
%The average speed of sound is then found from equation \eqnref{radarDistance},
%using the times measured and the predetermined ({\em a priori})  distance.
%The point is that %the distance measured by the callipers now represents the {\em a priori} knowledge.
%in order to use equation \eqnref{radarDistance} something must be taken for granted; 
%it is impossible to measure simultaneously both speeds and distances from temporal measurements.

%%All this is familiar. %Before ultrasound measurements can be made a calibration measurement to determine the speed of sound is required.
%%In order to use ultrasound to measure distances within people (without reaching for the callipers, and therefore for the scalpel)
%%it is necessary to first find the appropriate sound speed in the laboratory.
%%Bovine liver, typically, is used for this purpose, and is  cut into carefully measured thicknesses,
%%with care being taken to known a great deal about the sample before \eqnref{radarDistance} is applied.
%%[can you suggest a few  citations for here Jeff...]

%When using equation \eqnref{radarDistance} to measure distances the sound speed must be constant.
%This is because there is no way to measure its variations.
%This constancy of the speed of sound is identical to Einstein's  second postulate for special relativity\cite{Einstein1905},
%except that the sound speed takes the role of the speed of light.
%Likewise, Einstein used equations \eqnref{radar} for the definition of time and space, 
%except that  the  constant $c$ was understood to be the speed of light rather than the speed of sound.
%In the relativity literature equations \eqnref{radar} go by the names of radar-time 
%and radar-distance and their application to the example of the ``twin paradox'' is discussed by Dolby and Gull  \cite{Dolby2001}.

%In this thesis we  assume that the dynamics of entities is invariant to inertial motions even when measured acoustically (Einstein's first postulate).
%There has never been a counter example to this symmetry and we do not expect ultrasound to provide one.
%It then follows that measurements in ultrasound are subject to the considerations of special relativity,
%where the speed of sound rather than the speed of light takes the role of the limiting velocity.

%The surface of a  bubble that has been induced to pulsate by an ultrasound wave may collapse at a significant fraction of the speed of sound\cite{Neppiras1980}.
%The pulsations of a bubble as measured by ultrasound are therefore expected to disagree with the same pulsations measured by an optical techniques.
%To investigate this, we now derive and analyse a version of the Keller-Miksis model\cite{Keller1980} 
%- a commonly used model for a pulsating bubble - that is consistent with how ultrasound measures spatio-temporal locations.
%%This will be different than the current Keller-Miksis model that describes the pulsations of a bubble when observed optically under a microscope.


\section{The acoustically-measured Keller-Miksis model}
\subsection{The original formulation}\label{sec:KMoriginal}
The Keller-Miksis model\cite{Keller1980}  assumes that a gas bubble is located within a stationary and vorticity free fluid medium.
The fluid particles are described by the velocity potential, $\psi$.
The  bubble is assumed to remain spherical, with a  time dependent radius, $a \equiv a(t)$,
and assumed to remain at the origin.
From the spherical symmetry of the model only radial components of the velocity need to be considered. 

Keller and Miksis retain perturbations in density only up to first order, from which it follows that variations in the sound speed, $c$, are neglected
and that the velocity potential obeys the linear wave equation\todo{Fix: \eqnref{WaveEqn} not of correct form},
\begin{align}
\label{eqn:WaveEqn}
\lr{\frac{\d^2}{\d r^2} + \frac{2}{r}\frac{\d}{\d r} - \frac{1}{c^2}\frac{\d^2}{\d t^2}} \psi = 0.
\end{align}
The notation $\dr \equiv\frac{\d}{\d r}$ is used to denote the radial derivative
while both  $\dt \equiv \frac{\d}{\d t}$ and the over dot notation will be employed to denote the differential with respect to time.

The solution to \eqnref{WaveEqn} is
\begin{align}
  \label{eqn:Ansatz}
  \psi = \frac{1}{r}\lr{f_1\lr{t-\tfrac{r}{c}} + f_2\lr{t+\tfrac{r}{c}}},
\end{align}
where $r$ is the radial distance from the centre of the bubble
and $f_1$ and $f_2$ are functions to be determined.

Since equation \eqnref{WaveEqn} is second order two boundary conditions are required.
The first is that the radial velocity of the fluid, $v$, is equal to the velocity of the bubble wall.
That is 
\sub{
\begin{align}
  v = \dot a(t) \text{ at $r = a$.}
\end{align}
The second boundary condition is that the pressure in the liquid adjacent to the surface of bubble, $p(a,t)$,
must equal the pressure on the bubble wall, $p_b$,
\begin{align}
  p(a,t) = p_b. \label{eqn:pressureBC}
\end{align}
}
We will consider explicit expression for $p_b$ \secref{surface_pressure} and in \secref{viscosity}.

To apply these boundary conditions Keller and Miksis eliminated the spatial derivative of $\psi$ by using its definition,
\sub{
\begin{align}
  \dr \psi &= v, \label{eqn:VelocityPotNR}
\intertext{
and used Bernoulli's equation to eliminate the temporal derivatives,}
\dt \psi &= -\frac{1}{2}v^2 - h.\label{eqn:BernoulliNR}
\end{align}
}
$h$ is the enthalpy of the fluid.

The completes the specification of the model that was setup and solved by Keller and Miksis\cite{Keller1980}.

\subsection{Alterations required when making measurements with ultrasound}
\label{sec:measurement:alterations}
Equations that describe  measurements made with ultrasound require  Lorentz invariance.
Equations \eqnref{VelocityPotNR} and \eqnref{BernoulliNR} do not have this property and so 
do not apply to acoustical measurement.
To fix  \eqnref{VelocityPotNR} and \eqnref{BernoulliNR} the  relativistic generalisation to the velocity potential\cite{LandauBook} must be used,
\eqa{
  \label{eqn:RelVPot}
  \del \psi = -\frac{w u}{nc} \equiv -A.
}
Here $w$ is the heat function per proper volume and $n$ is the particle number per proper volume.
The velocity $u$ and the vector potential, $A$,  are spacetime vectors and $\del$ is the spacetime derivative.
This is the same as equation \eqnref{AisPotFlow} used in \chapref{medium}. 
The heat per proper volume can be written in terms of the total energy density, 
such that\cite{LandauBook, Doran2003}
\eqa{
  w = \epsilon + p = n m c^2 + nm e + p,
}
where $e$ is the thermodynamic energy and $m$ is the proper mass.
The inclusion of the rest energy is expected due to the discussion of \chapref{observables}.
The second equality uses equation \eqnref{relEpsilon} from \chapref{medium}.
Spacetime vectors are sometimes referred to as 4-vectors, although this terminology will not be used here.
The spacetime velocity, $u$, is parametrised so that $u^2 = c^2$ where $c$ is the speed of sound.
The symbol $u$ has been used to avoid confusion with the radial component, $v$, of the spatial velocity (a 3-vector).


%
%in order for the resulting equations to be consistent with the way ultrasound assigns spatio-temporal locations to echo sources.
The spatial and temporal projections of \eqnref{RelVPot} in the laboratory frame are
\sub{
\begin{align}
\dr \psi &= \frac{\gamma w v}{nc} \equiv \frac{\phi v}{c} \label{eqn:VelocityPot}
\intertext {and}
\dt \psi &=- \frac{\gamma w  c}{n} \equiv - \phi c \label{eqn:Bernoulli}
\end{align}
}
where  $\gamma = \lr{1-v^2/c^2}^{-1/2}$ is the Lorentz factor and  the potential 
\begin{align}
  \phi \equiv \frac{\gamma w}{n} \label{eqn:phiDefn}
\end{align}
has been introduced for convenience.
The heat function per particle is related to the enthalpy by the proper mass,
\begin{align}
  w/n = \lr{mc^2 + me + p/n} = m\lr{c^2 +  h}.
\end{align}
The thermodynamic relation $h = e + p/(nm)$ has been used in the second equality.

The potential $\phi$ is the relativistic generalisation to the total enthalpy (multiplied by the proper mass).
In the non-relativistic it becomes, 
\begin{align}
  \phi \rightarrow \lr{1+\tfrac{1}{2}\tfrac{v^2}{c^2}} \lr{mc^2 + mh} = mc^2 + \tfrac{1}{2}m v^2 + mh \text{\quad as ${v/c\rightarrow 0.}$}
\end{align}
The right hand side is the energy contributed by the rest mass plus the  standard non-relativistic expression for the  total enthalpy (multiplied by the mass),
as claimed.

By replacing equations \eqnref{VelocityPotNR} and \eqnref{BernoulliNR} with equations \eqnref{VelocityPot} and \eqnref{Bernoulli} the derivation of Keller and Miksis
can be used without further alteration.
The resulting equation will be Lorentz invariant and therefore satisfy the constraints imposed by using pulse-echo to  define the spatio-temporal locations of the echo's source.


\subsection{The derivation of the acoustically-measured-Keller-Miksis equation}
Differentiating equation \eqnref{Ansatz} with respect to time obtains
\begin{align}
  r \dt \psi =  f_1^\prime + f_2^\prime, \label{eqn:dtAnsatz}
\end{align}
where the prime denotes differentiation with respect to the argument,
while differentiating with respect to the radius gives,
\begin{align}
 r^2 c \del \psi = r \lr{f_2^\prime - f_1^\prime} - c\lr{f_1+f_2}. \label{eqn:drAnsatz}
\end{align}
Equations \eqnref{dtAnsatz} and \eqnref{drAnsatz}, evaluated at $r = a$, are combined to eliminate $f_1^\prime$,
\begin{align}
  a^2 \lr{\dt \psi + c \dr \psi} - 2 a f_2^\prime + c\lr{f_1 + f_2} = 0. \label{eqn:Combo1}
\end{align}
By using equations \eqnref{VelocityPot} and \eqnref{Bernoulli} in \eqnref{Combo1} (rather than \eqnref{VelocityPotNR} and \eqnref{BernoulliNR}) 
we obtain
\begin{align}
a^2 \lr{\phi \dot a -\phi c } - 2 a f_2^\prime + c\lr{f_1 + f_2} = 0.  \label{eqn:Combo2}
\end{align}
Differentiating \eqnref{Combo2} with respect to time and reusing \eqnref{dtAnsatz} and \eqnref{Bernoulli} gives
\begin{align}
  a \frac{\phi}{c^2} \ddot a + 2 {\dot a}^2\frac{\phi}{c^2} - \phi\lr{1 + \frac{\dot a}{c}} - a \frac{\dot \phi}{c}\lr{1-\frac{\dot a}{c}} - \frac{2}{c^2} f_2^\dprime\lr{1+\frac{\dot a}{c}} = 0
  \label{eqn:Combo3}
\end{align}

The driving acoustic pressure comes from the converging acoustic wave, $f_2$.
To evaluate this term, following Keller and Miksis, 
we assume that the incident wave is planar and decompose it into  spherical harmonics.
By assumption, however, the bubble pulsates purely radially and so only the zeroth harmonic interacts with the bubble.
The potential of the incoming wave, $\psi_i (r,t)$, is accordingly 
%\begin{align}
  $\psi_i (r,t) = \frac{1}{r}\lr{f_2(t+r/c) + f_3(t-r/c)}$.
%\end{align}
Requiring that the potential is finite at the bubble's centre implies that  $f_2 = -f_3$ and so
\begin{align}
  \psi_i (r,t) = \frac{f_2(t+r/c) - f_2(t-r/c)}{r}.
\end{align}
The bubble is small in comparison to the wavelength and so the velocity potential of the fluid near the origin satisfies
\begin{align}
  \label{eqn:Tau}
 \psi_i (a,t) = \frac{f_2(t+a/c) - f_2(t-a/c)}{a} \approx \frac{2}{c}f_2^\prime(\tau).
\end{align}
The differential on the right-hand-side of \eqnref{Tau} is a function of the proper time, $\tau$, of the bubble.
This is because the equation holds only in a frame of reference where the bubble is stationary and at the origin.
Differentiating \eqnref{Tau} with respect to the proper time we obtain
\begin{align}
  \frac{d}{d \tau} \psi_i \equiv \gamma \lr{\d_t+ v\dr } \psi_i = u \cdot \del \psi = \frac{2}{c}f^\dprime.
\end{align}
Using equation \eqnref{RelVPot}  this gives
\begin{align}
\frac{2}{c}f^\dprime =-\frac{wc}{n}= - \frac{\phi_i c}{\gamma}, \label{eqn:IncidentWave}
\end{align}
where $\phi_i$ is the incident  potential.
Substituting  \eqnref{IncidentWave} into \eqnref{Combo3} and using equation \eqnref{phiDefn} in the form $\phi = \gamma m\lr{c^2 + h}$ we obtain
%\begin{align}
%  a \frac{\phi}{c^2} \ddot a + 2 {\dot a}^2\frac{\phi}{c^2} - \lr{\phi-\frac{\phi_i}{\gamma}}\lr{1 + \frac{\dot a}{c}} - a \frac{\dot \phi}{c}\lr{1-\frac{\dot a}{c}}  = 0
%  \label{eqn:Combo4}
%\end{align}
\begin{align}
  a \ddot a & \gamma \lr{1 + \frac{h}{c^2}}\lr{1 - \frac{\dot a}{c}\gamma^2\lr{1- \frac{\dot a}{c}}}+ 2 {\dot a}^2\gamma\lr{1 + \frac{h}{c^2}}\nonumber\\
  &- \lrsquare{\gamma\lr{c^2 + h} - \lr{c^2 + h_i}}\lr{1 + \frac{\dot a}{c}}- \gamma \dot h \lr{\frac{a}{c}}\lr{1 - \frac{\dot a}{c}}
= 0\label{eqn:RKM}
\end{align}
Equation \eqnref{RKM} is the final answer. The enthalpy contains the pressure terms, and with them the boundary condition of equation \eqnref{pressureBC}.

\hl{
Equation {\ref{eqn:RKM}} describes the pulsations of the bubble that are measured with ultrasound.
They enable to bubble size, at any moment in time, to be derived from acoustical measurements.
These acoustically determined bubble sizes can then be fed into other models that determine the motions and interactions of the 
bubbles as measured by ultrasound.
}
\todo{Clarify if a is defined operationally or physically}


\hl{
  The pulsations predicted from equation {\ref{eqn:RKM}} will in general differ from the actual pulsations of the bubble as measured using a microscope.
  Nonetheless, within the self-contained system of what is measured by ultrasound, they are correct in the sense that they can be relied upon to make observable predications.
  In this sense the  acoustically-measured-Keller-Miksis is \emph{more correct} than the original, 
  for using the original model to make predications of what is measured by ultrasound will lead to logical contradiction. 
}
\todo{The sense to which the model is ``more correct'' (not exact)}


\hl{
The difference between what is measured acoustically and the true pulsations of a bubble vanish when the bubble wall is small.
}
As expected, in the  non-relativistic limit (with $\gamma \approx 1 + \frac{{\dot a}^2}{2c^2}$), equation \eqnref{RKM} reduces to
\begin{align}
  a \ddot a\lr{1 - \frac{\dot a}{c}}+ \frac{3}{2} {\dot a}^2\lr{1- \frac{1}{3}\frac{\dot a }{c}}
    - \lr{h - h_i}\lr{1 + \frac{\dot a}{c}}
  -  \dot h \lr{\frac{a}{c}}
= 0, \label{eqn:KM}
\end{align}
which is the original Keller-Miksis equation\cite{Hoff2001}.

%The acoustically-measured-Keller-Miksis model therefore reduces to the original when the maximum speed of the bubble wall is small in comparison to the sound speed.
In the non-relativistic approximation made to obtain equation \eqnref{KM} terms of order $M^2$ are discarded, 
where  $M = \abs{\frac{\dot a}{c}}$ is the Mach number of the bubble wall.
Therefore, the original Keller-Miksis model approximates the acoustically measured motion when $M^2$ is small.
I.e. %for acoustical measurements, the two equations may be used interchangeably 
up to about $M=0.4$.


%\subsection{The applicability of the equation}
%In the original derivation Keller and Miksis linearised the density. 
%The consequence of this is that fluctuations in the sound speed are ignored.
%This enabled our starting point, the linear wave equation of  \eqnref{WaveEqn}, to be derived.
%The discarded second order terms become important when $M^2$ is not small.
%The approximation holds, therefore, up to about Mach 0.4.
%Interestingly,  the optical and acoustical observer
%agree over the range of speeds to which the original Keller-Miksis equation is applicable.

%When the measurements are made with ultrasound,
%however, 
%no approximation is made to obtain equation \eqnref{WaveEqn}.
%The reason is that  ultrasound is not capable of  measuring variations in the speed of sound.
%Second order fluctuations in the density  can therefore play no role when ultrasound is used to describe the world.
%%It is instructive to see this argument borne out in the equations that describe an acoustically measured fluid.
%%The equations of motion must  be Lorentz invariant and this condition  is automatically fulfilled  when  the equations 
%%are obtained from the divergence of the energy-momentum tensor of the ideal fluid.
%%In the derivation we also need to use the hitherto unused  condition
%%that the sound speed takes the role  of the speed of light.
%%This condition is imposed by equating the two speeds.
%%This further requires that the energy density for the fluid, as measured acoustically, be a function of the pressure only (barotripic),
%%for the sound speed cannot equal the speed of light otherwise\cite{Taub1978}.

%The energy-momentum tensor of an ideal fluid is\cite{Doran2003}
%\eqal{
%  T(a) = w a \cdot u u - a p = (\epsilon + p) a \cdot u u - a p,
%}{EMtensor}
%where, $\epsilon \equiv \epsilon(p)$ is the barotropic total energy density,
%$p$ is the pressure
%and 
%$u=u(\tau)$ is the velocity vector of the spacetime path, with the parameterisation chosen such that $u^2 = 1$. % and $P$ is the pressure.
%Natural units have been chosen in this section, and so that the speed of light is unity.

%The speed of sound, $c$,  given at constant entropy density, $\sigma$, is\cite{LandauBook,Taub1978} 
%\begin{align}
%  c^2 = \given{\frac{\d p}{\d \epsillon}}{\sigma}. \label{eqn:soundspeed}
%\end{align}
%This is the same as the non-relativistic expression except that the energy density has replaced the mass density.
%Setting the speed of sound to equal the speed of light (unity) and the integrating 
% gives,
% \begin{align}
%   \epsilon = p^\prime - p_0 + \epsilon_0
% \end{align} 
%where $p^\prime$ is a pressure that fluctuates with position, so that $dp = dp^\prime$,
%and $p_0$  and $\epsilon_0$  are the ambient  pressure and mean energy density, respectively.
%%Rather can carry the constants $ p_0 $ and $\epsilon_0$ through the rest of the derivation we write
%The thermodynamic pressure is therefore
%\begin{align}
%  p \equiv p^\prime - p_0 + \epsilon_0. \label{eqn:pshort}
%\end{align}
%and the fluid obeys the equation of state, 
%\eqal{
%  \epsilon(p) = p.
%}{eos}
%At infinity $p^\prime = p_0$ 
%and so $\epsilon_\infty = p_\infty = \epsilon_0 \ne p_0$.

%%therefore enforces that the sound speed equals that of light (unity).
%%The notation of equation \eqn{eos} is quite compact.
%%For \eqnref{eos} to be consistent with the integral of the sound speed (equation \eqnref{soundspeed}),
%%the pressure $p$ must be interpreted as follows,
%%\begin{align}
%%  p = p^\prime - p_0 + \epsilon_0. \label{eqn:pshort}
%%\end{align}
%%Here $p^\prime$ is the pressure perturbation measured with a hydrophone,
%%$p_0$  and $\epsilon_0$  are the mean  pressure and energy density respectively.



%Applying \eqnref{eos} simplifies the energy momentum tensor considerably,
%\eqal{
%  T(a) =  p\lr{2 a \cdot u u  - a} \equiv  \frac{1}{4}  A a A,
%}{EMFluid}
%where vector potential, $A$, introduced on the right hand side satisfies
%\eqal{
%  A = 2p^{1/2}u =2 \epsilon^{1/2} u.
%}{defnA}

%The vector $A$ is the same as was introduced in \eqnref{RelVPot},
%i.e.
%\begin{align}
%  \del \psi = - \frac{wu}{n} = - 2\sqrt p u.
%\end{align}
%%where $n$ is the proper particle number density of the fluid and equation \eqnref{eos} has been used for the second equality.
%To demonstrate this second equality we  use an argument of Taub\cite{Taub1978}.
%Using the isentropic thermodynamic relation $m de = - p d\lr{\frac{1}{n}}$ with the relation for the internal energy, $\epsilon= nm( 1 + e(p))$,
%it follows that 
%\begin{align}
% n d\epsilon = \epsilon dn - n^2 p d \lr{\frac{1}{n}} = \lr{\epsilon + p} dn.
%\end{align}
%Applying equation \eqnref{eos} and integrating we obtain
%\begin{align}
%  n = \sqrt p, \label{eqn:nrootp}
%\end{align}
%from which we find
%\begin{align}
%A = 2\sqrt p  u = \frac{2\epsilon}{n} u = \frac{wu}{n} =  - \del \psi,
%\end{align}
%as claimed.


%The divergence of the energy momentum tensor (equation \eqnref{EMFluid}) vanishes in the absence of external fields.
%Therefore, by projecting the divergence of \eqnref{EMFluid} along the timelike component we find
%\eqa{
%  u\cdot\scope T(\scope\del)= \half  u\cdot A \del \cdot A = 0.
%}
%The check denotes the scope of the derivative.
%Since, from \eqnref{defnA}, $A$ is parallel to $u$  it follows that 
%\eqal{
%  \del \cdot A  =0
%}{eomTime}
%and so the vector potential $A$ is conserved.
%The spacelike projection, $\scope T(\scope \del) - u u\cdot \scope T(\scope\del)$, gives in turn,
%\eqal{
%  u \cdot \lr{\del \wedge A} = 0.
%}{eomSpace}
%The relativistic vorticity bivector is 
%\eqal{F = \del \wedge A,}{DefnVorticity}
%and so \eqnref{eomSpace} implies that the vorticity is orthogonal to the velocity.

%By taking the divergence of \eqnref{eomTime} and using the vector identity
%$\del \del \cdot A = \del^2 A - \del\cdot\lr{ \del \wedge A} $
% we find,
%\eqal{
%  \del^2 A = \del \cdot F = \del F.
%}{wave}
%The second equality follows because $\del F = \del \cdot F + \del \wedge F$ 
%and because the  operator identity $\del\wedge \del = 0$ causes $\del \wedge F$ to vanish.
%Equation \eqnref{wave} is a wave equation and so interpreting the right-hand-side of \eqnref{wave} as an acoustic source current, $J$, we obtain
%\eqal{
%  \del  F = J.
%}{Maxwell}
%Interestingly equation \eqnref{Maxwell} is Maxwell's equation and equation \eqnref{eomTime} has specified that we are working in the Lorenz gauge.
%We explore this observation more in \chapref{interactions}.
%%We do not explore this observation further here.


%When deriving the Keller-Miksis equation potential flow in the fluid was assumed.
%This implies that the vorticity tensor vanishes, for
%$F = \del \wedge A = -\del \wedge \del \psi = 0$.
%Therefore the acoustic sources vanish and the wave equation $\del^2 A =0$ is satisfied.
%This was the starting equation in the derivation of the Keller-Miksis model, 
%equation \eqnref{WaveEqn}. 
%Notice that other than potential flow, no approximation  has been made to derive \eqnref{WaveEqn}.
%%When using ultrasound, sound must be measured to propagate linearly.
%%This is simply because it is impossible for ultrasound to measure fluctuations in the speed of its propagating signal.
%The acoustically-measured-Keller-Miksis equation is exact.

%Since no perturbation scheme has been used to obtain \eqnref{WaveEqn} when the measurements are made acoustically 
%(in contradistinction to the case when optical measurements are made)
%the acoustically-measured-Keller-Miksis equation can claim validity over a much greater range of sound speeds than the original.
%The original equation may be obtained as an approximation to the acoustically-measured-Keller-Miksis equation when the sound speed is small.
%This approximation, interestingly, is to the same order as the (quite different) approximations of the original Keller-Miksis equation.
%Therefore both acoustic measurements and optical measurements agree that the original Keller-Miksis is valid so long as $M^2$ is small.



\subsection{The pressure on the surface of the bubble}\label{sec:surface_pressure}

When introducing the original Keller-Miksis model in \secref{KMoriginal} the pressure on the surface of the bubble was not specified (equation \eqnref{pressureBC}).
In this section and the next we fix this boundary condition for the cases with and without viscosity.


The gas within the bubble may be modelled with a polytropic exponent, $\kappa$. 
The pressure within the bubble is then $p_e \lr{\frac{a}{a_e}}^{-3\kappa}$ where $p_e$ is the pressure of the gas within the bubble at equilibrium,
and $a_e$ is the radius of the bubble at equilibrium\cite{Hoff2001}.
The contribution of the vapour pressure has been neglected for simplicity.
Within the bubble the pressure exceeds the pressure of the fluid at the surface, $p(a,t)$, due to the contributions of the surface tension, $\sigma$.
The pressure boundary condition is then
\begin{align}
  p_b(a) = p_e \lr{\frac{a}{a_e}}^{-3\kappa} - 2\frac{\sigma}{a}. 
\end{align}
To write the boundary condition in terms of the enthalpy we use the  thermodynamic relation $d h = \frac{1}{m n} dp$.
Substituting in $n=\frac{n_\infty}{\sqrt{p_\infty}}\sqrt{p}$, equation \eqnref{relN}, gives
\begin{align}
  h(a) =  \int_{p_\infty}^{p}  \frac{1}{m n}dp = 2\frac{\sqrt{p_\infty}}{mn_\infty}\lr{\sqrt{p}-\sqrt{ p_\infty}} = 2 \frac{p_\infty}{mn_\infty} \lr{\sqrt{p/p_\infty} - 1} 
\end{align}
%where  $n_\infty = \sqrt{p_\infty} = \sqrt{\epsilon_\infty} $ has been used in the second equality.
In the non-relativistic limit $p_\infty = \epsilon_\infty \approx n_\infty m c^2 \approx \unit{1}\giga\pascal$\cite{Hoff2001}.
This is much larger than any pressures reached in ultrasound and so we may write the pressure in terms of a fluctuations, $p^\prime$, around the ambient pressure at the surface of the bubble, $p_0$,
\begin{align}
  p = p^\prime - p_0 + p_\infty.
\end{align}
Then
\begin{align}
  h(a) =  2 \frac{p_\infty}{mn_\infty} \lr{\sqrt{1 + \frac{p^\prime - p_0}{p_\infty}} - 1} \approx \frac{p^\prime - p_0}{mn_\infty}
\end{align}
%In the non-relativistic limit $p_\infty = \epsilon_\infty \approx n_\infty m c^2 \approx \unit{1}\giga\pascal$\cite%{Hoff2001}.
%This is much larger than any pressures reached in ultrasound and so we may write, simply, that
%\begin{align}
%  h(a) \approx \frac{p - p_0}{mn_\infty}
%\end{align}
%
%
%The incident pressure $p_i$ is a perturbation about the  pressure at infinity $p_\infty$.
%Therefore
%\begin{align}
%  h_i(a) = 2\lr{\sqrt{p(a) + p_\infty }-\sqrt{ p_\infty}}.
%\end{align}
%
The acoustically-measured-Keller-Miksis equation is now complete. 
%We emphasise that for an free polytropic gas bubble pulsating in ideal vortex-free fluid, the model is exact.


\subsection{Acoustically-measured viscosity}\label{sec:viscosity}

Viscous fluids have so far been excluded from our analysis.
Viscosity usually plays a minor role in medical applications - 
an ideal fluid does a fairly good job for modelling propagation.
However, the viscosity does play an important role in dampening the oscillations of the bubble.
In ultrasound contrast physics, therefore, the viscosity is usually incorporated into the pressure boundary condition of equation \eqnref{pressureBC}.
% Even though viscous effects have been ignored to find \eqnref{WaveEqn} they are generally incorporated into the boundary conditions.
% If they were neglected altogether there would be no dampening mechanism in the final model.
% Transient solutions to equation \eqnref{WaveEqn} would never relax and numerical studies of the resulting equations predict exotic harmonics resulting from beating between the 
% stable and transient solutions of \eqnref{WaveEqn}.
% The  second boundary condition used by Keller and Miksis is therefore
% \begin{align}
% p(a, t) =  k a^{-3\gamma} + \frac{4\mu}{3}\lr{\dr \dot a - \frac{\dot a}{a}} - 2 \frac{\sigma}{a},
%  \label{eqn:BCPressure}
% \end{align}
% }
% where $\mu$ is the viscosity and $\sigma$ is the surface tension.
To model the viscous dampening it is assumed that the fluid is Newtonian; only the  {\em  dynamic viscosity}, $\eta$, is considered. %while the {\em second viscosity} is neglected\cite{LandauBook}.
Then the stress tensor measured with ultrasound is\cite{LandauBook}, 
\begin{align}
  \sigma(a) = a p - c\eta \lr{a \cdot \del u + \del u\cdot a - a\cdot u u\cdot \del u - u u \cdot \del  u\cdot a -\tfrac{2}{3}\del \cdot u \lr{a - a\cdot u v}}.\label{eqn:RelStress}
\end{align}
The derivative with respect to the vector $a$ yields the trace\cite{Hestenes1984},
\begin{align}
  \Tr\ \sigma(a) \equiv \d_a \cdot \sigma(a) = 4p. \label{eqn:RelStressTrace} 
  % 4 p -2 c\eta \lr{ \del\cdot v  -  u\cdot \scope\del \scope v \cdot v -  \del \cdot v } = 4 p.
\end{align}
%\begin{align}
%\sigma_{00}  &=  p - \tfrac{2}{3}\eta\lr{1-\gamma^2}\lr{2\dt \gamma + 2 \vu\cdot \vdel \gamma - \gamma \vdel \cdot \vv} 
%+ 2 \eta \vv \cdot \vdel \gamma,\\
%\sigma_{ii} \equiv \gamma_i \cdot \sigma(\gamma_i) &=  -p + 2 \eta\lr{ \vdel_i \lr{\gamma \vv_i} + \gamma \vv_i/c u\cdot \del \lr{\gamma \vv_i} + \tfrac{1}{3}c\del\cdot u \lr{1+\gamma^2 %\vv_i^2/c^2 }}
%\end{align}
We wish to evaluate the radial component of the stress tensor, $\sigma_{rr} \equiv \hat r \cdot \sigma(\hat r)$, where $\hat r$ is the radial unit vector.
This is because in our spherically symmetric model it is this component that dampens the oscillation.
From \eqnref{RelStress} this is evaluated to be 
\begin{align}
  \sigma_{rr}  &= -p + 2\eta \lrsquare{\dr v + \tfrac{\gamma^2 v}{c} \tfrac{c u\cdot \del }{\gamma}\lr{\tfrac{\gamma v}{c}} - \tfrac{1}{3} c\del\cdot u\lr{1 + \tfrac{\gamma^2v^2}{c^2}}}, \label{eqn:sigmarr1}
\end{align}
where $v$ is the radial component of the velocity.
To keep the expressions short  the inner products $ u\cdot \del = \frac{\gamma }{c} \lr{\dt + v \dr}$ and $\del\cdot u = \frac{1}{c}\lr{\dt\gamma + \dr(\gamma v)}$ have not been expanded.

Equation \eqnref{sigmarr1} can be simplified by noting that $\lr{1 + \tfrac{\gamma^2v^2}{c^2}} = \lr{1 - \gamma^2\lr{1-\tfrac{v^2}{c^2}} + \gamma^2 } =  \gamma^2$.
Using a similar trick we find,
\begin{align}
   \frac{\gamma^2 v}{c} \frac{c u\cdot \del }{\gamma}\lr{\frac{\gamma v}{c}} 
=-\lr{1-\gamma^2 } \frac{c u\cdot \del }{\gamma}\gamma +\frac{\gamma^3 v}{c^2} \frac{c u\cdot \del }{\gamma}v
=\gamma c u\cdot \del \gamma. 
\end{align}
The relation $d \gamma = \frac{\gamma^3}{c^2} v d v$ has been used to obtain the second equality.
Equation \eqnref{sigmarr1} then simplifies to
\begin{align}
  \sigma_{rr}  &= -p + 2\eta \lrsquare{\dr v + \gamma c u\cdot \del\gamma - \tfrac{1}{3}\gamma^2 c\del\cdot u}. \label{eqn:sigmarr2}
\end{align}
To make further progress we re-evaluate the trace using \eqnref{sigmarr2} and the other components of the diagonal of the stress tensor,
\begin{align}
  \sigma_{tt} &= p - 2 \eta \lrsquare{\dt \gamma - \gamma c u \cdot \del \gamma - \tfrac{1}{3} \lr{1-\gamma^2} c \del\cdot u    }\\
  \sigma_{\theta\theta} =  \sigma_{\phi\phi}  &= -p + 2\eta\lrsquare{\tfrac{v}{r} - \tfrac{1}{3} c \del \cdot u},
\end{align}
and find that 
\begin{align}
  \Tr\ \sigma(a) = 4p - 2 \eta \lrsquare{2 \tfrac{v}{r} + \dr v - \dr(\gamma v) }.\label{eqn:RelStressTrace2}
\end{align}
Equating \eqnref{RelStressTrace} with \eqnref{RelStressTrace2} implies that
\begin{align}
  \dr v = - 2 \frac{v}{r}  + \dr(\gamma v)
\end{align}
and so
\begin{align}
  \sigma_{rr}  &= -p - 4\eta \lrsquare{  \frac{v}{r}  -\tfrac{1}{2}\lr{1-\tfrac{1}{3}\gamma^2}\dr(\gamma v) - \tfrac{6\gamma^2}{5c^2} v \dt v  }. \label{eqn:sigmarr3}
\end{align}

In the non-relativistic limit this equals 
\begin{align}
 \sigma_{rr}  &= -p - 4\eta \lrsquare{  \frac{v}{r}  -\tfrac{1}{3}\dr v }, \label{eqn:NRsigmarr}
\end{align}
which is  the standard expression for the radial stress exerted on a bubble \cite{Hoff2001}.

The spatial derivative $\dr(\gamma v) $ in \eqnref{sigmarr3} may be evaluated from the relation $\dr (\phi v) = -\dt \phi$, 
obtained from the spatial projection of \eqnref{eomTime}  in the laboratory frame.
However, the resulting equation is complicated to evaluate and contributes of order $c^{-2}$ compared to the first term of \eqnref{sigmarr2},
which of itself is small if the viscosity of the fluid is low.
Therefore,
for analytic simplicity, we neglect the small terms on the right of 
 \eqnref{sigmarr3} and write
\begin{align}
 \sigma_{rr}  &\approx -p - 4\eta  \frac{v}{r}. \label{eqn:sigmarr4}
\end{align}
Equation \eqnref{sigmarr4} is of  adequate accuracy for our purposes and is used in the numerical studies that follow.
It has the additional virtue in that it is identical for both acoustical and optical measurement.

The pressure boundary condition is then
\begin{align}
  p_b(a) = p_e \lr{\frac{a}{a_e}}^{-3\kappa} - 2\frac{\sigma}{a} - 4\eta  \frac{\dot a}{r}. 
\end{align}

\section{Analysis of the equation}
 \begin{figure}[p!]
     \subfloat[\unit{100}\kilo\pascal]{
           \label{fig:R1rad}
           \includegraphics{radius_r2_f2_a0.1.0}}
      \subfloat[\unit{100}\kilo\pascal]{
           \label{fig:R1vel}
           \includegraphics{velocity_r2_f2_a0.1.3}}\\
     \subfloat[\unit{300}\kilo\pascal]{
           \label{fig:R3rad}
           \includegraphics{radius_r2_f2_a0.3.1}}
      \subfloat[\unit{300}\kilo\pascal]{
           \label{fig:R3vel}
           \includegraphics{velocity_r2_f2_a0.3.4}}\\
     \subfloat[\unit{500}\kilo\pascal]{
           \label{fig:R5rad}
           \includegraphics{radius_r2_f2_a0.5.2}}
      \subfloat[\unit{500}\kilo\pascal]{
           \label{fig:R5vel}
           \includegraphics{velocity_r2_f2_a0.5.5}}
      \caption{The calculated response of a two micron bubble to a \unit{1/2}\mega\hertz\ wave at various pressures
as measured optically (using the Keller-Miksis model) and acoustically (using the acoustically-measured-Keller-Miksis model)
      The radial response is shown in the figures on the left, the velocity response on the right.}
      \label{fig:BubbleResponse}
 \end{figure}
We finish by briefly examining the non-linear response of the two models to an incident sound pulse.
The strongly non-linear response of bubbles is important 
in medical applications because it is a property not  shared by the surrounding tissue,
and therefore provides a means of identifying the bubble.

To start, we compare  the response of the two models for 3 different pressures:
\unit{100}\kilo\pascal, \unit{300}\kilo\pascal\ and \unit{500}\kilo\pascal.
The equilibrium radius of the bubble is chosen to be \unit{2}\micro\meter, 
which, like the pressures, is typical for diagnostic ultrasound applications.
The pressure applied is a sinusoidal  with a frequency of \unit{500}\kilo\hertz.
The pulse consists of 10 cycles and the first and last quarter of the pulse is tempered with a cosine function.
The radial and velocity response of the bubble is plotted in \Figref{BubbleResponse}.




For the low incident pressure of \unit{100}\kilo\pascal\ the radial response of the bubble is essentially identical in the two models (\figref{R1rad}).
This is as would be expected, for as \figref{R1vel} shows,  at this pressure the velocity of the bubble wall 
is always a small fraction of the sound speed.

At the higher pressure of \unit{300}\kilo\pascal\ 
the original Keller-Miksis model predicts that the bubble wall collapses at very high velocities,
even surpassing the speed of sound on some occasions.
It is obvious that ultrasound measurements cannot measure the speed of a bubble wall undergoing  supersonic collapse, 
and so the predictions of the original Keller-Miksis equation is contrary to what is measured acoustically.
\Figref{R3vel} illustrates what ultrasound would measure when a bubble responds to the pulse.
The acoustically measured velocity is always slower than the sound speed.
The radial response as measured by ultrasound is predicted to be different to the response as measured optically, 
as is shown in \figref{R3rad}.

At the yet higher pressure of \unit{500}\kilo\pascal\ there is a small surprise.
As before the original Keller-Miksis model predicts that the bubble wall will collapse at speeds that cannot be measured acoustically,
and as before the acoustically-observed-Keller-Miksis model assigns the  spatial and temporal locations according to the pulse-echo definitions.
The surprise, 
however, is in \Figref{R5rad}.
In \figref{R5rad} the predicted radial response of the bubble looks very similar with both models;
the maximal radii are in good agreement, as are the  times at which the bubble reaches its minimal radii.
All this despite the large differences in the predicted velocity, \figref{R5vel}.

A clue as to why the radial response at some pressures looks very similar for both models (\figref{R1rad} and \figref{R5rad}),
while at other pressures it looks very different (\figref{R3rad}) 
- apparently without any obvious correlation with the speed of the bubble wall -
is found in the predicted scattering cross section of the two models.

\subsection{ The scattering cross section}\label{sec:measurement:scatteringxs}
The scattering cross section, $\sigma$, is found from the ratio of the  emitted acoustic power to the incident intensity, \cite{ShutilovBook}
\begin{align}
  \sigma(\omega) = 4\pi r^2 \oint \frac{ \lr{p(r, t) a(t)}^2}{p_i(r,t)^2} dt,\label{eqn:sigmaff}
\end{align}
where $p$ is the emitted pressure, $p_i$ is the incident pressure (a plain wave of frequency $\omega$, and $a$ is the bubble radius.
In the acoustic far field the pressure emitted by the bubble\cite{Howe2003} is 
\begin{align}
p(r, t) =  \frac{a^2 \ddot a+  2a {\dot a}^2}{r^2}, \label{eqn:pff}
\end{align}
and so the $r^2$ dependence of \eqnref{sigmaff} and \eqnref{pff} cancels.
The scattering cross section may be normalised by dividing out the area of the bubble at equilibrium, $4\pi a_e^2$.


The scattering cross section is only well defined for an incident planar wave.
The period of the emitted wave can be different from period of the incident wave.
The integral is carried out over the
time period where both incident and emitted waves are stable - the closed integral sign being an mnemonic of this.
This occurs when both the imaging and emitted wave oscillate an integer number of times within the period.
However, this can make the scattering cross section hard to evaluate, 
for such a period may not exist (the ratio of the periodicity of the incident and emitted waves may be irrational),
or else may be very long, and hence hard to find numerically.
%Furthermore, it is not applicable to the short pulses that are used in diagnostic ultrasound.
%The integral in \eqnref{sigmaff} is therefore evaluated over the shortest time 
%that is an integer multiple of the period of the incident and emitted sound waves.
%Such a interval is not always finite, however;
%the ratio of the periodicity of the incident and emitted waves may be irrational.


Since the temporal and spatial dependence of the scattering cross section are integrated out,
the scattering cross section is not expected to be dependant upon the measurement process.
On the other hand, however,
the scattering cross section is dependent upon the bubble wall's measured radius, velocity and acceleration
in \eqnref{pff} -
and therefore the two models will give different answers when $M^2$ is not small.
We note, however, that the original Keller-Miksis model, unlike the acoustically measured version,
never claimed to be accurate in presence of high velocities.
The scattering cross section should therefore be computed using the acoustically measured theory.


To find the scattering cross section  numerically 
a finite incident sinusoid must be used to drive the oscillation,
and the incident number of cycles must be sufficient for the transient response to dampen.
In this chapter, we use 750 cycles.
To evaluate whether the response has stabilised
 a 12 cycle section of the radial response  (i.e. from cycle 738 to 750) is chosen as a reference 
and the cross-correlation of the this 12-cycle segment is evaluated with previous 12-cycle segments.
When the average of the cross-correlation coefficients is to within \unit{0.1}\% of the average of the autocorrelation coefficients,
we consider that the bubble response is sufficiently stable
and that the ratio of the periodicities of the incident and driving wave are sufficiently close to being integer.
The scattering cross section is then evaluated over the segment.
This procedure will fail if 
\begin{enumerate}
\item the transient response has not been sufficiently damped.
\item the ratio of the periodicities of the incident and driving wave wave cannot be expressed in 12 cycles
(i.e. the super-harmonic ratio is not  1/6, 1/3,1/2, 2/3, 1, 2, 3/2, 3, 6).
\end{enumerate}
If the procedure fails then then the search is abandoned and the scattering cross section is not given.
While it is possible for more super-harmonics to be searched for,
this process cannot go on forever, for irrational super-harmonics will never be found.

The normalised scatting cross section of a two micron bubble
as a function of the incident frequency is plotted in  \figref{nxs_high}.
%The pressures of the incident wave are low in \figref{nxs_low} (\unit{10}\kilo\pascal\ and \unit{100}\kilo\pascal)
%and are high (\unit{300}\kilo\pascal\ and \unit{500}\kilo\pascal) in  \figref{nxs_high}.
Below the figures it is 
%\figref{nxs_low1}, \figref{nxs_high3} and \figref{nxs_high5} is 
plotted  where the scattering cross section could not be evaluated.

%In \figref{nxs_low01} the scattering cross section could always be found it is not misleading to interpolate between the evaluated points and this is done.

 \begin{figure}[p!]
      \centering
     \subfloat[\unit{10}\kilo\pascal]{
           \label{fig:nxs_low01}
           \includegraphics{xs_freq_r_2_amp_0.01.0}}\\
      \subfloat[\unit{100}\kilo\pascal]{
           \label{fig:nxs_low1}
           \includegraphics{xs_freq_r_2_amp_0.1.1}}
      \label{fig:nxs_low} 
 \caption{The calculated normalised scattering cross section as a function of frequency evaluated at various pressures as measured optically (using the Keller-Miksis model) and acoustically (using the acoustically-measured-Keller-Miksis model). The bubble has an equilibrium radius of \unit{2}\micro\metre. 
Below the graphs the frequencies at which the scattering cross section could not be evaluated is plotted.
The small vertical axis of this plot is meaningless, it is used to help convey the density of points.
}
 \end{figure}
\begin{figure}[p!]
      \centering
 \subfloat[\unit{300}\kilo\pascal]{
           \label{fig:nxs_high3}
           \includegraphics{xs_freq_r_2_amp_0.2.2}}\\
 \subfloat[\unit{500}\kilo\pascal]{
           \label{fig:nxs_high5}
           \includegraphics{xs_freq_r_2_amp_0.5.5}}
      \caption{The calculated normalised scattering cross section as a function of frequency evaluated at various pressures as measured optically (using the Keller-Miksis model) and acoustically (using the acoustically-measured-Keller-Miksis model). The bubble has an equilibrium radius of \unit{2}\micro\metre. 
Below the graphs the frequencies at which the scattering cross section could not be evaluated is plotted.
The small vertical axis of this plot is meaningless, it is used to help convey the density of points.
}
      \label{fig:nxs_high}
 \end{figure}



%The pressures used  are the same  as were used in \figref{BubbleResponse},
%except that here the very low pressure of \unit{10}\kilo\pascal\ is also considered.
At \unit{10}\kilo\pascal\ the  bubble's response is nearly linear when evaluated with both models,
and \figref{nxs_low01} exhibits a large resonance peak near to \unit{2}\mega\hertz  -
which is familiar from linear studies on bubble response\cite{Hoff2001}.
At the higher pressure of  \unit{100}\kilo\pascal, drawn in \figref{nxs_low1}, the fundamental resonance occurs at lower frequency, the bubble also responds when pulsated at the first harmonic and at fractions of the fundamental. 
The response of the two models at  \unit{100}\kilo\pascal\ is essentially identical, 
as was seen for the same pressure in  \figref{R1rad} and \figref{R1vel}.

At the higher pressures shown in \figref{nxs_high} differences do emerge between the responses predicted by the two models.
Considering  \figref{nxs_high3} first, 
we find that the fundamental again occurs at a lower frequency.
Above this resonance the scattering cross section predicted by the two models is essentially identical.
Near the resonance it becomes  hard to evaluate the scattering cross-section, and there is a large drop-out in returned values.
Such dropouts occur when unusual harmonics are present (or developing) within the bubble's response.
Below the resonance the scattering cross section for both models becomes more stable (with few dropouts)
but the scattering cross section evaluated using the acoustically-measured model is systematically 
lower than for when the scattering cross section is evaluated from the original model.
These observations are repeated in \figref{nxs_high5}, except that the discrepancy seems to begin at a harmonic of the fundamental.

\Figref{nxs_high} may now be used to give an explanation the observations of \figref{BubbleResponse}.
The divergence in the predicted scattering cross-section occur near resonance, 
where the role of harmonics (and therefore the dropout) is strong.
%The frequency used to pulsate the bubbles in \figref{BubbleResponse} was \unit{1/2}\mega\hertz.
%In \figref{nxs_high3} \unit{1/2}\mega\hertz\ is seen to occur in the region of the graph near resonance where 
%the scatting cross section is hard to evaluate.
%This suggests that the differences observed in the response of \figref{R3rad} 
%are due to differences in the harmonic response of the two equations.
%In \figref{nxs_high5}, \unit{1/2}\mega\hertz\ is well below the dropout region, 
%suggesting that the harmonics have little role.
%This would explain why the radial response appeared so similar.
%The difference in the scattering cross section between the two models in \figref{nxs_high5} 
%would therefore seem to result from the differing velocity and acceleration that is measured by the two models,
%which manifests itself in the scattering cross via the $\dot a$ and $\ddot a$ terms in equation \eqnref{pff}.



\section{Discussion}

In this chapter we have derived for the first time the pulsations of a bubble as they would be measured with ultrasound.
The model is based upon the Keller-Miksis equation.
The distinction between what was measured by ultrasound and what actually occurs as measured by a microscope
evaporates when the measured velocities are small in comparison to the velocity of sound.
At low speeds then, the acoustically-measured-Keller-Miksis equation equates to the original.


%Indeed, the original Keller-Miksis equation can be obtained by approximating the acoustically-measured version -
%valid for when velocity of the bubble wall is small in comparison to the sound speed.

%The derivation followed from noting that for ultrasound to measure distances (using pulse-echo) 
%the sound speed in the medium must be known {\em a priori}, 
%with no possibility of measuring variations in this sound speed.
%This has two consequences:
%\begin{enumerate}
%  \item The sound speed must be measured to be a constant.
%  \item The sound must be measured to be propagating linearly.
%\end{enumerate}
%If invariance to inertial translations is also assumed,
%then the constancy of the sound speed  implies that 
%ultrasound is subject to the considerations of special relativity,
%with the sound speed taking the role of the speed of light.
%This invariance has been assumed here.

%To describe the propagation of the sound we considered an ideal fluid medium.
%When measured acoustically the relativistic (Lorentz invariant) description of the ideal fluid should be used,
%with the additional constraint that the speed of sound equal the speed of light.
%We have shown that the sound does indeed propagate linearly in this case.
%Indeed, we have shown it obeys Maxwell's relations.

Finally, the response of the acoustically-measured-Keller-Miksis equation to an acoustic wave has been simulated,
and the results compared to the original Keller-Miksis equation.
The radial pulsations observed by the two models is similar when the harmonic response of the bubble is not strong -
otherwise the pulsations become quite different.
The velocity response for the two models diverges when the bubble wall speed is high.
The acoustically-measured-Keller-Miksis equation correctly maintains
that ultrasound cannot measure supersonic bubble collapse, 
and predicts what is measured acoustically when such collapses do occur.

When the bubble wall moves at speeds close or exceeding the sound speed,
differences are found between the scattering cross section obtained from the two models.
The acoustically-measured-Keller-Miksis model claims accuracy in this regime,
\hl{not in the sense that it models the actual pulsations of the bubble,
but rather that it models the motion of the bubble wall that are consistent with the other measurements of ultrasound}.
%The original does not.
%The scattering cross section predictions therefore
%provide a means of testing the acoustically-measured-Keller-Miksis equation directly.




% LocalWords:  Miksis Bamber Shorrock spatio radarDistance WaveEqn BernoulliNR
% LocalWords:  VelocityPotNR nc spacetime RelVPot mh VelocityPot Ansatz wv RKM
% LocalWords:  dtAnsatz drAnsatz IncidentWave phiDefn EMtensor barotropic defnA
% LocalWords:  EMFluid wu Taub dn timelike eomTime spacelike eomSpace bivector
% LocalWords:  DefnVorticity KMoriginal pressureBC polytropic nrootp dp rr tt
% LocalWords:  RelStress sigmarr RelStressTrace htp XSfreq XS Mie


%%% Local Variables: 
%%% mode: latex
%%% TeX-master: "../../tshorrock_thesis"
%%% End: 
