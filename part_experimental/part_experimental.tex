
\part{Experimental}\label{part:experimental}

\begin{quote}

\end{quote}



In \partref{theoretical} it was shown, with a computational model,
that a low frequency wave can alter the response of a bubble to a higher frequency pulse.
The lower frequency wave, the cavitating wave,
pulsates the bubble and induces transitory changes in its response to a higher frequency, imaging pulse.

This model is tested experimentally in this part of the thesis.
\Chapref{rationale} introduces the principle design decisions guiding the experiments.
\Chapsref{DPR500_protocol}-\ref{perfluoropentane_protocol}
then describe the experiments performed on three different bubble models: 
bubbles induced from Type II and Type III \todo{check these types} cavitiation of water,
commercial microbubbles (\Sonovue),
and bubbles generated by Type I\todo{check this type} cavitatioin of perfluoropentane.


%%% Local Variables: 
%%% mode: latex
%%% TeX-master: "../tshorrock_thesis"
%%% End: 
